%% Generated by Sphinx.
\def\sphinxdocclass{report}
\documentclass[letterpaper,10pt,english]{sphinxmanual}
\ifdefined\pdfpxdimen
   \let\sphinxpxdimen\pdfpxdimen\else\newdimen\sphinxpxdimen
\fi \sphinxpxdimen=.75bp\relax
\ifdefined\pdfimageresolution
    \pdfimageresolution= \numexpr \dimexpr1in\relax/\sphinxpxdimen\relax
\fi
%% let collapsible pdf bookmarks panel have high depth per default
\PassOptionsToPackage{bookmarksdepth=5}{hyperref}

\PassOptionsToPackage{booktabs}{sphinx}
\PassOptionsToPackage{colorrows}{sphinx}

\PassOptionsToPackage{warn}{textcomp}
\usepackage[utf8]{inputenc}
\ifdefined\DeclareUnicodeCharacter
% support both utf8 and utf8x syntaxes
  \ifdefined\DeclareUnicodeCharacterAsOptional
    \def\sphinxDUC#1{\DeclareUnicodeCharacter{"#1}}
  \else
    \let\sphinxDUC\DeclareUnicodeCharacter
  \fi
  \sphinxDUC{00A0}{\nobreakspace}
  \sphinxDUC{2500}{\sphinxunichar{2500}}
  \sphinxDUC{2502}{\sphinxunichar{2502}}
  \sphinxDUC{2514}{\sphinxunichar{2514}}
  \sphinxDUC{251C}{\sphinxunichar{251C}}
  \sphinxDUC{2572}{\textbackslash}
\fi
\usepackage{cmap}
\usepackage[T1]{fontenc}
\usepackage{amsmath,amssymb,amstext}
\usepackage{babel}



\usepackage{tgtermes}
\usepackage{tgheros}
\renewcommand{\ttdefault}{txtt}



\usepackage[Bjarne]{fncychap}
\usepackage{sphinx}

\fvset{fontsize=auto}
\usepackage{geometry}


% Include hyperref last.
\usepackage{hyperref}
% Fix anchor placement for figures with captions.
\usepackage{hypcap}% it must be loaded after hyperref.
% Set up styles of URL: it should be placed after hyperref.
\urlstyle{same}


\usepackage{sphinxmessages}
\setcounter{tocdepth}{1}



\title{pylovo}
\date{April 12, 2024}
\release{2024}
\author{Beneharo Reveron Baecker}
\newcommand{\sphinxlogo}{\vbox{}}
\renewcommand{\releasename}{Release}
\makeindex
\begin{document}

\ifdefined\shorthandoff
  \ifnum\catcode`\=\string=\active\shorthandoff{=}\fi
  \ifnum\catcode`\"=\active\shorthandoff{"}\fi
\fi

\pagestyle{empty}
\sphinxmaketitle
\pagestyle{plain}
\sphinxtableofcontents
\pagestyle{normal}
\phantomsection\label{\detokenize{index::doc}}


\begin{DUlineblock}{0em}
\item[] Our climate goals and current political events more than ever demonstrate the need for transitioning towards locally sourced,
environmentally less harmful heating solutions.
This transition is not easily made, however: building structure, existing heating and availability of renewable electricity vary wildly in
different parts of the country and often even within the same cities. Due to this heterogeneity, improvements such as housing renovation
or new heating technologies need to be tailored to specific districts and their individual needs. At the same time, it is important to
keep a view of the big picture, to avoid different districts working at cross\sphinxhyphen{}purposes.
\end{DUlineblock}

\begin{DUlineblock}{0em}
\item[] To look at these problems the research association “Energy \sphinxhyphen{} Sector Coupling and Micro\sphinxhyphen{}Grids”, or “STROM” for short,
is developing a digital automatized energy supply planning tool to rapidly advance the transformation of the energy system.
The tool will be able to use available local data including renewable resource potentials, distribution grid topology and energy demand
to help with the planning of power and heat supply structures. The tool is built on two central components: syngrid, a module for
generating synthetic distribution grids for a free\sphinxhyphen{}selected research area as well as the optimization framework urbs, a linear programming
model for multi\sphinxhyphen{}commodity energy systems with a focus on optimal storage sizing and use.
\end{DUlineblock}

\begin{DUlineblock}{0em}
\item[] While the individual components of this tool had already existed prior to the start of this IDP, it was the student’s responsibility to
combine them in a graphical application that allows a user to quickly visualize different grids and set parameters as well as results
of the optimization of said grids.
\end{DUlineblock}


\chapter{pylovo (python tool for low\sphinxhyphen{}voltage distribution grid generation)}
\label{\detokenize{index:pylovo-python-tool-for-low-voltage-distribution-grid-generation}}
\sphinxAtStartPar
This tool provides a comprehensive public\sphinxhyphen{}data\sphinxhyphen{}based module to generate synthetic distribution grid for a
freely\sphinxhyphen{}selected research area. The main data input is the buildings, roads and transformers geographic data obtained
from OpenStreetMap, with additional auxiliary datasets including postal code area polygons (to identify and select
research areas), consumer categories (to estimate loading performances of different types of buildings and households)
and infrastructure parameters, etc. The result outputs a feasible solution of aggregated distribution grid networks
within the research scope and automatically analyses the important grid statistics such that users to evaluate the
general grid properties.


\section{Preamble}
\label{\detokenize{index:preamble}}\begin{enumerate}
\sphinxsetlistlabels{\arabic}{enumi}{enumii}{}{.}%
\item {} 
\sphinxAtStartPar
This tool focuses far only on distribution grid level. The result is presented at a collection of local grids
where the transformer is connected to a constant external grid as the transmission level conjunction. For
consumers, over\sphinxhyphen{}sized loads could be supplied directly by medium voltage level grid or equiped with individual
transformers. Those loads are regarded as ‘large consumers’ and would not be presented in final result graphics (but
will be analysed in statistics).

\item {} 
\sphinxAtStartPar
All the geographic data during the process is presented in PostGIS Geometry format where a Spatial Reference System
should be defined to recognize the exact positions. By default, the SRID is selected as epsg:3035 since here the
basic unit of geographic analysis is meter. When transforming to pandapower network, the geodata should be presented
as lon/lat coordination which is epsg:4326 (automatically done in script), be careful of the two different reference
system settings;

\item {} 
\sphinxAtStartPar
The various input data obtained from different public sources do not always align in terms of available scopes, for
example, Open Street Map provides public construction data for a global scale, while the source of postal code areas
are limited within Germany, also certain regulations and parameters could differ when the research comes to another
country. Therefore, a default research scope so far for this model is Germany and will be extended to higher levels.

\item {} 
\sphinxAtStartPar
Due to complicated geographical situations, this tool so far CAN NOT guarantee 100\% accuracy under all circumstances,
please let us know what error occurs when you are using this model for your applications. We appreciate your
comments!

\end{enumerate}


\section{Repository structure}
\label{\detokenize{index:repository-structure}}
\begin{sphinxVerbatim}[commandchars=\\\{\}]
pylovo/
├───docs                                        contains sphinx documentation files
│   ├───build                                   contains html files after local doc generation
│   └───source                                  contains documentation source files
│       ├───docs\PYGZus{}gui                            describes usage, API of the gui
│       ├───docs\PYGZus{}pylovo                         describes usage of pylovo
│       ├───docs\PYGZus{}sphinx                         describes setup of documentation
│       ├───images
│       ├───\PYGZus{}static
│       └───\PYGZus{}templates
├───gui
│   ├───fig
│   ├───IDP\PYGZus{}MapTool\PYGZus{}Flask
│   │   ├───instance
│   │   ├───maptool                             main gui code folder
│   │   │   ├───network\PYGZus{}editor                  js code for network editor window
│   │   │   ├───postcode\PYGZus{}editor                 js code for postcode editor window
│   │   │   ├───static
│   │   │   ├───templates
│   │   │   ├───urbs\PYGZus{}editor                     js code for urbs setup window
│   │   │   ├───urbs\PYGZus{}results                    js code for urbs results window
│   │   │   ├───z\PYGZus{}feature\PYGZus{}jsons                 contains json files defining variables, their types, default values
│   │   │   │   ├───pandapower\PYGZus{}network\PYGZus{}features
│   │   │   │   └───urbs\PYGZus{}setup\PYGZus{}features
│   │   ├───pandapower2urbs                     code converting pdp net into urbs input file
│   │   ├───pandapower2urbs\PYGZus{}dataset\PYGZus{}template    excel files containing default values for urbs setup
│   │   └───results
│   └───urbs                                    code for the urbs optimization model
│       └───urbs\PYGZus{}result                         result data of successful urbs runs
├── examples/ :                                 Example usage of the tool
│   ├── basic\PYGZus{}grid\PYGZus{}generation.ipynb
│   └── map\PYGZus{}visualization\PYGZus{}examples.ipynb
├── raw\PYGZus{}data/ :                                 Part of the initially required data
├── results/ :                                  Folder for results
├── syngrid/ :                                  Python package
│   ├── \PYGZus{}\PYGZus{}init\PYGZus{}\PYGZus{}.py :                           Package requirement
│   ├── config\PYGZus{}data.py :                        Configuration about Data IO
│   ├── config\PYGZus{}version.py :                     Version dependent configuration
│   ├── dump\PYGZus{}functions.sql :                    SQL functions to create
│   ├── GridGenerator.py :                      Main class with grid generation methods
│   ├── pgReaderWriter.py :                     Class for database communication
│   ├── SyngridDatabaseConstructor.py :         Class to setup a syngrid database
│   └── utils.py :                              Class independent helper functions
├── main\PYGZus{}constructor.py :                       Sample script to construct a syngrid database
├── main\PYGZus{}grid\PYGZus{}generation.py :                   Sample script to generate grid in a region
└── setup.py :                                  Python package configuration
\end{sphinxVerbatim}

\sphinxstepscope


\subsection{Installation}
\label{\detokenize{docs_pylovo/installation:installation}}\label{\detokenize{docs_pylovo/installation::doc}}

\subsubsection{Prerequisites}
\label{\detokenize{docs_pylovo/installation:prerequisites}}\begin{itemize}
\item {} 
\sphinxAtStartPar
\sphinxhref{https://www.anaconda.com/}{Anaconda}: We strongly recommend setting up virtual environments for pylovo and the urbs optimizer.

\item {} 
\sphinxAtStartPar
\sphinxhref{https://doku.lrz.de/vpn-eduvpn-installation-und-konfiguration-11491448.html?showLanguage=en\_GB}{EduVPN}: If you are working from your own machine you will need a VPN to connect to the MWN network. A connection to the network is necessary for accessing the database server.

\end{itemize}


\subsubsection{Install package}
\label{\detokenize{docs_pylovo/installation:install-package}}
\begin{DUlineblock}{0em}
\item[] pylovo is developed as a Python package and can be installed with pip, ideally by using a virtual environment.
\item[] Start by cloning the repository from Github to a directory of your choice. The following command pulls the repository as well as the urbs submodule
\end{DUlineblock}

\begin{sphinxVerbatim}[commandchars=\\\{\}]
\PYG{n}{git} \PYG{n}{clone} \PYG{o}{\PYGZhy{}}\PYG{o}{\PYGZhy{}}\PYG{n}{recurse}\PYG{o}{\PYGZhy{}}\PYG{n}{submodules} \PYG{o}{\PYGZhy{}}\PYG{o}{\PYGZhy{}}\PYG{n}{remote}\PYG{o}{\PYGZhy{}}\PYG{n}{submodules} \PYG{n}{https}\PYG{p}{:}\PYG{o}{/}\PYG{o}{/}\PYG{n}{github}\PYG{o}{.}\PYG{n}{com}\PYG{o}{/}\PYG{n}{tum}\PYG{o}{\PYGZhy{}}\PYG{n}{ens}\PYG{o}{/}\PYG{n}{pylovo}\PYG{o}{.}\PYG{n}{git}
\end{sphinxVerbatim}

\sphinxAtStartPar
Next we set up our virtual environments. Open the anaconda powershell prompt and begin

\begin{sphinxVerbatim}[commandchars=\\\{\}]
\PYG{n}{cd} \PYG{n}{path}\PYG{o}{/}\PYG{n}{to}\PYG{o}{/}\PYG{n}{repo}\PYG{o}{/}\PYG{n}{pylovo}\PYG{o}{/}\PYG{n}{gui}\PYG{o}{/}\PYG{n}{IDP\PYGZus{}Maptool\PYGZus{}Flask}
\PYG{n}{conda} \PYG{n}{env} \PYG{n}{create} \PYG{o}{\PYGZhy{}}\PYG{n}{f} \PYG{n}{environment}\PYG{o}{.}\PYG{n}{yml}
\PYG{n}{conda} \PYG{n}{env} \PYG{n}{create} \PYG{o}{\PYGZhy{}}\PYG{n}{f} \PYG{n}{urbs310}\PYG{o}{.}\PYG{n}{yml}
\end{sphinxVerbatim}

\sphinxAtStartPar
You can test whether the environments have been properly created via

\begin{sphinxVerbatim}[commandchars=\\\{\}]
\PYG{c+c1}{\PYGZsh{}to enter a virtual environment}
\PYG{n}{conda} \PYG{n}{activate} \PYG{n}{TUM\PYGZus{}Syngrid}
\PYG{n}{conda} \PYG{n}{activate} \PYG{n}{urbs310}
\PYG{c+c1}{\PYGZsh{}to leave a virtual environment:}
\PYG{n}{conda} \PYG{n}{deactivate}
\end{sphinxVerbatim}

\sphinxAtStartPar
Finally we install pylovo from the local repository. The developer option installs additional packages.
It is crucial that you enter the newly created conda environment before installing pylovo.


\paragraph{User}
\label{\detokenize{docs_pylovo/installation:user}}
\begin{sphinxVerbatim}[commandchars=\\\{\}]
\PYG{c+c1}{\PYGZsh{}activate environment}
\PYG{n}{conda} \PYG{n}{activate} \PYG{n}{TUM\PYGZus{}Syngrid}
\PYG{c+c1}{\PYGZsh{}navigate to pylovo main code folder}
\PYG{n}{cd} \PYG{n}{path}\PYG{o}{/}\PYG{n}{to}\PYG{o}{/}\PYG{n}{git\PYGZus{}repo}\PYG{o}{/}\PYG{n}{pylovo}
\PYG{n}{pip} \PYG{n}{install} \PYG{o}{\PYGZhy{}}\PYG{n}{e} \PYG{o}{.}
\end{sphinxVerbatim}


\paragraph{Developer}
\label{\detokenize{docs_pylovo/installation:developer}}
\begin{sphinxVerbatim}[commandchars=\\\{\}]
\PYG{c+c1}{\PYGZsh{}activate environment}
\PYG{n}{conda} \PYG{n}{activate} \PYG{n}{TUM\PYGZus{}Syngrid}
\PYG{c+c1}{\PYGZsh{}navigate to pylovo main code folder}
\PYG{n}{cd} \PYG{n}{path}\PYG{o}{/}\PYG{n}{to}\PYG{o}{/}\PYG{n}{git\PYGZus{}repo}\PYG{o}{/}\PYG{n}{pylovo}
\PYG{n}{pip} \PYG{n}{install} \PYG{o}{\PYGZhy{}}\PYG{n}{e} \PYG{o}{.}\PYG{p}{[}\PYG{n}{dev}\PYG{p}{]}
\end{sphinxVerbatim}

\sphinxAtStartPar
And with that pylovo should be ready to go! You can test whether everything went correctly by navigating
to the IDP\_Maptool\_Flask folder and running the following commands

\begin{sphinxVerbatim}[commandchars=\\\{\}]
\PYG{c+c1}{\PYGZsh{}activate environment}
\PYG{n}{conda} \PYG{n}{activate} \PYG{n}{TUM\PYGZus{}Syngrid}
\PYG{c+c1}{\PYGZsh{}navigate to maptool folder and start web server}
\PYG{n}{cd} \PYG{n}{path}\PYG{o}{/}\PYG{n}{to}\PYG{o}{/}\PYG{n}{repo}\PYG{o}{/}\PYG{n}{pylovo}\PYG{o}{/}\PYG{n}{gui}\PYG{o}{/}\PYG{n}{IDP\PYGZus{}Maptool\PYGZus{}Flask}
\PYG{n}{flask} \PYG{o}{\PYGZhy{}}\PYG{o}{\PYGZhy{}}\PYG{n}{app} \PYG{n}{maptool} \PYG{o}{\PYGZhy{}}\PYG{o}{\PYGZhy{}}\PYG{n}{debug} \PYG{n}{run}
\end{sphinxVerbatim}

\sphinxAtStartPar
This should start the flask server and allow you to open the GUI by navigating to \sphinxurl{http:127.0.0.1:5000} in a webbrowser of your choice


\subsubsection{Advanced installation \sphinxhyphen{} Database construction}
\label{\detokenize{docs_pylovo/installation:advanced-installation-database-construction}}
\begin{DUlineblock}{0em}
\item[] Follow the instructions below, only if you want to create a new database for pylovo.
Make sure you have the required raw data.
\end{DUlineblock}

\begin{DUlineblock}{0em}
\item[] Initial steps to create a PostrgeSQL database on ENS virtual machine and connect to the db from local computer are listed below.
\end{DUlineblock}


\subsubsection{Install postgresql on linux}
\label{\detokenize{docs_pylovo/installation:install-postgresql-on-linux}}
\begin{DUlineblock}{0em}
\item[] Since arbitrary package installation can be problematic due to the user rights,
postgresql can be installed inside a conda environment. The instruction at the following link should be sufficient to create and run a database.
\item[] \sphinxurl{https://gist.github.com/gwangjinkim/f13bf596fefa7db7d31c22efd1627c7a}
\end{DUlineblock}


\subsubsection{Postgis \& PGRouting}
\label{\detokenize{docs_pylovo/installation:postgis-pgrouting}}
\begin{DUlineblock}{0em}
\item[] The postgis extension has to be installed via conda as well. The extension can only be created by the base user
\end{DUlineblock}

\begin{sphinxVerbatim}[commandchars=\\\{\}]
\PYG{n}{conda} \PYG{n}{install} \PYG{o}{\PYGZhy{}}\PYG{n}{c} \PYG{n}{conda}\PYG{o}{\PYGZhy{}}\PYG{n}{forge} \PYG{n}{postgis}
\PYG{n}{conda} \PYG{n}{install} \PYG{o}{\PYGZhy{}}\PYG{n}{c} \PYG{n}{conda}\PYG{o}{\PYGZhy{}}\PYG{n}{forge} \PYG{n}{pgrouting}
\PYG{c+c1}{\PYGZsh{} TODO hstore?}
\end{sphinxVerbatim}


\subsubsection{Access Database}
\label{\detokenize{docs_pylovo/installation:access-database}}

\paragraph{Outside the server (from ssh client a.k.a. your own computer)}
\label{\detokenize{docs_pylovo/installation:outside-the-server-from-ssh-client-a-k-a-your-own-computer}}
\begin{DUlineblock}{0em}
\item[] To gain access to the pylovo database from your own machine you will need to request a username and password from the ENS chair.
\item[] If you are working from your own machine you will also need to utilize a VPN to connect to the MWN network,
which us a prerequisite for connecting to the database server. We recommend using \sphinxhref{https://doku.lrz.de/vpn-eduvpn-installation-und-konfiguration-11491448.html?showLanguage=en\_GB}{EduVPN} for this purpose.
Follow the instructions in the link to set up a connection. You will need to use EduVPN to connect to the profile
\sphinxstylestrong{Technische Universität München via LRZ\sphinxhyphen{}VPN}.
\end{DUlineblock}

\begin{DUlineblock}{0em}
\item[] Once you have connected via EduVPN the tool will be able to connect to the database automatically
\end{DUlineblock}


\subsubsection{Create SQL functions}
\label{\detokenize{docs_pylovo/installation:create-sql-functions}}
\sphinxAtStartPar
Prewritten SQL functions must be created for once, when the database is created. Run the file syngrid/dump\_functions.sql:

\begin{sphinxVerbatim}[commandchars=\\\{\}]
\PYG{n}{psql} \PYG{o}{\PYGZhy{}}\PYG{n}{d} \PYG{n}{syngrid\PYGZus{}db} \PYG{o}{\PYGZhy{}}\PYG{n}{a} \PYG{o}{\PYGZhy{}}\PYG{n}{f} \PYG{l+s+s2}{\PYGZdq{}}\PYG{l+s+s2}{syngrid/dump\PYGZus{}functions.sql}\PYG{l+s+s2}{\PYGZdq{}}
\end{sphinxVerbatim}


\subsubsection{Load raw data to the database}
\label{\detokenize{docs_pylovo/installation:load-raw-data-to-the-database}}
\begin{DUlineblock}{0em}
\item[] pylovo requires the correct table structure and input data to already be loaded into the database.
Make sure that you have the raw data files and paths configured in config\_data.py
\end{DUlineblock}

\begin{DUlineblock}{0em}
\item[] Afterwards, the ETL process can be executed as:
\end{DUlineblock}

\begin{sphinxVerbatim}[commandchars=\\\{\}]
\PYG{n}{python} \PYG{n}{main\PYGZus{}constructor}\PYG{o}{.}\PYG{n}{py}
\end{sphinxVerbatim}


\subsubsection{Input data model}
\label{\detokenize{docs_pylovo/installation:input-data-model}}
\sphinxAtStartPar
The minimum data model is described below:
\begin{itemize}
\item {} 
\sphinxAtStartPar
res

\item {} 
\sphinxAtStartPar
oth

\item {} 
\sphinxAtStartPar
betriebsmittel

\item {} 
\sphinxAtStartPar
postcode

\item {} 
\sphinxAtStartPar
ways

\item {} 
\sphinxAtStartPar
consumer\_categories

\item {} 
\sphinxAtStartPar
transformers

\end{itemize}


\paragraph{Preprocess ways from OSM data}
\label{\detokenize{docs_pylovo/installation:preprocess-ways-from-osm-data}}\begin{enumerate}
\sphinxsetlistlabels{\arabic}{enumi}{enumii}{}{.}%
\item {} 
\sphinxAtStartPar
Connect to database via localhost

\item {} 
\sphinxAtStartPar
Download the OSM\sphinxhyphen{}streetnets you require from \sphinxurl{http://download.geofabrik.de/}

\item {} \begin{description}
\sphinxlineitem{Download Osm2po\sphinxhyphen{}5.3.6 from \sphinxurl{https://osm2po.de/releases/}}\begin{itemize}
\item {} 
\sphinxAtStartPar
!!!Has to be version 5.3.6, this guide does not work with later versions!!!

\end{itemize}

\end{description}

\item {} 
\sphinxAtStartPar
Extract the downloaded zip file

\item {} \begin{description}
\sphinxlineitem{Open the osm2po.config file in the extracted folder and make sure that all of the following lines are set correctly (lines starting with \# are commented out)}\begin{itemize}
\item {} 
\sphinxAtStartPar
Line 59:          tilesize=x

\item {} 
\sphinxAtStartPar
Line 190:         comment out “.default.wtr.finalMask = car”

\item {} 
\sphinxAtStartPar
Line 222\sphinxhyphen{}231:     make sure that only ferry is commented out

\item {} 
\sphinxAtStartPar
Line 341:         line must not be commented out, otherwise sql file will not be generated

\end{itemize}

\end{description}

\item {} \begin{description}
\sphinxlineitem{Open terminal and navigate to folder Osm2po\sphinxhyphen{}5.3.6. Execute the following command:}\begin{itemize}
\item {} 
\sphinxAtStartPar
java \sphinxhyphen{}Xmx1g \sphinxhyphen{}jar osm2po\sphinxhyphen{}core\sphinxhyphen{}5.3.6\sphinxhyphen{}signed.jar prefix=public “C:/Users/path/to/osm/file/osm\_file\_name.pbf”

\item {} 
\sphinxAtStartPar
change „C:/Users/path/to/osm/file/“ with path to geofabrik file you downloaded earlier

\item {} 
\sphinxAtStartPar
change „osm\_file\_name.pbf“ to name of the geofabrik file

\end{itemize}

\end{description}

\item {} \begin{description}
\sphinxlineitem{Navigate to newly created folder “public” and execute following command in the terminal:}\begin{itemize}
\item {} 
\sphinxAtStartPar
psql \sphinxhyphen{}U syngrid \sphinxhyphen{}d syngrid\_db \sphinxhyphen{}h localhost \sphinxhyphen{}p 1111 \sphinxhyphen{}f .public\_2po\_4pgr.sql

\end{itemize}

\end{description}

\item {} \begin{description}
\sphinxlineitem{Execute pylovo’s main\_constructor.py after table 2po\_4pgr has been created in the database}\begin{itemize}
\item {} 
\sphinxAtStartPar
make sure the ways\_to\_db method has been uncommented in main\_constructor.py

\item {} 
\sphinxAtStartPar
the ways in the 2po\_4pgr table will be inserted into the ways table and can now be used by pylovo

\end{itemize}

\end{description}

\end{enumerate}


\subsubsection{Further Input Data}
\label{\detokenize{docs_pylovo/installation:further-input-data}}
\sphinxstepscope


\paragraph{Municipal Register}
\label{\detokenize{docs_pylovo/municipal_register/municipal_register:municipal-register}}\label{\detokenize{docs_pylovo/municipal_register/municipal_register::doc}}
\sphinxAtStartPar
The area for grid generation that can be input by the user is the PLZ Code (Postleitzahl). For the import of the building
dataset the AGS (Amtlicher Gemeindeschlüssel) is needed (see {\hyperref[\detokenize{grid_generation/index::doc}]{\sphinxcrossref{\DUrole{doc}{Grid Generation}}}}).
For the classification, the Regiostar class of that area needs to be provided (see {\hyperref[\detokenize{classification/index::doc}]{\sphinxcrossref{\DUrole{doc}{Classification of Pylovo Grids}}}}).

\sphinxAtStartPar
The municipal register is created when setting up the database of the project by running \sphinxcode{\sphinxupquote{main\_constructor.py}}
where the following function is called:
\index{create\_municipal\_register() (in module municipal\_register.join\_regiostar\_gemeindeverz)@\spxentry{create\_municipal\_register()}\spxextra{in module municipal\_register.join\_regiostar\_gemeindeverz}}

\begin{fulllineitems}
\phantomsection\label{\detokenize{docs_pylovo/municipal_register/municipal_register:municipal_register.join_regiostar_gemeindeverz.create_municipal_register}}
\pysigstartsignatures
\pysiglinewithargsret{\sphinxcode{\sphinxupquote{municipal\_register.join\_regiostar\_gemeindeverz.}}\sphinxbfcode{\sphinxupquote{create\_municipal\_register}}}{}{{ $\rightarrow$ None}}
\pysigstopsignatures
\sphinxAtStartPar
join gemeindeverzeichnis with regiostar, so that each PLZ can be associated with a AGS and regiostar class.
The data is written do the database table ‘municipal\_register’.

\end{fulllineitems}


\sphinxAtStartPar
The data is read by the following functions:


\subparagraph{Gemeindeverzeichnis}
\label{\detokenize{docs_pylovo/municipal_register/municipal_register:gemeindeverzeichnis}}
\sphinxAtStartPar
With the following functions the necessary population data is imported:
\index{import\_plz\_einwohner() (in module municipal\_register.gemeindeverzeichnis.import\_functions)@\spxentry{import\_plz\_einwohner()}\spxextra{in module municipal\_register.gemeindeverzeichnis.import\_functions}}

\begin{fulllineitems}
\phantomsection\label{\detokenize{docs_pylovo/municipal_register/municipal_register:municipal_register.gemeindeverzeichnis.import_functions.import_plz_einwohner}}
\pysigstartsignatures
\pysiglinewithargsret{\sphinxcode{\sphinxupquote{municipal\_register.gemeindeverzeichnis.import\_functions.}}\sphinxbfcode{\sphinxupquote{import\_plz\_einwohner}}}{}{{ $\rightarrow$ DataFrame}}
\pysigstopsignatures
\sphinxAtStartPar
imports table with PLZ, population, area, latitude and longitude,
from source: \sphinxurl{https://www.suche-postleitzahl.org/downloads}
\begin{quote}\begin{description}
\sphinxlineitem{Returns}
\sphinxAtStartPar
table with population data

\sphinxlineitem{Return type}
\sphinxAtStartPar
pd.DataFrame

\end{description}\end{quote}

\end{fulllineitems}

\index{import\_zuordnung\_plz() (in module municipal\_register.gemeindeverzeichnis.import\_functions)@\spxentry{import\_zuordnung\_plz()}\spxextra{in module municipal\_register.gemeindeverzeichnis.import\_functions}}

\begin{fulllineitems}
\phantomsection\label{\detokenize{docs_pylovo/municipal_register/municipal_register:municipal_register.gemeindeverzeichnis.import_functions.import_zuordnung_plz}}
\pysigstartsignatures
\pysiglinewithargsret{\sphinxcode{\sphinxupquote{municipal\_register.gemeindeverzeichnis.import\_functions.}}\sphinxbfcode{\sphinxupquote{import\_zuordnung\_plz}}}{}{{ $\rightarrow$ DataFrame}}
\pysigstopsignatures
\sphinxAtStartPar
imports excel table with matching PLZ and AGS
from source: \sphinxurl{https://www.suche-postleitzahl.org/downloads}
\begin{quote}\begin{description}
\sphinxlineitem{Returns}
\sphinxAtStartPar
table with PLZ and AGS data

\sphinxlineitem{Return type}
\sphinxAtStartPar
pd.DataFrame

\end{description}\end{quote}

\end{fulllineitems}



\subparagraph{Regiostar}
\label{\detokenize{docs_pylovo/municipal_register/municipal_register:regiostar}}
\sphinxAtStartPar
The Regiostar Data is imported like this:
\index{import\_regiostar() (in module municipal\_register.regiostar.import\_regiostar)@\spxentry{import\_regiostar()}\spxextra{in module municipal\_register.regiostar.import\_regiostar}}

\begin{fulllineitems}
\phantomsection\label{\detokenize{docs_pylovo/municipal_register/municipal_register:municipal_register.regiostar.import_regiostar.import_regiostar}}
\pysigstartsignatures
\pysiglinewithargsret{\sphinxcode{\sphinxupquote{municipal\_register.regiostar.import\_regiostar.}}\sphinxbfcode{\sphinxupquote{import\_regiostar}}}{\emph{\DUrole{n}{) \sphinxhyphen{}\textgreater{} (\textless{}class \textquotesingle{}pandas.core.frame.DataFrame\textquotesingle{}\textgreater{}}}, \emph{\DUrole{n}{\textless{}class \textquotesingle{}pandas.core.frame.DataFrame\textquotesingle{}\textgreater{}}}}{}
\pysigstopsignatures
\sphinxAtStartPar
imports RegioStaR dataset from excel datasheet
Regiostar: Regionalstatistische Raumtypologie des Bundesministeriums für Digitales und Verkehr (BMVI)
classification of German Municipalities
source: \sphinxurl{https://bmdv.bund.de/SharedDocs/DE/Artikel/G/regionalstatistische-raumtypologie.html}
\begin{quote}\begin{description}
\sphinxlineitem{Returns}
\sphinxAtStartPar
table with AGS, name, Regiostar 5 and 7 classes

\sphinxlineitem{Return type}
\sphinxAtStartPar
pd.DataFrame

\end{description}\end{quote}

\end{fulllineitems}


\sphinxstepscope


\subsection{Grid Generation}
\label{\detokenize{grid_generation/index:grid-generation}}\label{\detokenize{grid_generation/index::doc}}
\sphinxAtStartPar
The Grid Generation is the center piece of the pylovo Repository. It contains all the steps and logic to generate
the LV\sphinxhyphen{}grids.
\begin{itemize}
\item {} 
\sphinxAtStartPar
A short graphical summary of the grid generation is provided.

\item {} 
\sphinxAtStartPar
The steps of the building data import are explained.

\item {} 
\sphinxAtStartPar
There is a more detailed explanation of the grid generation process.

\end{itemize}


\subsubsection{Contents}
\label{\detokenize{grid_generation/index:contents}}
\sphinxstepscope


\paragraph{Main workflow}
\label{\detokenize{grid_generation/main_workflow/main_workflow:main-workflow}}\label{\detokenize{grid_generation/main_workflow/main_workflow::doc}}

\subparagraph{Main workflow of the model}
\label{\detokenize{grid_generation/main_workflow/main_workflow:main-workflow-of-the-model}}\begin{enumerate}
\sphinxsetlistlabels{\arabic}{enumi}{enumii}{}{.}%
\item {} 
\sphinxAtStartPar
The research scope identification is done by either manually setting the plz code in GridGeneration.py main script or
an automatic search according to the administrative name of the district.

\item {} 
\sphinxAtStartPar
Run GridGeneration.py script, and the process will be proceeded to:
\begin{itemize}
\item {} 
\sphinxAtStartPar
extract correlated buildings, roads and transformers involved in the selected area;

\item {} 
\sphinxAtStartPar
estimate the buildings’ peak load and remove too large consumers (connected directly to medium\sphinxhyphen{}voltage grid)

\item {} 
\sphinxAtStartPar
connect the buildings and transformers to the roads and analyse the network topology, remove isolated components;

\item {} 
\sphinxAtStartPar
according to edge\sphinxhyphen{}distance matrix, assign transformers with corresponding neighboring buildings, regarding cable
length limit and capacity limit;

\item {} 
\sphinxAtStartPar
the remaining unsupplied buildings are subdivided into local distribution grids by hierarchical clustering, with
timely simultaneous peak load validation to determine proper cluster sizes;

\item {} 
\sphinxAtStartPar
the optimal positions of manually grouped distribution grids are determined by minimal power\sphinxhyphen{}distance algorithm,
aiming to minimize the network voltage band, energy losses on conductors and with shorter total cable length;

\end{itemize}

\item {} 
\sphinxAtStartPar
At the end of GridGeneration.py process, the basic nodal elements of all the local distribution grids have been
determined. The installation of cables are determined in Cable\_installation.py with support of pandapower;

\item {} 
\sphinxAtStartPar
The timely power flow calculation shall afterwards be conducted with random assignment of household load profiles,
the default validation period is one year from 00:00:00 01.01.2019 \sphinxhyphen{} 24:00:00 31.12.2019 with time step of 15
minutes, where users can freely shorten or prolong the period or increase the time step by modifying default
parameters and run power\_flow\_calculation.py

\item {} 
\sphinxAtStartPar
In branch result\_analysis presents the plot\_result.py where according to pandapower result from step 3, the grid
generation result will be analysed to multiple perspectives including:
\begin{itemize}
\item {} 
\sphinxAtStartPar
some general overviews of total numbers of transformers, loads, cable length, etc.;

\item {} 
\sphinxAtStartPar
numerical statistics of each size of transformers;

\item {} 
\sphinxAtStartPar
spatial distribution of transformers;

\item {} 
\sphinxAtStartPar
load estimation of household;

\item {} 
\sphinxAtStartPar
spatial detailed picture of a single distribution grid (picked by random index);

\end{itemize}

\end{enumerate}

\sphinxAtStartPar
users can by commenting or uncommenting corresponding codes in plot\_result.py to select the required plots.


\subparagraph{Main workflow of the model (depricated)}
\label{\detokenize{grid_generation/main_workflow/main_workflow:main-workflow-of-the-model-depricated}}\begin{enumerate}
\sphinxsetlistlabels{\arabic}{enumi}{enumii}{}{.}%
\item {} 
\sphinxAtStartPar
The research scope identification is done by either manually setting the plz code in GridGeneration.py main script or
an automatic search according to the administrative name of the district.

\item {} 
\sphinxAtStartPar
Run GridGeneration.py script, and the process will be proceeded to:
\begin{itemize}
\item {} 
\sphinxAtStartPar
extract correlated buildings, roads and transformers involved in the selected area

\item {} 
\sphinxAtStartPar
estimate the buildings’ peak load and remove too large consumers (connected directly to medium\sphinxhyphen{}voltage grid)

\item {} 
\sphinxAtStartPar
connect the buildings and transformers to the roads and analyse the network topology, remove isolated components

\item {} 
\sphinxAtStartPar
according to edge\sphinxhyphen{}distance matrix, assign transformers with corresponding neighbouring buildings, regarding cable
length limit and capacity limit

\item {} 
\sphinxAtStartPar
the remaining unsupplied buildings are subdivided into local distribution grids by hierarchical clustering, with
timely simultaneous peak load validation to determine proper cluster sizes

\item {} 
\sphinxAtStartPar
the optimal positions of manually grouped distribution grids are determined by a minimal power\sphinxhyphen{}distance algorithm,
aiming to minimize the network voltage band, energy losses on conductors and total cable length

\end{itemize}

\item {} 
\sphinxAtStartPar
At the end of GridGeneration.py process, the basic nodal elements of all the local distribution grids have been
determined. The installation of cables are determined in Cable\_installation.py with support of pandapower;

\item {} 
\sphinxAtStartPar
The timely power flow calculation shall afterward be conducted with random assignment of household load profiles, the
default validation period is one year from 00:00:00 01.01.2019 \sphinxhyphen{} 24:00:00 31.12.2019 with time step of 15 minutes,
where users can freely shorten or prolong the period or increase the time step by modifying defaulte parameters and
run power\_flow\_calculation.py

\item {} 
\sphinxAtStartPar
Some example vizualizations with an overview on total numbers and statistics of transformers, loads, cable length,
etc. on one side and a detialed spatial picture of chosen distribution grids on the other side are in development and
will be provided in a jupyter notebook in the examples directory.

\end{enumerate}


\subparagraph{Software preparation}
\label{\detokenize{grid_generation/main_workflow/main_workflow:software-preparation}}
\sphinxAtStartPar
The main script runs in Python, in addition you would need:
\begin{enumerate}
\sphinxsetlistlabels{\arabic}{enumi}{enumii}{}{.}%
\item {} 
\sphinxAtStartPar
PostgreSQL: default database;

\item {} 
\sphinxAtStartPar
PostGIS: extension for PostgreSQL, \sphinxurl{https://postgis.net/install/};

\item {} 
\sphinxAtStartPar
Pandapower: extension of Python where final grid result is displayed;

\end{enumerate}


\subparagraph{Output data}
\label{\detokenize{grid_generation/main_workflow/main_workflow:output-data}}\begin{enumerate}
\sphinxsetlistlabels{\arabic}{enumi}{enumii}{}{.}%
\item {} 
\sphinxAtStartPar
A folder of all local distribution grid results will be created as .csv files, named as ‘kcid\{a\}bcid\{b\}’ (a,b are
variables), such files can be easily read in python or input to pandapower;

\item {} 
\sphinxAtStartPar
A folder of all statistical analysis will be created, which the plotting process is based on;

\item {} 
\sphinxAtStartPar
Detailed building, road and transformer records will be saves in ‘\_result’ table in SQL;

\item {} 
\sphinxAtStartPar
All the graphics will also be saved as .png in a separate folder;

\end{enumerate}

\sphinxAtStartPar
{\color{red}\bfseries{}*}5. There are some auxiliary tables that result\_analysis.py would use but not correlated with final results, you can
either delete them and the process will generate them again, or keep them to save some computational effort for next
run.

\sphinxstepscope


\paragraph{Tool Usage}
\label{\detokenize{grid_generation/usage/usage:tool-usage}}\label{\detokenize{grid_generation/usage/usage::doc}}

\subparagraph{Database connection}
\label{\detokenize{grid_generation/usage/usage:database-connection}}
\sphinxAtStartPar
For connecting to the virtual machine,
see {[}Advanced Installation/Outside the VM{]}{[}outside\sphinxhyphen{}the\sphinxhyphen{}vm\sphinxhyphen{}from\sphinxhyphen{}ssh\sphinxhyphen{}client\sphinxhyphen{}aka\sphinxhyphen{}your\sphinxhyphen{}own\sphinxhyphen{}computer{]}

\sphinxAtStartPar
The database connections parameters can be changed in \_config\_data.py\_


\subparagraph{Configuration}
\label{\detokenize{grid_generation/usage/usage:configuration}}
\sphinxAtStartPar
If you want to manage the input data,
see {[}Advanced Installation/Load raw data to the database{]}{[}load\sphinxhyphen{}raw\sphinxhyphen{}data\sphinxhyphen{}to\sphinxhyphen{}the\sphinxhyphen{}database{]}

\sphinxAtStartPar
To change the \sphinxstylestrong{parameters} and the \sphinxstylestrong{version}, edit the file \_config\_version.py\_.


\subparagraph{Run}
\label{\detokenize{grid_generation/usage/usage:run}}
\sphinxAtStartPar
The script \_main\_grid\_generation.py\_ demonstrated how to run the grid generation tool.
If the grids for the given region and version are already generated, the code will terminate with an error statement.


\subparagraph{Result inspection}
\label{\detokenize{grid_generation/usage/usage:result-inspection}}
\sphinxAtStartPar
Download {[}QGIS{]}. Go to \sphinxtitleref{QGIS} directory in pylovo. Open QGIS file.
Database connection settings have to be set to the database that is used by pylovo.
Initial data (ways, buildings and transformers)
as well as the networks (transformers, cables, buildings) can be visualised.
Go to QGIS visualisation docu for more details


\subparagraph{Tutorials / Examples}
\label{\detokenize{grid_generation/usage/usage:tutorials-examples}}
\sphinxAtStartPar
Individual networks can also be visualised as explained in the \sphinxtitleref{examples} directory.
In the examples notebooks you will learn more about:
\begin{itemize}
\item {} 
\sphinxAtStartPar
the objects / elements the LV grids are made up of

\item {} 
\sphinxAtStartPar
the pandapower networks that are used to store the LV grids

\item {} 
\sphinxAtStartPar
graph representation of the networks

\item {} 
\sphinxAtStartPar
and parameter visualisation options

\end{itemize}

\sphinxAtStartPar
{[}QGIS{]}: \sphinxurl{https://www.qgis.org/de/site/forusers/download.html}
Refer to the Jupyter notebooks and other analysis tools?

\sphinxstepscope


\paragraph{Short Summary of the Grid Generation Steps}
\label{\detokenize{grid_generation/explanation/overview:short-summary-of-the-grid-generation-steps}}\label{\detokenize{grid_generation/explanation/overview::doc}}
\sphinxAtStartPar
These steps are performed to generate grids:

\noindent\sphinxincludegraphics[width=600\sphinxpxdimen]{{gridgen1to3}.png}

\noindent\sphinxincludegraphics[width=600\sphinxpxdimen]{{gridgen4to7}.png}

\sphinxstepscope


\paragraph{Building Data Import}
\label{\detokenize{grid_generation/explanation/building_data_import:building-data-import}}\label{\detokenize{grid_generation/explanation/building_data_import::doc}}
\sphinxAtStartPar
The building data is the basis for the grid generation as the building data contains geographical information as
well as the load that each consumer requires.

\sphinxAtStartPar
The building data is available at TUM as a dataset comprised of .shp files for residential \sphinxcode{\sphinxupquote{res}} and other
\sphinxcode{\sphinxupquote{oth}} buildings. The buildings are seperated by their Amtlicher Gemeindeschlüssel (AGS) a key for the
municipalities in Germany. The building dataset containing of files is to be unzipped and put into the directory
\sphinxcode{\sphinxupquote{raw\_data/buildings}}.

\sphinxAtStartPar
The files for Munich are thus named like this: \sphinxcode{\sphinxupquote{Oth\_9162000}} and \sphinxcode{\sphinxupquote{Res\_9162000}}.

\sphinxAtStartPar
The task of importing the building data before further Grid Generation steps is handled by
\index{import\_buildings\_for\_single\_plz() (in module raw\_data.import\_building\_data)@\spxentry{import\_buildings\_for\_single\_plz()}\spxextra{in module raw\_data.import\_building\_data}}

\begin{fulllineitems}
\phantomsection\label{\detokenize{grid_generation/explanation/building_data_import:raw_data.import_building_data.import_buildings_for_single_plz}}
\pysigstartsignatures
\pysiglinewithargsret{\sphinxcode{\sphinxupquote{raw\_data.import\_building\_data.}}\sphinxbfcode{\sphinxupquote{import\_buildings\_for\_single\_plz}}}{\emph{\DUrole{n}{plz}\DUrole{p}{:}\DUrole{w}{  }\DUrole{n}{str}}}{{ $\rightarrow$ None}}
\pysigstopsignatures
\sphinxAtStartPar
imports building data to db for plz:
\begin{itemize}
\item {} 
\sphinxAtStartPar
PLZ is matched with AGS

\item {} 
\sphinxAtStartPar
file name is generated

\item {} 
\sphinxAtStartPar
buildings files are imported to database with SyngridDatabaseConstructor

\item {} 
\sphinxAtStartPar
AGS is added to AGS as not to import same building data again

\end{itemize}
\begin{quote}\begin{description}
\sphinxlineitem{Parameters}
\sphinxAtStartPar
\sphinxstyleliteralstrong{\sphinxupquote{plz}} (\sphinxstyleliteralemphasis{\sphinxupquote{string}}) \textendash{} PLZ for building import

\end{description}\end{quote}

\end{fulllineitems}


\sphinxAtStartPar
Remark: The mapping of AGS with PLZ is not always unique. This might lead to unexpected building data import.

\sphinxAtStartPar
In this example, the PLZ for grid generation only overlapped with the AGS that buildings were imported in the most upper
part of the PLZ area.

\noindent\sphinxincludegraphics[width=500\sphinxpxdimen]{{ags_plz_mismatch}.png}

\sphinxAtStartPar
The imported buildings can be inspected using the QGIS visualisation {\hyperref[\detokenize{visualisation/qgis/qgis::doc}]{\sphinxcrossref{\DUrole{doc}{QGIS Visualisation}}}} in the
\sphinxcode{\sphinxupquote{raw\_data}} tab

\sphinxstepscope


\paragraph{Grid Generation Process}
\label{\detokenize{grid_generation/explanation/grid_generation_process:grid-generation-process}}\label{\detokenize{grid_generation/explanation/grid_generation_process::doc}}
\sphinxAtStartPar
The functionalities of the grid generation process are divided into three classes:
\index{SyngridDatabaseConstructor (class in syngrid.SyngridDatabaseConstructor)@\spxentry{SyngridDatabaseConstructor}\spxextra{class in syngrid.SyngridDatabaseConstructor}}

\begin{fulllineitems}
\phantomsection\label{\detokenize{grid_generation/explanation/grid_generation_process:syngrid.SyngridDatabaseConstructor.SyngridDatabaseConstructor}}
\pysigstartsignatures
\pysigline{\sphinxbfcode{\sphinxupquote{class\DUrole{w}{  }}}\sphinxcode{\sphinxupquote{syngrid.SyngridDatabaseConstructor.}}\sphinxbfcode{\sphinxupquote{SyngridDatabaseConstructor}}}
\pysigstopsignatures
\sphinxAtStartPar
Constructs a ready to use syngrid database. Be careful about overwriting the tables.

\end{fulllineitems}

\index{PgReaderWriter (class in syngrid.pgReaderWriter)@\spxentry{PgReaderWriter}\spxextra{class in syngrid.pgReaderWriter}}

\begin{fulllineitems}
\phantomsection\label{\detokenize{grid_generation/explanation/grid_generation_process:syngrid.pgReaderWriter.PgReaderWriter}}
\pysigstartsignatures
\pysiglinewithargsret{\sphinxbfcode{\sphinxupquote{class\DUrole{w}{  }}}\sphinxcode{\sphinxupquote{syngrid.pgReaderWriter.}}\sphinxbfcode{\sphinxupquote{PgReaderWriter}}}{\emph{\DUrole{o}{**}\DUrole{n}{kwargs}}}{}
\pysigstopsignatures
\sphinxAtStartPar
This class is the interface with the database. Functions communicating with the database
are listed under this class.

\end{fulllineitems}

\index{GridGenerator (class in syngrid.GridGenerator)@\spxentry{GridGenerator}\spxextra{class in syngrid.GridGenerator}}

\begin{fulllineitems}
\phantomsection\label{\detokenize{grid_generation/explanation/grid_generation_process:syngrid.GridGenerator.GridGenerator}}
\pysigstartsignatures
\pysiglinewithargsret{\sphinxbfcode{\sphinxupquote{class\DUrole{w}{  }}}\sphinxcode{\sphinxupquote{syngrid.GridGenerator.}}\sphinxbfcode{\sphinxupquote{GridGenerator}}}{\emph{\DUrole{n}{plz}}, \emph{\DUrole{o}{**}\DUrole{n}{kwargs}}}{}
\pysigstopsignatures
\sphinxAtStartPar
Generates the grid for the given plz area

\end{fulllineitems}


\sphinxAtStartPar
The higher level functions of the GridGenerator are explained in more detail since they contain the assumptions and logic
of grid generation. For a visual representation refer to {\hyperref[\detokenize{grid_generation/explanation/overview::doc}]{\sphinxcrossref{\DUrole{doc}{Short Summary of the Grid Generation Steps}}}}.


\subparagraph{Step 1}
\label{\detokenize{grid_generation/explanation/grid_generation_process:step-1}}\index{cache\_and\_preprocess\_static\_objects() (in module syngrid.GridGenerator.GridGenerator)@\spxentry{cache\_and\_preprocess\_static\_objects()}\spxextra{in module syngrid.GridGenerator.GridGenerator}}

\begin{fulllineitems}
\phantomsection\label{\detokenize{grid_generation/explanation/grid_generation_process:syngrid.GridGenerator.GridGenerator.cache_and_preprocess_static_objects}}
\pysigstartsignatures
\pysiglinewithargsret{\sphinxcode{\sphinxupquote{syngrid.GridGenerator.GridGenerator.}}\sphinxbfcode{\sphinxupquote{cache\_and\_preprocess\_static\_objects}}}{\emph{\DUrole{n}{self}}}{}
\pysigstopsignatures
\sphinxAtStartPar
Caches static objects (postcode, buildings, transformers) from raw data tables and
stores in temporary tables.
FROM: postcode, res, oth, transformers
INTO: postcode\_result, buildings\_tem
:return:

\end{fulllineitems}


\sphinxAtStartPar
The selected zip code (PLZ) is searched in the table
\sphinxcode{\sphinxupquote{postcode}} and stored in the \sphinxcode{\sphinxupquote{postcode\_result table}}. The zip code
defines the geographical area for which the network generation takes place.
The buildings, which are located in the area of the zip code, are selected from \sphinxcode{\sphinxupquote{res}} and \sphinxcode{\sphinxupquote{oth}}
and stored on \sphinxcode{\sphinxupquote{buildings\_tem}}. The tables ending with \sphinxcode{\sphinxupquote{tem}}
are tables that temporarily store the data for grid generation. For
the buildings of the zip code the house distance is calculated and the settlement type is
is derived from it. The settlement type decides which transformer types are installed in the zip code.
Each building is assigned a maximum load.
This depends on the building type. For residential buildings, the load is
scaled to households, for other buildings (commercial, public,
industrial buildings), the building floor area is taken into account in the calculation of the power.
of the power. Buildings without load or with load over 100kW
are not part of the low voltage network and are therefore deleted.
Finally, the transformers from transformers are also transferred to \sphinxcode{\sphinxupquote{buildings\_tem}}
are transferred.


\subparagraph{Step 2}
\label{\detokenize{grid_generation/explanation/grid_generation_process:step-2}}\index{preprocess\_ways() (in module syngrid.GridGenerator.GridGenerator)@\spxentry{preprocess\_ways()}\spxextra{in module syngrid.GridGenerator.GridGenerator}}

\begin{fulllineitems}
\phantomsection\label{\detokenize{grid_generation/explanation/grid_generation_process:syngrid.GridGenerator.GridGenerator.preprocess_ways}}
\pysigstartsignatures
\pysiglinewithargsret{\sphinxcode{\sphinxupquote{syngrid.GridGenerator.GridGenerator.}}\sphinxbfcode{\sphinxupquote{preprocess\_ways}}}{\emph{\DUrole{n}{self}}}{}
\pysigstopsignatures
\sphinxAtStartPar
Cache ways, create network, connect buildings to the ways network
FROM: ways, buildings\_tem
INTO: ways\_tem, buildings\_tem, ways\_tem\_vertices\_pgr, {\color{red}\bfseries{}ways\_tem\_}
:return:

\end{fulllineitems}


\sphinxAtStartPar
The ways from \sphinxcode{\sphinxupquote{ways}}, which are located in the zip code area, will be
stored on \sphinxcode{\sphinxupquote{ways\_tem}}. In way sections that overlap, connection nodes are created.
Then the buildings are connected to the ways.
For this purpose, a path section is created that leads perpendicularly
from the existing ways to the center of the building. Finally,
the buildings from \sphinxcode{\sphinxupquote{buildings\_tem}} are assigned a node from \sphinxcode{\sphinxupquote{ways\_tem}}.


\subparagraph{Step 3}
\label{\detokenize{grid_generation/explanation/grid_generation_process:step-3}}\index{apply\_kmeans\_clustering() (in module syngrid.GridGenerator.GridGenerator)@\spxentry{apply\_kmeans\_clustering()}\spxextra{in module syngrid.GridGenerator.GridGenerator}}

\begin{fulllineitems}
\phantomsection\label{\detokenize{grid_generation/explanation/grid_generation_process:syngrid.GridGenerator.GridGenerator.apply_kmeans_clustering}}
\pysigstartsignatures
\pysiglinewithargsret{\sphinxcode{\sphinxupquote{syngrid.GridGenerator.GridGenerator.}}\sphinxbfcode{\sphinxupquote{apply\_kmeans\_clustering}}}{\emph{\DUrole{n}{self}}}{}
\pysigstopsignatures
\sphinxAtStartPar
Reads ways and vertices from ways\_tem, applies k\sphinxhyphen{}means clustering
FROM: ways\_tem, buildings\_tem
INTO: ways\_tem, vertices\_pgr, buildings\_tem,
:return:

\end{fulllineitems}


\sphinxAtStartPar
Since the number of buildings in a postal code is too large for a coherent
network, the buildings are divided into subgroups.
The kmeans cluster algorithm divides the buildings into subgroups based on the geographic distance.
The number of kmeans clusters
for a postal code is usually single\sphinxhyphen{}digit. Each kmeans cluster is assigned an ID
(kcid, kmeans cluster ID).


\subparagraph{Step 4}
\label{\detokenize{grid_generation/explanation/grid_generation_process:step-4}}\index{position\_substations() (in module syngrid.GridGenerator.GridGenerator)@\spxentry{position\_substations()}\spxextra{in module syngrid.GridGenerator.GridGenerator}}

\begin{fulllineitems}
\phantomsection\label{\detokenize{grid_generation/explanation/grid_generation_process:syngrid.GridGenerator.GridGenerator.position_substations}}
\pysigstartsignatures
\pysiglinewithargsret{\sphinxcode{\sphinxupquote{syngrid.GridGenerator.GridGenerator.}}\sphinxbfcode{\sphinxupquote{position\_substations}}}{\emph{\DUrole{n}{self}}}{}
\pysigstopsignatures
\sphinxAtStartPar
Iterates over k\sphinxhyphen{}means clusters and building clusters inside and positions substations for each cluster.
Considers existing transformers.
FROM: buildings\_tem, building\_clusters
INTO: buildings\_tem, building\_clusters,
:return:

\end{fulllineitems}


\sphinxAtStartPar
For the positioning of the transformers, existing transformers from OSM are considered first.
Buildings at a certain distance from the transformer are connected to the transformer.
The shape of the supply area transforms
from a circle (allowed linear distance), to a polygon,
because the distance to the consumers is measured along the streets.


\subparagraph{Step 5}
\label{\detokenize{grid_generation/explanation/grid_generation_process:step-5}}
\sphinxAtStartPar
The kmeans clusters are further divided into so\sphinxhyphen{}called building clusters.
A buildings cluster becomes a network in which all buildings are connected to a low\sphinxhyphen{}voltage transformer.
The buildings are grouped
by means of an agglomerative hierarchical average linkage clustering.
The results of the hierarchical clustering can be displayed as a
Dendogram. The distance between two clusters is calculated as
average of all distances of buildings from cluster A to cluster B.
For this purpose, a distance matrix of the buildings is set up. After
each clustering step, there is a loop which verifies that the transformer power suffices to supply the consumers taking
into consideration
coincidence factor. The multilevel coincidence factor for each cable section is
evaluated by summing  the classified consumers (residential, public, commercial) of that cable section


\subparagraph{Step 6}
\label{\detokenize{grid_generation/explanation/grid_generation_process:step-6}}\index{install\_cables() (in module syngrid.GridGenerator.GridGenerator)@\spxentry{install\_cables()}\spxextra{in module syngrid.GridGenerator.GridGenerator}}

\begin{fulllineitems}
\phantomsection\label{\detokenize{grid_generation/explanation/grid_generation_process:syngrid.GridGenerator.GridGenerator.install_cables}}
\pysigstartsignatures
\pysiglinewithargsret{\sphinxcode{\sphinxupquote{syngrid.GridGenerator.GridGenerator.}}\sphinxbfcode{\sphinxupquote{install\_cables}}}{\emph{\DUrole{n}{self}}}{}
\pysigstopsignatures
\sphinxAtStartPar
the pandapower network for each cluster (kcid, bcid) is generated and filled with the corresponding
bus and line elements

\end{fulllineitems}


\sphinxAtStartPar
Branches are created to connect the consumers to the transformer, resulting in a radial network.
First, the consumers are connected to the road by cables.
A consumer is created as a \sphinxcode{\sphinxupquote{consumer node}} bus and every point
where several cables intersect as a \sphinxcode{\sphinxupquote{connection node bus}}. Next, the connections between the connection
nodes are drawn. Finally,
the connection nodes are connected to the transformers \sphinxcode{\sphinxupquote{LVbus}}.
The cables are run along the streets from \sphinxcode{\sphinxupquote{ways\_tem}}. When the branches are created, the
Minimal Spanning Tree Algorithm determines a configuration of the network,
whose edge lengths are as short as possible and thus inexpensive. From a
repertoire of cable types, a suitable cable is selected. The process of
cable installation is based on the realistic decision making of a technician
and avoids the use of additional costly network components.


\subparagraph{Step 7}
\label{\detokenize{grid_generation/explanation/grid_generation_process:step-7}}\index{saveInformationAndResetTables() (in module syngrid.pgReaderWriter.PgReaderWriter)@\spxentry{saveInformationAndResetTables()}\spxextra{in module syngrid.pgReaderWriter.PgReaderWriter}}

\begin{fulllineitems}
\phantomsection\label{\detokenize{grid_generation/explanation/grid_generation_process:syngrid.pgReaderWriter.PgReaderWriter.saveInformationAndResetTables}}
\pysigstartsignatures
\pysiglinewithargsret{\sphinxcode{\sphinxupquote{syngrid.pgReaderWriter.PgReaderWriter.}}\sphinxbfcode{\sphinxupquote{saveInformationAndResetTables}}}{\emph{\DUrole{n}{self}}, \emph{\DUrole{n}{plz}}}{}
\pysigstopsignatures
\end{fulllineitems}


\sphinxAtStartPar
The data from the tables with the extension
\sphinxcode{\sphinxupquote{tem}} are deleted and transferred to the result tables \sphinxcode{\sphinxupquote{bulidings\_result}},
\sphinxcode{\sphinxupquote{ways\_result}}.

\sphinxstepscope


\subsection{Classification of Pylovo Grids}
\label{\detokenize{classification/index:classification-of-pylovo-grids}}\label{\detokenize{classification/index::doc}}
\begin{DUlineblock}{0em}
\item[] Pylovo is a python tool for low\sphinxhyphen{}voltage distribution grid generation.
\item[] The package \sphinxstylestrong{classification} provides the utility of finding representative grids for LV\sphinxhyphen{}grids
\item[] of Germany or a federal state like Bayern.
\end{DUlineblock}


\subsubsection{Overview of the Classification}
\label{\detokenize{classification/index:overview-of-the-classification}}
\begin{DUlineblock}{0em}
\item[] An overview of the classification steps is provided here:
\end{DUlineblock}

\sphinxstepscope


\paragraph{Short Summary of the Classification Steps}
\label{\detokenize{classification/overview/overview:short-summary-of-the-classification-steps}}\label{\detokenize{classification/overview/overview::doc}}
\sphinxAtStartPar
The steps taken to classify / cluster grids is shown below:

\noindent\sphinxincludegraphics[width=400\sphinxpxdimen]{{classification_workflow_1}.png}

\noindent\sphinxincludegraphics[width=400\sphinxpxdimen]{{classification_workflow_2}.png}


\subsubsection{Run the Classification}
\label{\detokenize{classification/index:run-the-classification}}
\begin{DUlineblock}{0em}
\item[] To use this package pylovo needs to be fully set up. Please refer to {\hyperref[\detokenize{docs_pylovo/installation::doc}]{\sphinxcrossref{\DUrole{doc}{Installation}}}}.
\item[] To apply the classification visit this page:
\end{DUlineblock}

\sphinxstepscope


\paragraph{Usage}
\label{\detokenize{classification/usage/usage:usage}}\label{\detokenize{classification/usage/usage::doc}}

\subparagraph{Step 1: Configure the classification}
\label{\detokenize{classification/usage/usage:step-1-configure-the-classification}}
\sphinxAtStartPar
Open the file \sphinxcode{\sphinxupquote{config\_classification.py}}. Set up the classification by giving it a classification version,
a description and a region, e.g.:

\begin{sphinxVerbatim}[commandchars=\\\{\}]
\PYG{n}{CLASSIFICATION\PYGZus{}VERSION} \PYG{o}{=} \PYG{l+m+mi}{1}
\PYG{n}{VERSION\PYGZus{}COMMENT} \PYG{o}{=} \PYG{l+s+s2}{\PYGZdq{}}\PYG{l+s+s2}{set sample with inspected building quality}\PYG{l+s+s2}{\PYGZdq{}}
\PYG{n}{CLASSIFICATION\PYGZus{}REGION} \PYG{o}{=} \PYG{l+s+s1}{\PYGZsq{}}\PYG{l+s+s1}{Bayern}\PYG{l+s+s1}{\PYGZsq{}}
\end{sphinxVerbatim}

\sphinxAtStartPar
A classification version is a unique identifier for your classification and can only be used once. The version comment can be used
to describe the settings used in the classification version.
For the region either choose one of the federal states of Germany

\begin{sphinxVerbatim}[commandchars=\\\{\}]
\PYG{n}{REGION\PYGZus{}DICT} \PYG{o}{=} \PYG{p}{\PYGZob{}}\PYG{l+m+mi}{1}\PYG{p}{:} \PYG{l+s+s1}{\PYGZsq{}}\PYG{l+s+s1}{Schleswig\PYGZhy{}Hohlstein}\PYG{l+s+s1}{\PYGZsq{}}\PYG{p}{,}
           \PYG{l+m+mi}{2}\PYG{p}{:} \PYG{l+s+s1}{\PYGZsq{}}\PYG{l+s+s1}{Hamburg}\PYG{l+s+s1}{\PYGZsq{}}\PYG{p}{,}
           \PYG{l+m+mi}{3}\PYG{p}{:} \PYG{l+s+s1}{\PYGZsq{}}\PYG{l+s+s1}{Niedersachsen}\PYG{l+s+s1}{\PYGZsq{}}\PYG{p}{,}
           \PYG{l+m+mi}{4}\PYG{p}{:} \PYG{l+s+s1}{\PYGZsq{}}\PYG{l+s+s1}{Bremen}\PYG{l+s+s1}{\PYGZsq{}}\PYG{p}{,}
           \PYG{l+m+mi}{5}\PYG{p}{:} \PYG{l+s+s1}{\PYGZsq{}}\PYG{l+s+s1}{Nordrhein\PYGZhy{}Westfalen}\PYG{l+s+s1}{\PYGZsq{}}\PYG{p}{,}
           \PYG{l+m+mi}{6}\PYG{p}{:} \PYG{l+s+s1}{\PYGZsq{}}\PYG{l+s+s1}{Hessen}\PYG{l+s+s1}{\PYGZsq{}}\PYG{p}{,}
           \PYG{l+m+mi}{7}\PYG{p}{:} \PYG{l+s+s1}{\PYGZsq{}}\PYG{l+s+s1}{Rheinland\PYGZhy{}Pfalz}\PYG{l+s+s1}{\PYGZsq{}}\PYG{p}{,}
           \PYG{l+m+mi}{8}\PYG{p}{:} \PYG{l+s+s1}{\PYGZsq{}}\PYG{l+s+s1}{Baden\PYGZhy{}Württemberg}\PYG{l+s+s1}{\PYGZsq{}}\PYG{p}{,}
           \PYG{l+m+mi}{9}\PYG{p}{:} \PYG{l+s+s1}{\PYGZsq{}}\PYG{l+s+s1}{Bayern}\PYG{l+s+s1}{\PYGZsq{}}\PYG{p}{,}
           \PYG{l+m+mi}{10}\PYG{p}{:} \PYG{l+s+s1}{\PYGZsq{}}\PYG{l+s+s1}{Saarland}\PYG{l+s+s1}{\PYGZsq{}}\PYG{p}{,}
           \PYG{l+m+mi}{11}\PYG{p}{:} \PYG{l+s+s1}{\PYGZsq{}}\PYG{l+s+s1}{Berlin}\PYG{l+s+s1}{\PYGZsq{}}\PYG{p}{,}
           \PYG{l+m+mi}{12}\PYG{p}{:} \PYG{l+s+s1}{\PYGZsq{}}\PYG{l+s+s1}{Brandenburg}\PYG{l+s+s1}{\PYGZsq{}}\PYG{p}{,}
           \PYG{l+m+mi}{13}\PYG{p}{:} \PYG{l+s+s1}{\PYGZsq{}}\PYG{l+s+s1}{Mecklenburg\PYGZhy{}Vorpommern}\PYG{l+s+s1}{\PYGZsq{}}\PYG{p}{,}
           \PYG{l+m+mi}{14}\PYG{p}{:} \PYG{l+s+s1}{\PYGZsq{}}\PYG{l+s+s1}{Sachsen}\PYG{l+s+s1}{\PYGZsq{}}\PYG{p}{,}
           \PYG{l+m+mi}{15}\PYG{p}{:} \PYG{l+s+s1}{\PYGZsq{}}\PYG{l+s+s1}{Sachsen\PYGZhy{}Anhalt}\PYG{l+s+s1}{\PYGZsq{}}\PYG{p}{,}
           \PYG{l+m+mi}{16}\PYG{p}{:} \PYG{l+s+s1}{\PYGZsq{}}\PYG{l+s+s1}{Thüringen}\PYG{l+s+s1}{\PYGZsq{}}\PYG{p}{,}
           \PYG{p}{\PYGZcb{}}
\end{sphinxVerbatim}

\begin{DUlineblock}{0em}
\item[] that are listed as values in the dictionary or use \sphinxcode{\sphinxupquote{\textquotesingle{}Germany\textquotesingle{}}} for the entire country.
\item[] Make sure that you have the building data for the chosen region.
\end{DUlineblock}

\sphinxAtStartPar
Once you run the classification the constants that you have set will be saved
in the database table \sphinxcode{\sphinxupquote{classification\_version}}.


\subparagraph{Step 2: Calculate the Data for Clustering}
\label{\detokenize{classification/usage/usage:step-2-calculate-the-data-for-clustering}}
\begin{DUlineblock}{0em}
\item[] To generate the grids, calculate the grid parameters and filter erroneous grids run the file \sphinxcode{\sphinxupquote{prepare\_data\_for\_clustering.py}}.
\end{DUlineblock}

\sphinxAtStartPar
Threshold values for filtering can be set in \sphinxcode{\sphinxupquote{clustering.config\_clustering}}.

\sphinxAtStartPar
The processes run here are explained in {\hyperref[\detokenize{classification/classification_steps/sampling::doc}]{\sphinxcrossref{\DUrole{doc}{Create a Sample Set of PLZ for Clustering}}}},
{\hyperref[\detokenize{classification/classification_steps/grid_generation_for_classification::doc}]{\sphinxcrossref{\DUrole{doc}{Grid Generation for Classification}}}},
{\hyperref[\detokenize{classification/classification_steps/parameters::doc}]{\sphinxcrossref{\DUrole{doc}{Calculate Parameters for Grids}}}} and
{\hyperref[\detokenize{classification/classification_steps/filter_grids::doc}]{\sphinxcrossref{\DUrole{doc}{Filter Grids}}}}.


\subparagraph{Step 3: Inspect sampling results (optional)}
\label{\detokenize{classification/usage/usage:step-3-inspect-sampling-results-optional}}
\sphinxAtStartPar
In the notebook \sphinxcode{\sphinxupquote{examples\_sampling.analyse\_sampling\_results}} the samples drawn in the region of classification
are visualised on a map.

\noindent\sphinxincludegraphics[width=400\sphinxpxdimen]{{karte_samples2}.png}

\sphinxAtStartPar
The distribution of samples within Regiostar 7 classes is shown.

\noindent\sphinxincludegraphics[width=400\sphinxpxdimen]{{anz_samples}.png}


\subparagraph{Step 4: Inspect grid generation results (optional)}
\label{\detokenize{classification/usage/usage:step-4-inspect-grid-generation-results-optional}}
\sphinxAtStartPar
The generated grids can be visualised in QGIS (all grids)

\noindent\sphinxincludegraphics[width=400\sphinxpxdimen]{{qgis6}.png}

\sphinxAtStartPar
or using plotting functions (individual grids). For more details see {\hyperref[\detokenize{visualisation/index::doc}]{\sphinxcrossref{\DUrole{doc}{Visualisation}}}}


\subparagraph{Step 5: Inspect grid parameters (optional)}
\label{\detokenize{classification/usage/usage:step-5-inspect-grid-parameters-optional}}
\sphinxAtStartPar
In the package \sphinxcode{\sphinxupquote{examples\_grid\_parameters}} the parameters of the grids can be analysed. In the notebook
\sphinxcode{\sphinxupquote{analyse\_clustering\_parameters}} a matrix of scatter plots called pairplot is shown to gain an overview of the data.

\noindent\sphinxincludegraphics[width=400\sphinxpxdimen]{{pairplot}.png}

\sphinxAtStartPar
The grids can be sorted by a parameter to show grids with specific characteristics.

\sphinxAtStartPar
The notebook \sphinxcode{\sphinxupquote{vsw\_analysis}} focuses on the ‘Verbrauchersummenwiderstand’ (resistance in the network) that can be
an indicator for voltage drop of branches.


\subparagraph{Step 6: Choose parameters for clustering}
\label{\detokenize{classification/usage/usage:step-6-choose-parameters-for-clustering}}
\sphinxAtStartPar
The package \sphinxcode{\sphinxupquote{examples\_correlation\_and\_factor\_analysis}} has the tools to choose the parameters for clustering.
In this work, it is proposed to choose the clustering parameters according to the factor analysis.

\sphinxAtStartPar
The notebook \sphinxcode{\sphinxupquote{1\_0\_factor\_analysis}} guides you through the process of finding the number of and the parameters
that are mathematically optimal for clustering. The resulting parameters are listed at the end of the document.

\noindent\sphinxincludegraphics[width=400\sphinxpxdimen]{{factor_analysis}.png}

\sphinxAtStartPar
The proposed parameters do not need to be taken for clustering. Other preferences and considerations can be taken into account.
Additional information like the explained variance of the factors or components can be found in the notebook
\sphinxcode{\sphinxupquote{1\_1\_explained variance\_eigen\_decomposition}}.

\noindent\sphinxincludegraphics[width=400\sphinxpxdimen]{{eigendecomposition}.png}

\sphinxAtStartPar
The correlation matrix and clustermap are plotted in the notebook \sphinxcode{\sphinxupquote{1\_2\_correlation\_matrix}}

\sphinxAtStartPar
After you have choosen the parameters set them in \sphinxcode{\sphinxupquote{clustering.config\_clustering}}, like:

\begin{sphinxVerbatim}[commandchars=\\\{\}]
\PYG{c+c1}{\PYGZsh{} set clustering parameters}
\PYG{n}{param1} \PYG{o}{=} \PYG{l+s+s1}{\PYGZsq{}}\PYG{l+s+s1}{no\PYGZus{}branches}\PYG{l+s+s1}{\PYGZsq{}}
\PYG{n}{param2} \PYG{o}{=} \PYG{l+s+s1}{\PYGZsq{}}\PYG{l+s+s1}{avg\PYGZus{}trafo\PYGZus{}dis}\PYG{l+s+s1}{\PYGZsq{}}
\PYG{n}{param3} \PYG{o}{=} \PYG{l+s+s1}{\PYGZsq{}}\PYG{l+s+s1}{max\PYGZus{}no\PYGZus{}of\PYGZus{}households\PYGZus{}of\PYGZus{}a\PYGZus{}branch}\PYG{l+s+s1}{\PYGZsq{}}
\PYG{n}{param4} \PYG{o}{=} \PYG{l+s+s1}{\PYGZsq{}}\PYG{l+s+s1}{no\PYGZus{}house\PYGZus{}connections\PYGZus{}per\PYGZus{}branch}\PYG{l+s+s1}{\PYGZsq{}}
\PYG{n}{LIST\PYGZus{}OF\PYGZus{}CLUSTERING\PYGZus{}PARAMETERS} \PYG{o}{=} \PYG{p}{[}\PYG{n}{param1}\PYG{p}{,} \PYG{n}{param2}\PYG{p}{,} \PYG{n}{param3}\PYG{p}{,} \PYG{n}{param4}\PYG{p}{]}
\end{sphinxVerbatim}


\subparagraph{Step 7: Choose number of clusters}
\label{\detokenize{classification/usage/usage:step-7-choose-number-of-clusters}}
\sphinxAtStartPar
In the package \sphinxcode{\sphinxupquote{examples\_indices}} you will find two indices for finding the optimal number of clusters:
\begin{itemize}
\item {} 
\sphinxAtStartPar
Calinski Harabasz Index or CH Index and

\item {} 
\sphinxAtStartPar
Davies Bouldin Index or DB Index

\end{itemize}

\sphinxAtStartPar
It is recommended to choose the number of clusters with the CH index from the notebook
\sphinxcode{\sphinxupquote{1\_CH\_index}}. The DB Index can be used for reference.

\noindent\sphinxincludegraphics[width=400\sphinxpxdimen]{{ch_index}.png}

\sphinxAtStartPar
Again according to the goals of clustering with orientation of the index results set the numbers of clusters
for the cluster algorithms in \sphinxcode{\sphinxupquote{clustering.config\_clustering}}:

\begin{sphinxVerbatim}[commandchars=\\\{\}]
\PYG{c+c1}{\PYGZsh{} set number of clusters}
\PYG{n}{N\PYGZus{}CLUSTERS\PYGZus{}KMEDOID} \PYG{o}{=} \PYG{l+m+mi}{5}
\PYG{n}{N\PYGZus{}CLUSTERS\PYGZus{}KMEANS} \PYG{o}{=} \PYG{l+m+mi}{5}
\PYG{n}{N\PYGZus{}CLUSTERS\PYGZus{}GMM} \PYG{o}{=} \PYG{l+m+mi}{4}  \PYG{c+c1}{\PYGZsh{} refers to gmm tied}
\end{sphinxVerbatim}


\subparagraph{Step 8: Clustering results}
\label{\detokenize{classification/usage/usage:step-8-clustering-results}}
\sphinxAtStartPar
You now have the option to investigate the results in \sphinxcode{\sphinxupquote{examples\_clustering}}. For each of the clusterin algorithms
\begin{itemize}
\item {} 
\sphinxAtStartPar
kmeans,

\item {} 
\sphinxAtStartPar
kmedoids and

\item {} 
\sphinxAtStartPar
gmm tied

\end{itemize}

\sphinxAtStartPar
there are two notebooks. In the first one, the representative grids  are presented. Their clustering parameters can be
compared with the radar plot. The representative grids are plotted individually.

\noindent\sphinxincludegraphics[width=400\sphinxpxdimen]{{radar_plot_kmeans}.png}

\sphinxAtStartPar
In the second notebook is more concerned with the overall clusters and the distribution of clusters over the regiostar
classes are plotted.

\noindent\sphinxincludegraphics[width=400\sphinxpxdimen]{{distribution_regio5_clusters}.png}

\sphinxAtStartPar
To view the clustering results in QGIS run \sphinxcode{\sphinxupquote{apply\_clustering\_for\_QGIS\_visualisation}} and open QGIS.
There you have the option to identify the clusters of the grids by color

\noindent\sphinxincludegraphics[width=400\sphinxpxdimen]{{clusters_qgis}.png}

\sphinxAtStartPar
More details about the clustering functions can be found in {\hyperref[\detokenize{classification/classification_steps/clustering::doc}]{\sphinxcrossref{\DUrole{doc}{Cluster the Grids}}}}


\subsubsection{How the Classification Works}
\label{\detokenize{classification/index:how-the-classification-works}}
\begin{DUlineblock}{0em}
\item[] Here are some more details about which functions are called for the classification steps.
\end{DUlineblock}

\sphinxstepscope


\paragraph{Classification Steps}
\label{\detokenize{classification/classification_steps/index:classification-steps}}\label{\detokenize{classification/classification_steps/index::doc}}
\sphinxstepscope


\subparagraph{Create a Sample Set of PLZ for Clustering}
\label{\detokenize{classification/classification_steps/sampling:create-a-sample-set-of-plz-for-clustering}}\label{\detokenize{classification/classification_steps/sampling::doc}}

\subparagraph{Motivation}
\label{\detokenize{classification/classification_steps/sampling:motivation}}
\sphinxAtStartPar
To capture the diversity of different types of settlements and represent them for the chosen region adequately a
sampling algorithm is developed. For the area of interest that could be a Bundesland or Germany generating grids for
the whole region takes to long. The number of PLZ is reduced from 8181 (Germany) or e.g. 2065 (Bayern) to 100 PLZ in
the representative sample set.


\subparagraph{Usage}
\label{\detokenize{classification/classification_steps/sampling:usage}}
\sphinxAtStartPar
The sample set is created with the function:
\index{create\_sample\_set() (in module classification.sampling.sample)@\spxentry{create\_sample\_set()}\spxextra{in module classification.sampling.sample}}

\begin{fulllineitems}
\phantomsection\label{\detokenize{classification/classification_steps/sampling:classification.sampling.sample.create_sample_set}}
\pysigstartsignatures
\pysiglinewithargsret{\sphinxcode{\sphinxupquote{classification.sampling.sample.}}\sphinxbfcode{\sphinxupquote{create\_sample\_set}}}{}{}
\pysigstopsignatures
\sphinxAtStartPar
complete process of creating a sample set of representative PLZ for a Region
that is either Germany or a federal state
All subprocesses of sampling the PLZ are executed in this function.
The result is written to database table ‘sample set’ with the classification version set in
config\_classification

\end{fulllineitems}


\sphinxAtStartPar
The result can be retrieved from the database as a DataFrame with
\index{get\_sample\_set() (in module classification.sampling.sample)@\spxentry{get\_sample\_set()}\spxextra{in module classification.sampling.sample}}

\begin{fulllineitems}
\phantomsection\label{\detokenize{classification/classification_steps/sampling:classification.sampling.sample.get_sample_set}}
\pysigstartsignatures
\pysiglinewithargsret{\sphinxcode{\sphinxupquote{classification.sampling.sample.}}\sphinxbfcode{\sphinxupquote{get\_sample\_set}}}{}{{ $\rightarrow$ DataFrame}}
\pysigstopsignatures
\sphinxAtStartPar
get a sample set from the database that has already been created
\begin{quote}\begin{description}
\sphinxlineitem{Returns}
\sphinxAtStartPar
table of a complete sample set

\sphinxlineitem{Return type}
\sphinxAtStartPar
pd.DataFrame

\end{description}\end{quote}

\end{fulllineitems}



\subparagraph{Joining Regiostar with the Municipal Register}
\label{\detokenize{classification/classification_steps/sampling:joining-regiostar-with-the-municipal-register}}
\sphinxAtStartPar
In order to sample areas from the Regiostar Classes they need to be mapped with the Municipal Register
(Gemeindeverzeichnis) from Germany. You can refer to {\hyperref[\detokenize{docs_pylovo/municipal_register/municipal_register::doc}]{\sphinxcrossref{\DUrole{doc}{Municipal Register}}}}
for more details.


\subparagraph{Sampling Algorithm}
\label{\detokenize{classification/classification_steps/sampling:sampling-algorithm}}
\sphinxAtStartPar
To make sure all settlement types are represented in the sample set the Regio 7 Dataset with seven different types of
municipalities is chosen.

\noindent\sphinxincludegraphics[width=600\sphinxpxdimen]{{regiostar}.png}

\sphinxAtStartPar
The share of PLZ samples from each RegioStar 7 class corresponds to the  population share of the class of the total
population. (An analysis by the Chair of Renewable and
Sustainable Energy Systems at TUM have shown that the percentage of energy usage of a Regio7 class corresponds to its
population share.)

\sphinxAtStartPar
For Bayern the distribution of samples are:

\noindent\sphinxincludegraphics[width=400\sphinxpxdimen]{{anz_samples}.png}

\sphinxAtStartPar
Within the classes the population density distribution of the class is reproduced.

\noindent\sphinxincludegraphics[width=600\sphinxpxdimen]{{pop_den}.png}

\sphinxstepscope


\subparagraph{Grid Generation for Classification}
\label{\detokenize{classification/classification_steps/grid_generation_for_classification:grid-generation-for-classification}}\label{\detokenize{classification/classification_steps/grid_generation_for_classification::doc}}
\sphinxAtStartPar
For the grid generation two steps are performed:
\index{import\_buildings\_for\_multiple\_plz() (in module raw\_data.import\_building\_data)@\spxentry{import\_buildings\_for\_multiple\_plz()}\spxextra{in module raw\_data.import\_building\_data}}

\begin{fulllineitems}
\phantomsection\label{\detokenize{classification/classification_steps/grid_generation_for_classification:raw_data.import_building_data.import_buildings_for_multiple_plz}}
\pysigstartsignatures
\pysiglinewithargsret{\sphinxcode{\sphinxupquote{raw\_data.import\_building\_data.}}\sphinxbfcode{\sphinxupquote{import\_buildings\_for\_multiple\_plz}}}{\emph{\DUrole{n}{sample\_plz}}}{}
\pysigstopsignatures
\sphinxAtStartPar
imports building data to db for multiple plz

\end{fulllineitems}


\sphinxAtStartPar
and
\index{generate\_grid\_for\_multiple\_plz() (in module classification.database\_communication.perform\_classification\_tasks\_for\_multiple\_plz)@\spxentry{generate\_grid\_for\_multiple\_plz()}\spxextra{in module classification.database\_communication.perform\_classification\_tasks\_for\_multiple\_plz}}

\begin{fulllineitems}
\phantomsection\label{\detokenize{classification/classification_steps/grid_generation_for_classification:classification.database_communication.perform_classification_tasks_for_multiple_plz.generate_grid_for_multiple_plz}}
\pysigstartsignatures
\pysiglinewithargsret{\sphinxcode{\sphinxupquote{classification.database\_communication.perform\_classification\_tasks\_for\_multiple\_plz.}}\sphinxbfcode{\sphinxupquote{generate\_grid\_for\_multiple\_plz}}}{\emph{\DUrole{n}{df\_samples}\DUrole{p}{:}\DUrole{w}{  }\DUrole{n}{DataFrame}}}{{ $\rightarrow$ None}}
\pysigstopsignatures
\sphinxAtStartPar
generates grid for all plz contained in the column ‘plz’ of df\_samples
\begin{quote}\begin{description}
\sphinxlineitem{Parameters}
\sphinxAtStartPar
\sphinxstyleliteralstrong{\sphinxupquote{df\_samples}} (\sphinxstyleliteralemphasis{\sphinxupquote{pd.DataFrame}}) \textendash{} table that contains PLZ for grid generation

\end{description}\end{quote}

\end{fulllineitems}


\sphinxAtStartPar
These functions perform the same tasks described in {\hyperref[\detokenize{grid_generation/index::doc}]{\sphinxcrossref{\DUrole{doc}{Grid Generation}}}} for multiple PLZ.
Check these out to find out more about the processes of building import and grid generation.
Check out the {\hyperref[\detokenize{visualisation/qgis/qgis::doc}]{\sphinxcrossref{\DUrole{doc}{QGIS Visualisation}}}} to inspect the resulting grids on a map.

\sphinxstepscope


\subparagraph{Calculate Parameters for Grids}
\label{\detokenize{classification/classification_steps/parameters:calculate-parameters-for-grids}}\label{\detokenize{classification/classification_steps/parameters::doc}}\index{GridParameters (class in classification.parameter\_calculation.GridParameters)@\spxentry{GridParameters}\spxextra{class in classification.parameter\_calculation.GridParameters}}

\begin{fulllineitems}
\phantomsection\label{\detokenize{classification/classification_steps/parameters:classification.parameter_calculation.GridParameters.GridParameters}}
\pysigstartsignatures
\pysiglinewithargsret{\sphinxbfcode{\sphinxupquote{class\DUrole{w}{  }}}\sphinxcode{\sphinxupquote{classification.parameter\_calculation.GridParameters.}}\sphinxbfcode{\sphinxupquote{GridParameters}}}{\emph{\DUrole{n}{plz}}, \emph{\DUrole{n}{bcid}}, \emph{\DUrole{n}{kcid}}, \emph{\DUrole{n}{pgReaderWriter}}}{}
\pysigstopsignatures
\sphinxAtStartPar
A set of parameters and functions to calculate and save grid parameters
for a plz that are generated with a certain VERSION\_ID

\end{fulllineitems}


\sphinxAtStartPar
has the attributes

\begin{sphinxVerbatim}[commandchars=\\\{\}]
\PYG{n+nb+bp}{self}\PYG{o}{.}\PYG{n}{version\PYGZus{}id} \PYG{o}{=} \PYG{n}{VERSION\PYGZus{}ID}
\PYG{n+nb+bp}{self}\PYG{o}{.}\PYG{n}{plz} \PYG{o}{=} \PYG{n}{plz}
\PYG{n+nb+bp}{self}\PYG{o}{.}\PYG{n}{bcid} \PYG{o}{=} \PYG{n}{bcid}
\PYG{n+nb+bp}{self}\PYG{o}{.}\PYG{n}{kcid} \PYG{o}{=} \PYG{n}{kcid}
\PYG{n+nb+bp}{self}\PYG{o}{.}\PYG{n}{no\PYGZus{}connection\PYGZus{}buses} \PYG{o}{=} \PYG{k+kc}{None}
\PYG{n+nb+bp}{self}\PYG{o}{.}\PYG{n}{no\PYGZus{}branches} \PYG{o}{=} \PYG{k+kc}{None}
\PYG{n+nb+bp}{self}\PYG{o}{.}\PYG{n}{no\PYGZus{}house\PYGZus{}connections} \PYG{o}{=} \PYG{k+kc}{None}
\PYG{n+nb+bp}{self}\PYG{o}{.}\PYG{n}{no\PYGZus{}house\PYGZus{}connections\PYGZus{}per\PYGZus{}branch} \PYG{o}{=} \PYG{k+kc}{None}
\PYG{n+nb+bp}{self}\PYG{o}{.}\PYG{n}{no\PYGZus{}households} \PYG{o}{=} \PYG{k+kc}{None}
\PYG{n+nb+bp}{self}\PYG{o}{.}\PYG{n}{no\PYGZus{}household\PYGZus{}equ} \PYG{o}{=} \PYG{k+kc}{None}
\PYG{n+nb+bp}{self}\PYG{o}{.}\PYG{n}{no\PYGZus{}households\PYGZus{}per\PYGZus{}branch} \PYG{o}{=} \PYG{k+kc}{None}
\PYG{n+nb+bp}{self}\PYG{o}{.}\PYG{n}{max\PYGZus{}no\PYGZus{}of\PYGZus{}households\PYGZus{}of\PYGZus{}a\PYGZus{}branch} \PYG{o}{=} \PYG{k+kc}{None}
\PYG{n+nb+bp}{self}\PYG{o}{.}\PYG{n}{house\PYGZus{}distance\PYGZus{}km} \PYG{o}{=} \PYG{k+kc}{None}
\PYG{n+nb+bp}{self}\PYG{o}{.}\PYG{n}{transformer\PYGZus{}mva} \PYG{o}{=} \PYG{k+kc}{None}
\PYG{n+nb+bp}{self}\PYG{o}{.}\PYG{n}{osm\PYGZus{}trafo} \PYG{o}{=} \PYG{k+kc}{None}
\PYG{n+nb+bp}{self}\PYG{o}{.}\PYG{n}{max\PYGZus{}trafo\PYGZus{}dis} \PYG{o}{=} \PYG{k+kc}{None}
\PYG{n+nb+bp}{self}\PYG{o}{.}\PYG{n}{avg\PYGZus{}trafo\PYGZus{}dis} \PYG{o}{=} \PYG{k+kc}{None}
\PYG{n+nb+bp}{self}\PYG{o}{.}\PYG{n}{cable\PYGZus{}length\PYGZus{}km} \PYG{o}{=} \PYG{k+kc}{None}
\PYG{n+nb+bp}{self}\PYG{o}{.}\PYG{n}{cable\PYGZus{}len\PYGZus{}per\PYGZus{}house} \PYG{o}{=} \PYG{k+kc}{None}
\PYG{n+nb+bp}{self}\PYG{o}{.}\PYG{n}{max\PYGZus{}power\PYGZus{}mw} \PYG{o}{=} \PYG{k+kc}{None}
\PYG{n+nb+bp}{self}\PYG{o}{.}\PYG{n}{simultaneous\PYGZus{}peak\PYGZus{}load\PYGZus{}mw} \PYG{o}{=} \PYG{k+kc}{None}
\PYG{n+nb+bp}{self}\PYG{o}{.}\PYG{n}{resistance} \PYG{o}{=} \PYG{k+kc}{None}
\PYG{n+nb+bp}{self}\PYG{o}{.}\PYG{n}{reactance} \PYG{o}{=} \PYG{k+kc}{None}
\PYG{n+nb+bp}{self}\PYG{o}{.}\PYG{n}{ratio} \PYG{o}{=} \PYG{k+kc}{None}
\PYG{n+nb+bp}{self}\PYG{o}{.}\PYG{n}{vsw\PYGZus{}per\PYGZus{}branch} \PYG{o}{=} \PYG{k+kc}{None}
\PYG{n+nb+bp}{self}\PYG{o}{.}\PYG{n}{max\PYGZus{}vsw\PYGZus{}of\PYGZus{}a\PYGZus{}branch} \PYG{o}{=} \PYG{k+kc}{None}
\end{sphinxVerbatim}

\sphinxAtStartPar
that are calculated in the function:
\index{calc\_grid\_parameters() (in module classification.parameter\_calculation.GridParameters.GridParameters)@\spxentry{calc\_grid\_parameters()}\spxextra{in module classification.parameter\_calculation.GridParameters.GridParameters}}

\begin{fulllineitems}
\phantomsection\label{\detokenize{classification/classification_steps/parameters:classification.parameter_calculation.GridParameters.GridParameters.calc_grid_parameters}}
\pysigstartsignatures
\pysiglinewithargsret{\sphinxcode{\sphinxupquote{classification.parameter\_calculation.GridParameters.GridParameters.}}\sphinxbfcode{\sphinxupquote{calc\_grid\_parameters}}}{\emph{\DUrole{n}{self}}}{{ $\rightarrow$ None}}
\pysigstopsignatures
\sphinxAtStartPar
calculate parameters of each grid
save results to table ‘clustering\_parameters’ on database

\end{fulllineitems}


\sphinxAtStartPar
To calculate the parameters for all grids in the classification version use:
\index{calculate\_parameters\_for\_mulitple\_plz() (in module classification.database\_communication.perform\_classification\_tasks\_for\_multiple\_plz)@\spxentry{calculate\_parameters\_for\_mulitple\_plz()}\spxextra{in module classification.database\_communication.perform\_classification\_tasks\_for\_multiple\_plz}}

\begin{fulllineitems}
\phantomsection\label{\detokenize{classification/classification_steps/parameters:classification.database_communication.perform_classification_tasks_for_multiple_plz.calculate_parameters_for_mulitple_plz}}
\pysigstartsignatures
\pysiglinewithargsret{\sphinxcode{\sphinxupquote{classification.database\_communication.perform\_classification\_tasks\_for\_multiple\_plz.}}\sphinxbfcode{\sphinxupquote{calculate\_parameters\_for\_mulitple\_plz}}}{\emph{\DUrole{n}{df\_samples}\DUrole{p}{:}\DUrole{w}{  }\DUrole{n}{DataFrame}}}{{ $\rightarrow$ None}}
\pysigstopsignatures
\sphinxAtStartPar
calculates grid parameters for all plz contained in the column ‘plz’ of df\_samples
\begin{quote}\begin{description}
\sphinxlineitem{Parameters}
\sphinxAtStartPar
\sphinxstyleliteralstrong{\sphinxupquote{df\_samples}} (\sphinxstyleliteralemphasis{\sphinxupquote{pd.DataFrame}}) \textendash{} table that contains PLZ for parameter calculation

\end{description}\end{quote}

\end{fulllineitems}


\sphinxstepscope


\subparagraph{Filter Grids}
\label{\detokenize{classification/classification_steps/filter_grids:filter-grids}}\label{\detokenize{classification/classification_steps/filter_grids::doc}}
\sphinxAtStartPar
Threshold values for grid parameters:
filter grids that are unrealistic and should thus not be considered for clustering.

\sphinxAtStartPar
This function checks which grids exceed the threshold values:

\sphinxAtStartPar
When the grids’ parameters are retrieved via this function, the grids exceeding the thresholds are not returned:

\sphinxstepscope


\subparagraph{Cluster the Grids}
\label{\detokenize{classification/classification_steps/clustering:cluster-the-grids}}\label{\detokenize{classification/classification_steps/clustering::doc}}
\sphinxAtStartPar
With the clustering configuration set in \sphinxcode{\sphinxupquote{config\_clustering}} the clustering is performed in the three functions:
\index{kmedoids\_clustering() (in module classification.clustering.clustering\_algorithms)@\spxentry{kmedoids\_clustering()}\spxextra{in module classification.clustering.clustering\_algorithms}}

\begin{fulllineitems}
\phantomsection\label{\detokenize{classification/classification_steps/clustering:classification.clustering.clustering_algorithms.kmedoids_clustering}}
\pysigstartsignatures
\pysiglinewithargsret{\sphinxcode{\sphinxupquote{classification.clustering.clustering\_algorithms.}}\sphinxbfcode{\sphinxupquote{kmedoids\_clustering}}}{\emph{\DUrole{n}{df\_parameters\_of\_grids: \textasciitilde{}pandas.core.frame.DataFrame}}, \emph{\DUrole{n}{list\_of\_clustering\_parameters: list}}, \emph{\DUrole{n}{n\_clusters: int) \sphinxhyphen{}\textgreater{} (\textless{}class \textquotesingle{}pandas.core.frame.DataFrame\textquotesingle{}\textgreater{}}}, \emph{\DUrole{n}{\textless{}class \textquotesingle{}pandas.core.frame.DataFrame\textquotesingle{}\textgreater{}}}}{}
\pysigstopsignatures
\sphinxAtStartPar
Clustering the grids with kmedoids algorithm


\subparagraph{Parameters}
\label{\detokenize{classification/classification_steps/clustering:parameters}}\begin{description}
\sphinxlineitem{df\_parameters\_of\_grids}{[}DataFrame{]}
\sphinxAtStartPar
Grids with parameters to be clustered

\sphinxlineitem{list\_of\_clustering\_parameters}{[}list of strings{]}
\sphinxAtStartPar
Parameters used for clustering.

\sphinxlineitem{n\_clusters: int}
\sphinxAtStartPar
Number of clusters.

\end{description}


\subparagraph{Returns}
\label{\detokenize{classification/classification_steps/clustering:returns}}\begin{description}
\sphinxlineitem{Dataframe:}
\sphinxAtStartPar
Grids that are attributed to a cluster.

\sphinxlineitem{Dataframe:}
\sphinxAtStartPar
Grids that are medoids (cluster centers).

\end{description}

\end{fulllineitems}

\index{gmm\_tied\_clustering() (in module classification.clustering.clustering\_algorithms)@\spxentry{gmm\_tied\_clustering()}\spxextra{in module classification.clustering.clustering\_algorithms}}

\begin{fulllineitems}
\phantomsection\label{\detokenize{classification/classification_steps/clustering:classification.clustering.clustering_algorithms.gmm_tied_clustering}}
\pysigstartsignatures
\pysiglinewithargsret{\sphinxcode{\sphinxupquote{classification.clustering.clustering\_algorithms.}}\sphinxbfcode{\sphinxupquote{gmm\_tied\_clustering}}}{\emph{\DUrole{n}{df\_parameters\_of\_grids: \textasciitilde{}pandas.core.frame.DataFrame}}, \emph{\DUrole{n}{list\_of\_clustering\_parameters: list}}, \emph{\DUrole{n}{n\_clusters: int) \sphinxhyphen{}\textgreater{} (\textless{}class \textquotesingle{}pandas.core.frame.DataFrame\textquotesingle{}\textgreater{}}}, \emph{\DUrole{n}{\textless{}class \textquotesingle{}pandas.core.frame.DataFrame\textquotesingle{}\textgreater{}}}}{}
\pysigstopsignatures
\sphinxAtStartPar
Clustering the grids with gmm tied algorithm


\subparagraph{Parameters}
\label{\detokenize{classification/classification_steps/clustering:id1}}\begin{description}
\sphinxlineitem{df\_parameters\_of\_grids}{[}DataFrame{]}
\sphinxAtStartPar
Grids with parameters to be clustered

\sphinxlineitem{list\_of\_clustering\_parameters}{[}list of strings{]}
\sphinxAtStartPar
Parameters used for clustering.

\sphinxlineitem{n\_clusters: int}
\sphinxAtStartPar
Number of clusters.

\end{description}


\subparagraph{Returns}
\label{\detokenize{classification/classification_steps/clustering:id2}}\begin{description}
\sphinxlineitem{Dataframe:}
\sphinxAtStartPar
Grids that are attributed to a cluster.

\sphinxlineitem{Dataframe:}
\sphinxAtStartPar
Grids that are medoids (cluster centers).

\end{description}

\end{fulllineitems}

\index{kmeans\_clustering() (in module classification.clustering.clustering\_algorithms)@\spxentry{kmeans\_clustering()}\spxextra{in module classification.clustering.clustering\_algorithms}}

\begin{fulllineitems}
\phantomsection\label{\detokenize{classification/classification_steps/clustering:classification.clustering.clustering_algorithms.kmeans_clustering}}
\pysigstartsignatures
\pysiglinewithargsret{\sphinxcode{\sphinxupquote{classification.clustering.clustering\_algorithms.}}\sphinxbfcode{\sphinxupquote{kmeans\_clustering}}}{\emph{\DUrole{n}{df\_parameters\_of\_grids: \textasciitilde{}pandas.core.frame.DataFrame}}, \emph{\DUrole{n}{list\_of\_clustering\_parameters: list}}, \emph{\DUrole{n}{n\_clusters: int) \sphinxhyphen{}\textgreater{} (\textless{}class \textquotesingle{}pandas.core.frame.DataFrame\textquotesingle{}\textgreater{}}}, \emph{\DUrole{n}{\textless{}class \textquotesingle{}pandas.core.frame.DataFrame\textquotesingle{}\textgreater{}}}}{}
\pysigstopsignatures
\sphinxAtStartPar
Clustering the grids with kmeans algorithm


\subparagraph{Parameters}
\label{\detokenize{classification/classification_steps/clustering:id3}}\begin{description}
\sphinxlineitem{df\_parameters\_of\_grids}{[}DataFrame{]}
\sphinxAtStartPar
Grids with parameters to be clustered

\sphinxlineitem{list\_of\_clustering\_parameters}{[}list of strings{]}
\sphinxAtStartPar
Parameters used for clustering.

\sphinxlineitem{n\_clusters: int}
\sphinxAtStartPar
Number of clusters.

\end{description}


\subparagraph{Returns}
\label{\detokenize{classification/classification_steps/clustering:id4}}\begin{description}
\sphinxlineitem{Dataframe:}
\sphinxAtStartPar
Grids that are attributed to a cluster.

\sphinxlineitem{Dataframe:}
\sphinxAtStartPar
Grids that are medoids (cluster centers).

\end{description}

\end{fulllineitems}


\sphinxAtStartPar
In each of the functions above the following function is called:
\index{reindex\_cluster\_indices() (in module classification.clustering.clustering\_algorithms)@\spxentry{reindex\_cluster\_indices()}\spxextra{in module classification.clustering.clustering\_algorithms}}

\begin{fulllineitems}
\phantomsection\label{\detokenize{classification/classification_steps/clustering:classification.clustering.clustering_algorithms.reindex_cluster_indices}}
\pysigstartsignatures
\pysiglinewithargsret{\sphinxcode{\sphinxupquote{classification.clustering.clustering\_algorithms.}}\sphinxbfcode{\sphinxupquote{reindex\_cluster\_indices}}}{\emph{\DUrole{n}{df\_parameters\_of\_grids: \textasciitilde{}pandas.core.frame.DataFrame}}, \emph{\DUrole{n}{representative\_networks: \textasciitilde{}pandas.core.frame.DataFrame) \sphinxhyphen{}\textgreater{} (\textless{}class \textquotesingle{}pandas.core.frame.DataFrame\textquotesingle{}\textgreater{}}}, \emph{\DUrole{n}{\textless{}class \textquotesingle{}pandas.core.frame.DataFrame\textquotesingle{}\textgreater{}}}}{}
\pysigstopsignatures
\sphinxAtStartPar
sort the cluster indices by the representatives networks number of households ascending.
this means the representative grid of cluster index 0 has the least number of households
\begin{quote}\begin{description}
\sphinxlineitem{Parameters}\begin{itemize}
\item {} 
\sphinxAtStartPar
\sphinxstyleliteralstrong{\sphinxupquote{df\_parameters\_of\_grids}} (\sphinxstyleliteralemphasis{\sphinxupquote{pd.DataFrame}}) \textendash{} set of parameters for grids that are clustered and thus have a column named ‘clusters’

\item {} 
\sphinxAtStartPar
\sphinxstyleliteralstrong{\sphinxupquote{representative\_networks}} (\sphinxstyleliteralemphasis{\sphinxupquote{pd.DataFrame}}) \textendash{} set of representative grids

\end{itemize}

\sphinxlineitem{Returns}
\sphinxAtStartPar
df\_parameters\_of\_grids with re\sphinxhyphen{}indexed clusters

\sphinxlineitem{Return type}
\sphinxAtStartPar
pd.DataFrame

\sphinxlineitem{Returns}
\sphinxAtStartPar
representative\_networks with re\sphinxhyphen{}indexed clusters

\sphinxlineitem{Return type}
\sphinxAtStartPar
pd.DataFrame

\end{description}\end{quote}

\end{fulllineitems}


\sphinxAtStartPar
This is a feature for better interpretability of the returned representative grids and grid parameters.

\sphinxAtStartPar
The cluster results can be made visiualised in QGIS by running:

\sphinxstepscope


\subsection{Visualisation}
\label{\detokenize{visualisation/index:visualisation}}\label{\detokenize{visualisation/index::doc}}

\subsubsection{Contents}
\label{\detokenize{visualisation/index:contents}}
\sphinxstepscope


\paragraph{QGIS Visualisation}
\label{\detokenize{visualisation/qgis/qgis:qgis-visualisation}}\label{\detokenize{visualisation/qgis/qgis::doc}}
\sphinxAtStartPar
Open QGIS.


\subparagraph{Visualisation structure}
\label{\detokenize{visualisation/qgis/qgis:visualisation-structure}}
\sphinxAtStartPar
The visualisation takes the geometry columns from the pylovo database. The layer names in QGIS correspond to the
database table names. In QGIS the layers have been grouped:

\noindent\sphinxincludegraphics[width=200\sphinxpxdimen]{{layer_menu}.png}

\sphinxAtStartPar
Individual elements or groups can be selected for visualisation.


\subparagraph{Raw Data}
\label{\detokenize{visualisation/qgis/qgis:raw-data}}
\sphinxAtStartPar
Look at the geodata pylovo uses to generate the grids. This data does not change for different versions.

\noindent\sphinxincludegraphics[width=600\sphinxpxdimen]{{qgis5}.png}

\sphinxAtStartPar
If you find any irregularities in your grid:
Check the raw data: Have the buildings been imported as expected? Are the ways complete and connected?


\subparagraph{Grids}
\label{\detokenize{visualisation/qgis/qgis:grids}}
\sphinxAtStartPar
These are the results of the pylovo grid generation. If you have created networks for the same PLZ with multiple
versions
make sure to filter the version you would like to see.
Click on the filter symbol next to the layer name. In the query panel you can apply a filter like:
\sphinxtitleref{“version\_id”=’3.8’}.

\noindent\sphinxincludegraphics[width=600\sphinxpxdimen]{{qgis6}.png}

\sphinxAtStartPar
If you would only like to see a specific grid enter a query like:
\sphinxtitleref{“version\_id” = ‘3.3’ AND “in\_building\_cluster” = ‘5’ AND “k\_mean\_cluster” = ‘1’}.

\sphinxAtStartPar
Pylovo partitions the buildings of a PLZ for a grid using k\sphinxhyphen{}means cluster (kcid, k\sphinxhyphen{}means cluster ID) and
building cluster (bcid, builings cluster ID).

\sphinxAtStartPar
Some more specific information about the different layers:


\subparagraph{Transformers}
\label{\detokenize{visualisation/qgis/qgis:transformers}}\begin{itemize}
\item {} 
\sphinxAtStartPar
black circle with white filling

\item {} 
\sphinxAtStartPar
within circle: cluster \sphinxhyphen{} ID: {[}kcid{]}.{[}bcid{]}

\item {} 
\sphinxAtStartPar
e.g. 5.18: kcid 5, bcid 18

\item {} 
\sphinxAtStartPar
transformers that a negative bcid are real transormers that were imported from OSM e.g. 1.\sphinxhyphen{}1

\end{itemize}


\subparagraph{Buildings}
\label{\detokenize{visualisation/qgis/qgis:buildings}}
\sphinxAtStartPar
The buildings (consumers) of each grid are colored in a different colour.
If you visualise a new postcode area that has a cluster ids that has not previously existed, e.g. 9.59 and 10.01,
they will have the same colour.
In this case douple click on the layer. Go to the \sphinxtitleref{symbol} tab. click on \sphinxtitleref{delete all}, \sphinxtitleref{classify}.
Now for all cluster IDs the colours will be newly created.


\subparagraph{Postcode}
\label{\detokenize{visualisation/qgis/qgis:postcode}}
\sphinxAtStartPar
On the outside of the postcode area marked in red, you will find the PLZ code.


\subparagraph{Clusters}
\label{\detokenize{visualisation/qgis/qgis:clusters}}
\sphinxAtStartPar
By selecting the cluster layer, the cluster index for each grid can be shown. It can be identified by the color of the
transformer.

\noindent\sphinxincludegraphics[width=800\sphinxpxdimen]{{images/visualisation/clusters_qgis}.png}


\subparagraph{Representative Grids}
\label{\detokenize{visualisation/qgis/qgis:representative-grids}}
\sphinxAtStartPar
Similarly the representative grids of the clusters can be shown.

\noindent\sphinxincludegraphics[width=800\sphinxpxdimen]{{images/visualisation/rep_grids_qgis}.png}


\subparagraph{Visualize from csv}
\label{\detokenize{visualisation/qgis/qgis:visualize-from-csv}}
\sphinxAtStartPar
With the files \sphinxcode{\sphinxupquote{export\_grid\_gis\_data\_as\_csv}} and \sphinxcode{\sphinxupquote{export\_grid\_gis\_data\_as\_csv\_for\_multiple\_plz}} the
pandapower networks’ information can exported to csv.  This is performed by the function
\index{get\_bus\_line\_geo\_for\_network() (in module plotting.export\_net)@\spxentry{get\_bus\_line\_geo\_for\_network()}\spxextra{in module plotting.export\_net}}

\begin{fulllineitems}
\phantomsection\label{\detokenize{visualisation/qgis/qgis:plotting.export_net.get_bus_line_geo_for_network}}
\pysigstartsignatures
\pysiglinewithargsret{\sphinxcode{\sphinxupquote{plotting.export\_net.}}\sphinxbfcode{\sphinxupquote{get\_bus\_line\_geo\_for\_network}}}{\emph{\DUrole{n}{pandapower\_net}}, \emph{\DUrole{n}{plz}}, \emph{\DUrole{n}{net\_index}\DUrole{o}{=}\DUrole{default_value}{0}}}{}
\pysigstopsignatures
\sphinxAtStartPar
get bus and line data for a single pandapower net,
export lines (cables) and buses (trafo position, consumers, connections) as geometric elements in csv table
to be used in qgis

\end{fulllineitems}


\sphinxAtStartPar
This way additional function about the grids can be visualised. This function would also enable to show grids in QGIS
that are not on the database.

\noindent\sphinxincludegraphics[width=800\sphinxpxdimen]{{net_description}.png}

\sphinxstepscope


\paragraph{Plotting Networks}
\label{\detokenize{visualisation/plotting/index:plotting-networks}}\label{\detokenize{visualisation/plotting/index::doc}}
\sphinxAtStartPar
The configuration of plot characteristics can be set in config\_plots

\sphinxAtStartPar
see also examples notebooks to see the use of the functions

\sphinxstepscope


\subparagraph{Plot Networks}
\label{\detokenize{visualisation/plotting/plot_networks:plot-networks}}\label{\detokenize{visualisation/plotting/plot_networks::doc}}
\sphinxAtStartPar
There are different ways to plot grids listed below:


\subparagraph{Network Representation}
\label{\detokenize{visualisation/plotting/plot_networks:network-representation}}
\noindent\sphinxincludegraphics[width=400\sphinxpxdimen]{{contextily}.png}

\sphinxAtStartPar
If the contextily basemap cannot be loaded, change the zoomfactor.

\noindent\sphinxincludegraphics[width=400\sphinxpxdimen]{{simple_grid}.png}

\noindent\sphinxincludegraphics[width=400\sphinxpxdimen]{{grid_on_map}.png}


\subparagraph{Tree Graph Representation}
\label{\detokenize{visualisation/plotting/plot_networks:tree-graph-representation}}
\noindent\sphinxincludegraphics[width=400\sphinxpxdimen]{{generic_plot}.png}

\noindent\sphinxincludegraphics[width=400\sphinxpxdimen]{{tree_network}.png}

\noindent\sphinxincludegraphics[width=400\sphinxpxdimen]{{tree_network_improved_spacing}.png}

\noindent\sphinxincludegraphics[width=400\sphinxpxdimen]{{radial_network}.png}

\sphinxstepscope


\subparagraph{Plot Network Data per PLZ}
\label{\detokenize{visualisation/plotting/plot_network_data_per_plz:plot-network-data-per-plz}}\label{\detokenize{visualisation/plotting/plot_network_data_per_plz::doc}}
\noindent\sphinxincludegraphics[width=600\sphinxpxdimen]{{pie_plot}.png}

\noindent\sphinxincludegraphics[width=600\sphinxpxdimen]{{hist_trafos}.png}

\noindent\sphinxincludegraphics[width=600\sphinxpxdimen]{{boxplot_plz}.png}
\index{plot\_trafo\_on\_map() (syngrid.GridGenerator method)@\spxentry{plot\_trafo\_on\_map()}\spxextra{syngrid.GridGenerator method}}

\begin{fulllineitems}
\phantomsection\label{\detokenize{visualisation/plotting/plot_network_data_per_plz:syngrid.GridGenerator.plot_trafo_on_map}}
\pysigstartsignatures
\pysiglinewithargsret{\sphinxcode{\sphinxupquote{syngrid.GridGenerator.}}\sphinxbfcode{\sphinxupquote{plot\_trafo\_on\_map}}}{}{}
\pysigstopsignatures
\end{fulllineitems}


\noindent\sphinxincludegraphics[width=600\sphinxpxdimen]{{trafos_on_map}.png}

\noindent\sphinxincludegraphics[width=600\sphinxpxdimen]{{cable_type_distribution}.png}

\sphinxstepscope


\subparagraph{Plot Graphs for Classification}
\label{\detokenize{visualisation/plotting/plot_classification:plot-graphs-for-classification}}\label{\detokenize{visualisation/plotting/plot_classification::doc}}
\sphinxAtStartPar
The plots to visualise results from the classification process are:


\subparagraph{Sampling}
\label{\detokenize{visualisation/plotting/plot_classification:sampling}}
\noindent\sphinxincludegraphics[width=400\sphinxpxdimen]{{images/visualisation/karte_samples2}.png}

\noindent\sphinxincludegraphics[width=400\sphinxpxdimen]{{images/visualisation/anz_samples}.png}


\subparagraph{Parameter Reduction}
\label{\detokenize{visualisation/plotting/plot_classification:parameter-reduction}}
\noindent\sphinxincludegraphics[width=400\sphinxpxdimen]{{images/visualisation/karte_samples2}.png}

\noindent\sphinxincludegraphics[width=400\sphinxpxdimen]{{images/visualisation/anz_samples}.png}

\sphinxstepscope

\sphinxAtStartPar
sphinx\sphinxhyphen{}quickstart on Wed Jul 12 12:49:48 2023.
You can adapt this file completely to your liking, but it should at least
contain the root \sphinxtitleref{toctree} directive.


\subsection{Pylovo GUI}
\label{\detokenize{docs_gui/index:pylovo-gui}}\label{\detokenize{docs_gui/index::doc}}
\sphinxAtStartPar
Pylovo is a python tool for low\sphinxhyphen{}voltage distribution grid generation. This documentation
concerns itself mainly with the code and usage of its web\sphinxhyphen{}based user interface.

\sphinxAtStartPar
Check out the {\hyperref[\detokenize{docs_pylovo/installation::doc}]{\sphinxcrossref{\DUrole{doc}{Installation}}}} section for further information on how to setup the project.


\subsubsection{Contents}
\label{\detokenize{docs_gui/index:contents}}
\sphinxstepscope


\paragraph{Usage}
\label{\detokenize{docs_gui/usage/usage:usage}}\label{\detokenize{docs_gui/usage/usage::doc}}

\subparagraph{Starting the webserver}
\label{\detokenize{docs_gui/usage/usage:starting-the-webserver}}
\sphinxAtStartPar
First, you need to activate the virtual environment, navigate to the tool directory and start the flask server

\begin{sphinxVerbatim}[commandchars=\\\{\}]
\PYG{c+c1}{\PYGZsh{}if you gave your env another name, use that one instead of TUM\PYGZus{}Syngrid}
\PYG{n}{conda} \PYG{n}{activate} \PYG{n}{TUM\PYGZus{}Syngrid}
\PYG{n}{cd} \PYG{n}{path}\PYG{o}{/}\PYG{n}{to}\PYG{o}{/}\PYG{n}{pylovo}\PYG{o}{/}\PYG{n}{gui}\PYG{o}{/}\PYG{n}{IDP\PYGZus{}Maptool\PYGZus{}Flask}
\PYG{n}{flask} \PYG{o}{\PYGZhy{}}\PYG{o}{\PYGZhy{}}\PYG{n}{app} \PYG{n}{maptool} \PYG{o}{\PYGZhy{}}\PYG{o}{\PYGZhy{}}\PYG{n}{debug} \PYG{n}{run}
\end{sphinxVerbatim}

\sphinxAtStartPar
We are setting multiple flags for running the server:
\begin{itemize}
\item {} 
\sphinxAtStartPar
\sphinxstylestrong{app}:  The name of the folder containing the init file for the flask code, in this case maptool

\item {} 
\sphinxAtStartPar
\sphinxstylestrong{debug}: If this flag is not set, the server will not automatically restart, if adjustments to the code are made

\end{itemize}

\sphinxAtStartPar
Make sure you have connected to the LRZ net via EduVPN, otherwise the tool will not be able to access the pylovo database.


\subparagraph{Accessing the tool}
\label{\detokenize{docs_gui/usage/usage:accessing-the-tool}}
\begin{DUlineblock}{0em}
\item[] Open a browser of your choice and enter the address \sphinxurl{http://127.0.0.1:5000}.
\item[] You should now have the following view:
\end{DUlineblock}

\noindent\sphinxincludegraphics[width=800\sphinxpxdimen]{{maptool_view_default}.png}

\begin{DUlineblock}{0em}
\item[] The tool opens to the area selection page by default.
\end{DUlineblock}


\subparagraph{Using the tool}
\label{\detokenize{docs_gui/usage/usage:using-the-tool}}
\sphinxstepscope


\subparagraph{Area Selection}
\label{\detokenize{docs_gui/usage/postcode_editor:area-selection}}\label{\detokenize{docs_gui/usage/postcode_editor::doc}}
\sphinxAtStartPar
This is the page that will open by default by navigating to \sphinxurl{http://127.0.0.1:5000}.
Here the user can either draw shapes directly on the map to delineate areas in which new networks should be generated or ask the database
for previously generated networks by submitting an ID.


\subparagraph{Visual overview}
\label{\detokenize{docs_gui/usage/postcode_editor:visual-overview}}
\noindent\sphinxincludegraphics[width=800\sphinxpxdimen]{{maptool_view_area_selection_explained}.png}

\sphinxAtStartPar
NOTE: If you have never used the GUI before, selecting Network/Urbs Setup/Urbs Results in the editor selection (1) will cause errors as the tool will check for local files that
do not exist yet.


\subparagraph{Network generation by area}
\label{\detokenize{docs_gui/usage/postcode_editor:network-generation-by-area}}

\subparagraph{Area selection}
\label{\detokenize{docs_gui/usage/postcode_editor:id1}}
\begin{DUlineblock}{0em}
\item[] The user can draw shapes directly on the map using the toolbar in the upper left corner of the screen.
Buttons 2.1\sphinxhyphen{}2.3 in the visual overview allow the creation of different basic shape types,
while buttons 2.4\sphinxhyphen{}2.7 allow editing already created shapes.
\item[] Once a shape has been drawn, the \sphinxstylestrong{Select Area} button (4) will be enabled.
\end{DUlineblock}

\begin{DUlineblock}{0em}
\item[] \sphinxstylestrong{Important note}: Only a single shape is allowed at any given time. Finishing drawing a new shape will delete any previously drawn shapes.
\end{DUlineblock}


\subparagraph{Building selection \& network generation}
\label{\detokenize{docs_gui/usage/postcode_editor:building-selection-network-generation}}
\noindent\sphinxincludegraphics[width=0.490\linewidth]{{maptool_view_area_selection_buildings_explained}.png}

\noindent\sphinxincludegraphics[width=0.490\linewidth]{{maptool_view_area_selection_buildings_explained_delete}.png}

\begin{DUlineblock}{0em}
\item[] Once the user presses the \sphinxstylestrong{Select Area} button, all buildings within the selected area will be displayed on the map. The user can now delete
individual buildings by clicking on them and then clicking the \sphinxstylestrong{delete Building} button in the popup window.
\item[] Once the user is happy with their selection of buildings, they can generate a network based on their selection by pressing the \sphinxstylestrong{Generate Network} button.
\end{DUlineblock}


\subparagraph{Network generation \& selection by ID}
\label{\detokenize{docs_gui/usage/postcode_editor:network-generation-selection-by-id}}
\noindent\sphinxincludegraphics[width=0.490\linewidth]{{maptool_view_id_selection_version_explained}.png}

\noindent\sphinxincludegraphics[width=0.490\linewidth]{{maptool_view_id_selection_explained}.png}

\begin{DUlineblock}{0em}
\item[] If a user instead wants to work on a previously generated network, they can do so by supplying the associated ID code.
After entering it into the ID selection field (3.1) and pressing the \sphinxstylestrong{Submit} button (3.2), the tool will ask the user to select an available
version of the networks and display all networks for a given ID and version on the map.
\item[] The user can select a network either by clicking on it or selecting it from the list on the right side of the window.
\item[] Finally, they can confirm their selection by pressing the \sphinxstylestrong{Select Network} button on the bottom right. The button will remain disabled until
the user has selected a network
\end{DUlineblock}

\sphinxstepscope


\subparagraph{Network Editor}
\label{\detokenize{docs_gui/usage/network_editor:network-editor}}\label{\detokenize{docs_gui/usage/network_editor::doc}}
\sphinxAtStartPar
In the network editor view the user can fine\sphinxhyphen{}tune individual features and extend the network by adding or removing new features, before handing over the network for
setup of the urbs run.


\subparagraph{Features}
\label{\detokenize{docs_gui/usage/network_editor:features}}

\subparagraph{Bus}
\label{\detokenize{docs_gui/usage/network_editor:bus}}
\begin{DUlineblock}{0em}
\item[] The nodes of the network graph. The synthetic grids generated by pylovo distinguish between connection node buses and consumer node buses.
Any bus that has a load attached is considered a consumer bus.
\end{DUlineblock}


\subparagraph{Line}
\label{\detokenize{docs_gui/usage/network_editor:line}}
\begin{DUlineblock}{0em}
\item[] The paths of the network graph. Each line is connected to two buses.
\end{DUlineblock}


\subparagraph{Trafo}
\label{\detokenize{docs_gui/usage/network_editor:trafo}}
\begin{DUlineblock}{0em}
\item[] A transformer for connecting higher and lower voltage levels. The high voltage bus the trafo is attached to is also the bus the external grid is connected to.
\end{DUlineblock}


\subparagraph{Ext\_grid}
\label{\detokenize{docs_gui/usage/network_editor:ext-grid}}
\begin{DUlineblock}{0em}
\item[] The connection point of an external electrical grid to the local editable grid. There can only be one ext\_grid for each network.
\end{DUlineblock}


\subparagraph{Std\_type}
\label{\detokenize{docs_gui/usage/network_editor:std-type}}
\begin{DUlineblock}{0em}
\item[] Std\_types refer to the types of cables and transformators we can define our lines and trafo as. Changes to a std\_type are applied to all features
that have this std\_type assigned to them.
\end{DUlineblock}


\subparagraph{Editing features}
\label{\detokenize{docs_gui/usage/network_editor:editing-features}}
\begin{DUlineblock}{0em}
\item[] The user can select a feature either by clicking on it on the map or by selecting the feature type from the button column on the left side of the screen and then choosing
the feature they want to edit from the newly appeared list.
\end{DUlineblock}


\subparagraph{Editing std\_type features}
\label{\detokenize{docs_gui/usage/network_editor:editing-std-type-features}}
\begin{DUlineblock}{0em}
\item[] Std\_types are assigned to lines and trafos. Since they have to be consistent for all lines/trafos their properties can only be adjusted in the std\_type editor accessible
over the left\sphinxhyphen{}side button column. Std\_type properties are still listed in the line \& trafo editors, but cannot be changed. Instead you select which std\_type you want to
assign to each trafo \& line individually.
\end{DUlineblock}


\subparagraph{Adding features}
\label{\detokenize{docs_gui/usage/network_editor:adding-features}}
\begin{DUlineblock}{0em}
\item[] Every feature type list has an “Add {[}Feature{]}” button at the top. Clicking this button changes the map settings to draw mode. You can exit draw mode at any time
by pressing \sphinxstylestrong{esc}
\end{DUlineblock}


\subparagraph{Adding buses}
\label{\detokenize{docs_gui/usage/network_editor:adding-buses}}
\begin{DUlineblock}{0em}
\item[] Buses can be added anywhere on the map without restrictions. Simply click on the map where you want the bus to appear. The corresponding editor window will
automatically be opened.
\item[] IMPORTANT NOTE: The tool makes no assumptions about where a bus is placed. It is up to the user to ensure the placement makes logical sense in the context of the
current network.
\end{DUlineblock}


\subparagraph{Adding lines}
\label{\detokenize{docs_gui/usage/network_editor:adding-lines}}
\begin{DUlineblock}{0em}
\item[] Lines need to connect to two different buses to be viable. The user needs to start drawing by clicking on a bus, otherwise the tool will immediately exit draw mode.
After starting from a bus, the user can draw as many intermediate points as they want. To finish creating the line the user needs to double click on a bus again.
\item[] IMPORTANT  NOTE: Using urbs requires there be no circular line connections in the network. At the moment the tool does not check that this is the case. It is up to the user
to make sure newly created lines do not create circles in the network graph.
\end{DUlineblock}


\subparagraph{Adding trafos}
\label{\detokenize{docs_gui/usage/network_editor:adding-trafos}}
\begin{DUlineblock}{0em}
\item[] Adding a trafo works the same way as adding a line. However, the network only allows for one trafo at a time. As long as a trafo exists,
the “Add Trafo” button will be unavailable.
\end{DUlineblock}


\subparagraph{Adding external grids}
\label{\detokenize{docs_gui/usage/network_editor:adding-external-grids}}
\begin{DUlineblock}{0em}
\item[] Adding external grids works much in the same way as adding buses, with the difference that external grids need to be placed on an already existing bus.
Once the user has entered draw mode, they can create a new grid by double clicking on a bus on the map.
\item[] Similarly to the trafo, we only allow one external grid at a time. While an ext\_grid exists, the Add button will be disabled.
\end{DUlineblock}


\subparagraph{Adding secondary features}
\label{\detokenize{docs_gui/usage/network_editor:adding-secondary-features}}
\begin{DUlineblock}{0em}
\item[] Only buses have secondary features.
\item[] While all of them can be added by clicking the corresponding button in the bus editor window, only loads are relevant for the actual usage of the tool
as of right now.
\item[] We determine whether a bus will be interpreted as a node to which we can attach processes in the urbs setup step, by the fact that they have at least
one load attached to them.
\end{DUlineblock}


\subparagraph{Deleting features}
\label{\detokenize{docs_gui/usage/network_editor:deleting-features}}
\noindent\sphinxincludegraphics{{maptool_view_feature_delete_explained}.png}

\begin{DUlineblock}{0em}
\item[] All network features can be deleted by clicking on the “Delete Feature” button in the editor window.
\item[] For lines, trafos and ext\_grids, clicking this button will instantly delete the feature. Since buses are referenced by other
features, said features will also be deleted, if you choose to remove a bus. The affected features will be highlighted red on the map
and the user will be prompted on whether they want to actually go ahead with the deletion.
\end{DUlineblock}

\sphinxstepscope


\subparagraph{Urbs Setup Editor}
\label{\detokenize{docs_gui/usage/urbs_setup:urbs-setup-editor}}\label{\detokenize{docs_gui/usage/urbs_setup::doc}}
\begin{DUlineblock}{0em}
\item[] This editor is used for setting up all parameters we need for the eventual run of our optimization model
via urbs.
\end{DUlineblock}


\subparagraph{Features}
\label{\detokenize{docs_gui/usage/urbs_setup:features}}

\subparagraph{Building}
\label{\detokenize{docs_gui/usage/urbs_setup:building}}
\begin{DUlineblock}{0em}
\item[] Here parameters for individual buildings are set. Each building corresponds to a bus with load from previous steps. The data entered here is supplemented with additional information
fetched from the database.
\end{DUlineblock}


\subparagraph{Demand}
\label{\detokenize{docs_gui/usage/urbs_setup:demand}}
\begin{DUlineblock}{0em}
\item[] Demands are time series for an entire year measured in hours describing how much of a commodity is requested at any given hour. At the moment demand profiles are predefined for
a select few commodities. For these commodities the user can select as many profiles as they want.
\end{DUlineblock}


\subparagraph{Transmission}
\label{\detokenize{docs_gui/usage/urbs_setup:transmission}}
\begin{DUlineblock}{0em}
\item[] Describes possible transports of commodities between buses. The default trafo profile “kont” is generated based on the “sn\_mva” of the trafo set in the network editor step.
\end{DUlineblock}


\subparagraph{Global}
\label{\detokenize{docs_gui/usage/urbs_setup:global}}
\begin{DUlineblock}{0em}
\item[] Here we set a few global values needed for the urbs run.
\end{DUlineblock}


\subparagraph{Commodity}
\label{\detokenize{docs_gui/usage/urbs_setup:commodity}}
\begin{DUlineblock}{0em}
\item[] Commodities are goods that can be generated, stored, transmitted or consumed. They come in several types:
\end{DUlineblock}
\begin{itemize}
\item {} 
\sphinxAtStartPar
\sphinxstylestrong{Buy:} commodities which can be sold with a buy price that may vary for each time step

\item {} 
\sphinxAtStartPar
\sphinxstylestrong{Sell:} commodities which can be sold with a sell price that may vary for each time step

\item {} 
\sphinxAtStartPar
\sphinxstylestrong{SupIm:} stands for Supply Intermittent commodities for which availability is not constant

\item {} 
\sphinxAtStartPar
\sphinxstylestrong{Stock:} commodities which may be purchased for a fixed price

\item {} 
\sphinxAtStartPar
\sphinxstylestrong{Demand:} requested commodities of the energy system, usually end products of the model

\end{itemize}

\begin{DUlineblock}{0em}
\item[] Since demand and SupIm rely on predefined profiles at the moment, we recommend the user not to set them as types.
\item[] Commodities can and must be attached to processes to be considered in the urbs model
\end{DUlineblock}


\subparagraph{Process}
\label{\detokenize{docs_gui/usage/urbs_setup:process}}
\begin{DUlineblock}{0em}
\item[] Processes change input commodities into output commodities. Each process can have multiple commodities attached to itself, regardless of any other process.
\end{DUlineblock}


\subparagraph{Process\sphinxhyphen{}Commodity}
\label{\detokenize{docs_gui/usage/urbs_setup:process-commodity}}
\begin{DUlineblock}{0em}
\item[] Process\sphinxhyphen{}Commodities are secondary features attached to individual processes. They are defined as input or output commodities.
\end{DUlineblock}


\subparagraph{Process\sphinxhyphen{}Configuration}
\label{\detokenize{docs_gui/usage/urbs_setup:process-configuration}}
\begin{DUlineblock}{0em}
\item[] A table that defines how many of each process are attached to each bus with loads. We distinguish between 3 value types:
\end{DUlineblock}
\begin{itemize}
\item {} 
\sphinxAtStartPar
\sphinxstylestrong{Empty field:} It is forbidden to add any amount of this process to this bus

\item {} 
\sphinxAtStartPar
\sphinxstylestrong{field value = 0:} There are currently none of this process attached to this bus, but it is possible for the optimization model to add to it during runtime

\item {} 
\sphinxAtStartPar
\sphinxstylestrong{field value \textgreater{} 0:} Sets an upper bound for how many of this process can be attached to this bus

\end{itemize}


\subparagraph{Storage}
\label{\detokenize{docs_gui/usage/urbs_setup:storage}}
\begin{DUlineblock}{0em}
\item[] Describes multiple technical facilities that can store energy. Each is connected to one commodity.
\end{DUlineblock}


\subparagraph{Storage\sphinxhyphen{}Configuration}
\label{\detokenize{docs_gui/usage/urbs_setup:storage-configuration}}
\begin{DUlineblock}{0em}
\item[] A table that defines how many of each storage type are attached to each bus with loads. We distinguish between 3 input types:
\end{DUlineblock}
\begin{itemize}
\item {} 
\sphinxAtStartPar
\sphinxstylestrong{Empty field:} It is forbidden to add any amount of this storage to this bus

\item {} 
\sphinxAtStartPar
\sphinxstylestrong{field value = 0:} There are currently none of this storage attached to this bus, but it is possible for the optimization model to add to it during runtime

\item {} 
\sphinxAtStartPar
\sphinxstylestrong{field value \textgreater{} 0:} Sets an upper bound for how many of this storage can be attached to this bus

\end{itemize}


\subparagraph{SupIm}
\label{\detokenize{docs_gui/usage/urbs_setup:supim}}
\begin{DUlineblock}{0em}
\item[] SupIm (Supply Intermittent) are time series for an entire year measured in hours describing how much of a commodity of type SupIm is requested at any given hour. At the moment demand profiles are predefined for
a select few commodities. For these commodities the user can select as many profiles as they want.
\end{DUlineblock}


\subparagraph{Adding commodities}
\label{\detokenize{docs_gui/usage/urbs_setup:adding-commodities}}
\begin{DUlineblock}{0em}
\item[] You can add new commodities by selecting the commodity (C) tab on right side of the screen and pressing the “Add Commodity” button.
\item[] A popup window will open. Enter the name of the new commodity and click the “Add Commodity” button to finish the process.
\end{DUlineblock}


\subparagraph{Adding Buy\sphinxhyphen{}Sell\sphinxhyphen{}Price}
\label{\detokenize{docs_gui/usage/urbs_setup:adding-buy-sell-price}}
\sphinxAtStartPar
This feature is work in progress


\subparagraph{Adding processes}
\label{\detokenize{docs_gui/usage/urbs_setup:adding-processes}}
\begin{DUlineblock}{0em}
\item[] You can add new commodities by selecting the process (P) tab on right side of the screen and pressing the “Add Process” button.
\item[] A popup window will open. Adding a process requires attaching at least one commodity to it. You can select one of the pre\sphinxhyphen{}existing commodities via the dropdown menu.
\item[] You can also choose to select “New Commodity” from the dropdown. After setting the name and clicking “Create Process” the process and the new commodity will be added to their
respective lists.
\end{DUlineblock}


\subparagraph{Adding process commodities}
\label{\detokenize{docs_gui/usage/urbs_setup:adding-process-commodities}}
\begin{DUlineblock}{0em}
\item[] You have to add one commodity on creation of a given process. Additional process commodities can be added by selecting a process in the GUI and scrolling down to the
bottom of the editor window. Clicking on the “Add PRO\_COM\_PROP” button will allow you to add a preexisting or new commodity and set it as input or output commodity.
\end{DUlineblock}

\sphinxstepscope


\subparagraph{Urbs Results}
\label{\detokenize{docs_gui/usage/urbs_results:urbs-results}}\label{\detokenize{docs_gui/usage/urbs_results::doc}}
\begin{DUlineblock}{0em}
\item[] This window visualizes the results of the urbs run through predefined plots. In the backend data is extracted from the urbs hdf5 output file and
converted into a plot, which is converted to html and sent to the frontend.
\end{DUlineblock}


\subparagraph{Showing outputs}
\label{\detokenize{docs_gui/usage/urbs_results:showing-outputs}}
\begin{DUlineblock}{0em}
\item[] You can show the output for any given feature by either selecting it on the map or via the tabs on the right side of the screen
\end{DUlineblock}


\subparagraph{Defining outputs}
\label{\detokenize{docs_gui/usage/urbs_results:defining-outputs}}
\begin{DUlineblock}{0em}
\item[] All plots are defined in the file \sphinxstylestrong{urbs\_results\_plotting.py} and called in the function \sphinxstyleemphasis{urbs\_results\_generate\_plot} of the corresponding \sphinxstylestrong{routes.py} file.
\item[] The user can define additional plots and call them by adding them to the \sphinxstylestrong{routes.py} file. Every plot function needs to fulfill certain criteria to work properly.
\item[] Each function needs to take the following inputs:
\end{DUlineblock}
\begin{itemize}
\item {} 
\sphinxAtStartPar
\sphinxstylestrong{hdf\_path}: path to the hdf5 save location, generated automatically by the \sphinxstylestrong{routes.py} code, just needs to be added as a variable to the function call

\item {} 
\sphinxAtStartPar
\sphinxstylestrong{site\_name}: for buses we need a single site name, for lines both site names as keys to query the hdf5 file for data. The names are returned from the frontend by default and simply need to be added to the function calls as variables

\item {} 
\sphinxAtStartPar
\sphinxstylestrong{save\_path}: path to the plot save location. It is automatically set in the function definition and should not be changed by the user

\end{itemize}

\begin{DUlineblock}{0em}
\item[] Furthermore the function needs to return the filename the user has defined within the plotting function.
\item[] Last but not least the user needs to call the plotly \sphinxstyleemphasis{write\_html} function with the following parameters:
\end{DUlineblock}
\begin{itemize}
\item {} 
\sphinxAtStartPar
\sphinxstylestrong{file= save\_path + filename + “.html”}: Sets the absolute save path of the plot

\item {} 
\sphinxAtStartPar
\sphinxstylestrong{full\_html = false:}: makes sure we generate a single div to be inserted into our existing webpage instead of generating a full independent webpage

\item {} 
\sphinxAtStartPar
\sphinxstylestrong{include\_plotlyjs=False}: we don’t need to insert the plotlyjs code into our html because we are adding it via CDN in our base html code

\item {} 
\sphinxAtStartPar
\sphinxstylestrong{div\_id=site\_name + filename}: we need to give the html div generated from the plot a unique ID for later insertion in the frontend

\end{itemize}

\sphinxstepscope


\paragraph{Development}
\label{\detokenize{docs_gui/dev/development:development}}\label{\detokenize{docs_gui/dev/development::doc}}

\subparagraph{Tool execution flow}
\label{\detokenize{docs_gui/dev/development:tool-execution-flow}}

\subparagraph{Postcode editor}
\label{\detokenize{docs_gui/dev/development:postcode-editor}}
\sphinxAtStartPar
There are two ways program execution can go in the postcode editor window. Either the user decides to request information about already
existing networks via a numerical ID or they draw a shape on the map to generate a new grid of networks.


\subparagraph{Area selection via ID}
\label{\detokenize{docs_gui/dev/development:area-selection-via-id}}
\noindent\sphinxincludegraphics{{pylovo_sequence_postcode_id}.png}


\subparagraph{Area selection via shape}
\label{\detokenize{docs_gui/dev/development:area-selection-via-shape}}
\noindent\sphinxincludegraphics{{pylovo_sequence_postcode_area}.png}

\sphinxAtStartPar
If the user decides to create a new grid via area selection, the sequence will continue the same way as with network selection via ID once the user
has decided to generate the new networks.


\subparagraph{Network editor}
\label{\detokenize{docs_gui/dev/development:network-editor}}

\subparagraph{Setup}
\label{\detokenize{docs_gui/dev/development:setup}}
\noindent\sphinxincludegraphics{{pylovo_sequence_network}.png}


\subparagraph{Usage}
\label{\detokenize{docs_gui/dev/development:usage}}
\noindent\sphinxincludegraphics{{pylovo_sequence_network_usage}.png}


\subparagraph{Urbs setup editor}
\label{\detokenize{docs_gui/dev/development:urbs-setup-editor}}

\subparagraph{Setup}
\label{\detokenize{docs_gui/dev/development:id1}}
\noindent\sphinxincludegraphics{{pylovo_sequence_urbs_setup}.png}


\subparagraph{Usage}
\label{\detokenize{docs_gui/dev/development:id2}}

\subparagraph{New commodity creation}
\label{\detokenize{docs_gui/dev/development:new-commodity-creation}}
\noindent\sphinxincludegraphics{{pylovo_sequence_urbs_setup_comm_creation}.png}


\subparagraph{New process creation}
\label{\detokenize{docs_gui/dev/development:new-process-creation}}
\noindent\sphinxincludegraphics{{pylovo_sequence_urbs_setup_process_creation}.png}

\sphinxstepscope


\paragraph{Javascript API Reference}
\label{\detokenize{docs_gui/js_api/index:javascript-api-reference}}\label{\detokenize{docs_gui/js_api/index::doc}}
\sphinxstepscope


\subparagraph{Postcode Editor}
\label{\detokenize{docs_gui/js_api/postcode_editor/index:postcode-editor}}\label{\detokenize{docs_gui/js_api/postcode_editor/index::doc}}
\sphinxstepscope


\subparagraph{display\_postcode.js}
\label{\detokenize{docs_gui/js_api/postcode_editor/display_postcode:display-postcode-js}}\label{\detokenize{docs_gui/js_api/postcode_editor/display_postcode::doc}}\index{createPLZIDDropdown() (built\sphinxhyphen{}in function)@\spxentry{createPLZIDDropdown()}\spxextra{built\sphinxhyphen{}in function}}

\begin{fulllineitems}
\phantomsection\label{\detokenize{docs_gui/js_api/postcode_editor/display_postcode:createPLZIDDropdown}}
\pysigstartsignatures
\pysiglinewithargsret{\sphinxbfcode{\sphinxupquote{\DUrole{n}{createPLZIDDropdown}}}}{}{}
\pysigstopsignatures
\sphinxAtStartPar
fetches all postcode ids for which results already exist in the database and writes them to a dropdown menu in the GUI
so that the user can select them without having to know exactly what exists in the db

\end{fulllineitems}

\index{selectVersionOfPostalCodeNetwork() (built\sphinxhyphen{}in function)@\spxentry{selectVersionOfPostalCodeNetwork()}\spxextra{built\sphinxhyphen{}in function}}

\begin{fulllineitems}
\phantomsection\label{\detokenize{docs_gui/js_api/postcode_editor/display_postcode:selectVersionOfPostalCodeNetwork}}
\pysigstartsignatures
\pysiglinewithargsret{\sphinxbfcode{\sphinxupquote{\DUrole{n}{selectVersionOfPostalCodeNetwork}}}}{}{}
\pysigstopsignatures
\sphinxAtStartPar
onclick function for submit plz button
returns all versions of network associated with the passed id and generates the radiobuttons for the version select gui element

\end{fulllineitems}

\index{chooseVersionOfPlzNetwork() (built\sphinxhyphen{}in function)@\spxentry{chooseVersionOfPlzNetwork()}\spxextra{built\sphinxhyphen{}in function}}

\begin{fulllineitems}
\phantomsection\label{\detokenize{docs_gui/js_api/postcode_editor/display_postcode:chooseVersionOfPlzNetwork}}
\pysigstartsignatures
\pysiglinewithargsret{\sphinxbfcode{\sphinxupquote{\DUrole{n}{chooseVersionOfPlzNetwork}}}}{}{}
\pysigstopsignatures
\sphinxAtStartPar
onclick function for the Choose Version button of the version select gui element
\begin{quote}\begin{description}
\sphinxlineitem{Returns}
\sphinxAtStartPar
nothing. Return statement exists simply to break out of function in case of faulty input

\end{description}\end{quote}

\end{fulllineitems}

\index{getPostalCodeAreaByID() (built\sphinxhyphen{}in function)@\spxentry{getPostalCodeAreaByID()}\spxextra{built\sphinxhyphen{}in function}}

\begin{fulllineitems}
\phantomsection\label{\detokenize{docs_gui/js_api/postcode_editor/display_postcode:getPostalCodeAreaByID}}
\pysigstartsignatures
\pysiglinewithargsret{\sphinxbfcode{\sphinxupquote{\DUrole{n}{getPostalCodeAreaByID}}}}{\emph{\DUrole{n}{plz}}}{}
\pysigstopsignatures
\sphinxAtStartPar
if the user wants to look at preexisting networks we initially post the plz again and get the outline of our network area as a geojson file,
which we display on the map
We then fetch all networks included in that area and display only the lines to avoid performance hits due to too many objects
\begin{quote}\begin{description}
\sphinxlineitem{Arguments}\begin{itemize}
\item {} 
\sphinxAtStartPar
\sphinxstyleliteralstrong{\sphinxupquote{plz}} (\sphinxstyleliteralemphasis{\sphinxupquote{int}}) \textendash{} the 5 or 6 digit ID with which a grid is associated

\end{itemize}

\end{description}\end{quote}

\end{fulllineitems}

\index{getPostalCodeAreaByShape() (built\sphinxhyphen{}in function)@\spxentry{getPostalCodeAreaByShape()}\spxextra{built\sphinxhyphen{}in function}}

\begin{fulllineitems}
\phantomsection\label{\detokenize{docs_gui/js_api/postcode_editor/display_postcode:getPostalCodeAreaByShape}}
\pysigstartsignatures
\pysiglinewithargsret{\sphinxbfcode{\sphinxupquote{\DUrole{n}{getPostalCodeAreaByShape}}}}{\emph{\DUrole{n}{btn}}}{}
\pysigstopsignatures
\sphinxAtStartPar
if the user wants to generate networks from a newly selected area shape, we initially return the shape the user has selected and receive a response
containing the building shapes contained in the selected area
We check res and oth buildings for emptiness and display them on the map
We also change the Select Area button to Generate Network button
\begin{quote}\begin{description}
\sphinxlineitem{Arguments}\begin{itemize}
\item {} 
\sphinxAtStartPar
\sphinxstyleliteralstrong{\sphinxupquote{btn}} (\sphinxstyleliteralemphasis{\sphinxupquote{HTML\_button\_element}}) \textendash{} 

\end{itemize}

\end{description}\end{quote}

\end{fulllineitems}

\index{openAreaPopup() (built\sphinxhyphen{}in function)@\spxentry{openAreaPopup()}\spxextra{built\sphinxhyphen{}in function}}

\begin{fulllineitems}
\phantomsection\label{\detokenize{docs_gui/js_api/postcode_editor/display_postcode:openAreaPopup}}
\pysigstartsignatures
\pysiglinewithargsret{\sphinxbfcode{\sphinxupquote{\DUrole{n}{openAreaPopup}}}}{}{}
\pysigstopsignatures
\sphinxAtStartPar
onclick function for the Generate Network button
makes the area version input GUI visible and adds a listener to it that makes sure the form can only be submitted if the inputs are correct

\end{fulllineitems}

\index{returnSelectedBuildings() (built\sphinxhyphen{}in function)@\spxentry{returnSelectedBuildings()}\spxextra{built\sphinxhyphen{}in function}}

\begin{fulllineitems}
\phantomsection\label{\detokenize{docs_gui/js_api/postcode_editor/display_postcode:returnSelectedBuildings}}
\pysigstartsignatures
\pysiglinewithargsret{\sphinxbfcode{\sphinxupquote{\DUrole{n}{returnSelectedBuildings}}}}{}{}
\pysigstopsignatures
\sphinxAtStartPar
onclick function for the Generate Network button within the area version input GUI
at the moment it only returns the new id and gives an error warning, if the selected version already exists for a given ID
It also closes the GUI form

\end{fulllineitems}

\index{displayPreviewNet() (built\sphinxhyphen{}in function)@\spxentry{displayPreviewNet()}\spxextra{built\sphinxhyphen{}in function}}

\begin{fulllineitems}
\phantomsection\label{\detokenize{docs_gui/js_api/postcode_editor/display_postcode:displayPreviewNet}}
\pysigstartsignatures
\pysiglinewithargsret{\sphinxbfcode{\sphinxupquote{\DUrole{n}{displayPreviewNet}}}}{\emph{\DUrole{n}{kcid}}, \emph{\DUrole{n}{bcid}}, \emph{\DUrole{n}{line\_geoJSON}}}{}
\pysigstopsignatures
\sphinxAtStartPar
adds all lines of a network to a new layer and displays it on the map
further defines several inner functions to handle click, mouseover and mouseout functionality and attaches them to the layer
We only display the lines of all networks for performance reasons, showing buses adds too many nodes.
\begin{quote}\begin{description}
\sphinxlineitem{Arguments}\begin{itemize}
\item {} 
\sphinxAtStartPar
\sphinxstyleliteralstrong{\sphinxupquote{kcid}} (\sphinxstyleliteralemphasis{\sphinxupquote{int}}) \textendash{} the k cluster id of a network

\item {} 
\sphinxAtStartPar
\sphinxstyleliteralstrong{\sphinxupquote{bcid}} (\sphinxstyleliteralemphasis{\sphinxupquote{int}}) \textendash{} the building cluster id of a network

\item {} 
\sphinxAtStartPar
\sphinxstyleliteralstrong{\sphinxupquote{line\_geoJSON}} (\sphinxstyleliteralemphasis{\sphinxupquote{geoJSON\_dict}}) \textendash{} a dict containing all the lines of a network

\end{itemize}

\end{description}\end{quote}

\end{fulllineitems}

\index{styleWhenMouseOver() (built\sphinxhyphen{}in function)@\spxentry{styleWhenMouseOver()}\spxextra{built\sphinxhyphen{}in function}}

\begin{fulllineitems}
\phantomsection\label{\detokenize{docs_gui/js_api/postcode_editor/display_postcode:styleWhenMouseOver}}
\pysigstartsignatures
\pysiglinewithargsret{\sphinxbfcode{\sphinxupquote{\DUrole{n}{styleWhenMouseOver}}}}{}{}
\pysigstopsignatures
\sphinxAtStartPar
changes color of network in GUI when the mouse hovers above it

\end{fulllineitems}

\index{styleWhenMouseOut() (built\sphinxhyphen{}in function)@\spxentry{styleWhenMouseOut()}\spxextra{built\sphinxhyphen{}in function}}

\begin{fulllineitems}
\phantomsection\label{\detokenize{docs_gui/js_api/postcode_editor/display_postcode:styleWhenMouseOut}}
\pysigstartsignatures
\pysiglinewithargsret{\sphinxbfcode{\sphinxupquote{\DUrole{n}{styleWhenMouseOut}}}}{}{}
\pysigstopsignatures
\sphinxAtStartPar
changes color of network in GUI back of default once the mouse no longer hovers above it

\end{fulllineitems}

\index{populateNetList() (built\sphinxhyphen{}in function)@\spxentry{populateNetList()}\spxextra{built\sphinxhyphen{}in function}}

\begin{fulllineitems}
\phantomsection\label{\detokenize{docs_gui/js_api/postcode_editor/display_postcode:populateNetList}}
\pysigstartsignatures
\pysiglinewithargsret{\sphinxbfcode{\sphinxupquote{\DUrole{n}{populateNetList}}}}{\emph{\DUrole{n}{listName}}, \emph{\DUrole{n}{list}}}{}
\pysigstopsignatures
\sphinxAtStartPar
creates options for each network we have gotten from the backend
\begin{quote}\begin{description}
\sphinxlineitem{Arguments}\begin{itemize}
\item {} 
\sphinxAtStartPar
\sphinxstyleliteralstrong{\sphinxupquote{listName}} (\sphinxstyleliteralemphasis{\sphinxupquote{string}}) \textendash{} key for the html select element we want to attach options to

\item {} 
\sphinxAtStartPar
\sphinxstyleliteralstrong{\sphinxupquote{list}} (\sphinxstyleliteralemphasis{\sphinxupquote{list}}) \textendash{} list containing the objects we want to create select options for

\end{itemize}

\end{description}\end{quote}

\end{fulllineitems}

\index{highlightSelectedPreviewLayer() (built\sphinxhyphen{}in function)@\spxentry{highlightSelectedPreviewLayer()}\spxextra{built\sphinxhyphen{}in function}}

\begin{fulllineitems}
\phantomsection\label{\detokenize{docs_gui/js_api/postcode_editor/display_postcode:highlightSelectedPreviewLayer}}
\pysigstartsignatures
\pysiglinewithargsret{\sphinxbfcode{\sphinxupquote{\DUrole{n}{highlightSelectedPreviewLayer}}}}{\emph{\DUrole{n}{sel}}}{}
\pysigstopsignatures
\sphinxAtStartPar
onclick function for the network select html element
makes sure a network selected in the html select element is highlighted on the map by manually triggering a click event for it
\begin{quote}\begin{description}
\sphinxlineitem{Arguments}\begin{itemize}
\item {} 
\sphinxAtStartPar
\sphinxstyleliteralstrong{\sphinxupquote{sel}} (\sphinxstyleliteralemphasis{\sphinxupquote{HTML\_select\_element}}) \textendash{} reference to the network select html element

\end{itemize}

\end{description}\end{quote}

\end{fulllineitems}

\index{sendBackSelectedNetworkKcidBcid() (built\sphinxhyphen{}in function)@\spxentry{sendBackSelectedNetworkKcidBcid()}\spxextra{built\sphinxhyphen{}in function}}

\begin{fulllineitems}
\phantomsection\label{\detokenize{docs_gui/js_api/postcode_editor/display_postcode:sendBackSelectedNetworkKcidBcid}}
\pysigstartsignatures
\pysiglinewithargsret{\sphinxbfcode{\sphinxupquote{\DUrole{n}{sendBackSelectedNetworkKcidBcid}}}}{}{}
\pysigstopsignatures
\sphinxAtStartPar
extracts kcid and bcid from the feature properties of the selected element in the netlist and sends it to the backend

\end{fulllineitems}

\index{highlightBuildingFeature() (built\sphinxhyphen{}in function)@\spxentry{highlightBuildingFeature()}\spxextra{built\sphinxhyphen{}in function}}

\begin{fulllineitems}
\phantomsection\label{\detokenize{docs_gui/js_api/postcode_editor/display_postcode:highlightBuildingFeature}}
\pysigstartsignatures
\pysiglinewithargsret{\sphinxbfcode{\sphinxupquote{\DUrole{n}{highlightBuildingFeature}}}}{\emph{\DUrole{n}{e}}}{}
\pysigstopsignatures
\sphinxAtStartPar
mouseover function for buildings on the map
\begin{quote}\begin{description}
\sphinxlineitem{Arguments}\begin{itemize}
\item {} 
\sphinxAtStartPar
\sphinxstyleliteralstrong{\sphinxupquote{e}} (\sphinxstyleliteralemphasis{\sphinxupquote{event}}) \textendash{} The event that gets triggered on mouseover of an object on the map

\end{itemize}

\end{description}\end{quote}

\end{fulllineitems}

\index{resetBuildingHighlight() (built\sphinxhyphen{}in function)@\spxentry{resetBuildingHighlight()}\spxextra{built\sphinxhyphen{}in function}}

\begin{fulllineitems}
\phantomsection\label{\detokenize{docs_gui/js_api/postcode_editor/display_postcode:resetBuildingHighlight}}
\pysigstartsignatures
\pysiglinewithargsret{\sphinxbfcode{\sphinxupquote{\DUrole{n}{resetBuildingHighlight}}}}{\emph{\DUrole{n}{e}}}{}
\pysigstopsignatures
\sphinxAtStartPar
mouseout function for buildings on the map
\begin{quote}\begin{description}
\sphinxlineitem{Arguments}\begin{itemize}
\item {} 
\sphinxAtStartPar
\sphinxstyleliteralstrong{\sphinxupquote{e}} (\sphinxstyleliteralemphasis{\sphinxupquote{event}}) \textendash{} The event that gets triggered on mouseout of an object on the map

\end{itemize}

\end{description}\end{quote}

\end{fulllineitems}

\index{zoomToBuildingFeature() (built\sphinxhyphen{}in function)@\spxentry{zoomToBuildingFeature()}\spxextra{built\sphinxhyphen{}in function}}

\begin{fulllineitems}
\phantomsection\label{\detokenize{docs_gui/js_api/postcode_editor/display_postcode:zoomToBuildingFeature}}
\pysigstartsignatures
\pysiglinewithargsret{\sphinxbfcode{\sphinxupquote{\DUrole{n}{zoomToBuildingFeature}}}}{\emph{\DUrole{n}{e}}}{}
\pysigstopsignatures
\sphinxAtStartPar
onclick function for buildings on the map
\begin{quote}\begin{description}
\sphinxlineitem{Arguments}\begin{itemize}
\item {} 
\sphinxAtStartPar
\sphinxstyleliteralstrong{\sphinxupquote{e}} (\sphinxstyleliteralemphasis{\sphinxupquote{event}}) \textendash{} The event that gets triggered on click on an object on the map

\end{itemize}

\end{description}\end{quote}

\end{fulllineitems}

\index{displayBuildingEditOptions() (built\sphinxhyphen{}in function)@\spxentry{displayBuildingEditOptions()}\spxextra{built\sphinxhyphen{}in function}}

\begin{fulllineitems}
\phantomsection\label{\detokenize{docs_gui/js_api/postcode_editor/display_postcode:displayBuildingEditOptions}}
\pysigstartsignatures
\pysiglinewithargsret{\sphinxbfcode{\sphinxupquote{\DUrole{n}{displayBuildingEditOptions}}}}{\emph{\DUrole{n}{e}}}{}
\pysigstopsignatures
\sphinxAtStartPar
aggregate onclick function for buildings on the map in case click events should have multiple effects
\begin{quote}\begin{description}
\sphinxlineitem{Arguments}\begin{itemize}
\item {} 
\sphinxAtStartPar
\sphinxstyleliteralstrong{\sphinxupquote{e}} (\sphinxstyleliteralemphasis{\sphinxupquote{event}}) \textendash{} The event that gets triggered on click on an object on the map

\end{itemize}

\end{description}\end{quote}

\end{fulllineitems}

\index{onEachFeature() (built\sphinxhyphen{}in function)@\spxentry{onEachFeature()}\spxextra{built\sphinxhyphen{}in function}}

\begin{fulllineitems}
\phantomsection\label{\detokenize{docs_gui/js_api/postcode_editor/display_postcode:onEachFeature}}
\pysigstartsignatures
\pysiglinewithargsret{\sphinxbfcode{\sphinxupquote{\DUrole{n}{onEachFeature}}}}{\emph{\DUrole{n}{feature}}, \emph{\DUrole{n}{layer}}}{}
\pysigstopsignatures
\sphinxAtStartPar
is called when buildings are displayed on the map and defines different behaviours for each object on the map
\begin{quote}\begin{description}
\sphinxlineitem{Arguments}\begin{itemize}
\item {} 
\sphinxAtStartPar
\sphinxstyleliteralstrong{\sphinxupquote{feature}} (\sphinxstyleliteralemphasis{\sphinxupquote{dict}}) \textendash{} 

\item {} 
\sphinxAtStartPar
\sphinxstyleliteralstrong{\sphinxupquote{layer}} (\sphinxstyleliteralemphasis{\sphinxupquote{leaflet\_layer\_object}}) \textendash{} 

\end{itemize}

\end{description}\end{quote}

\end{fulllineitems}

\index{createBuildingPopup() (built\sphinxhyphen{}in function)@\spxentry{createBuildingPopup()}\spxextra{built\sphinxhyphen{}in function}}

\begin{fulllineitems}
\phantomsection\label{\detokenize{docs_gui/js_api/postcode_editor/display_postcode:createBuildingPopup}}
\pysigstartsignatures
\pysiglinewithargsret{\sphinxbfcode{\sphinxupquote{\DUrole{n}{createBuildingPopup}}}}{\emph{\DUrole{n}{feature}}, \emph{\DUrole{n}{layer}}}{}
\pysigstopsignatures
\sphinxAtStartPar
is called for each building when they are placed on the map and attaches a popup to it, containing a button that allows the user to delete that building
the popup could contain other information as well
\begin{quote}\begin{description}
\sphinxlineitem{Arguments}\begin{itemize}
\item {} 
\sphinxAtStartPar
\sphinxstyleliteralstrong{\sphinxupquote{feature}} (\sphinxstyleliteralemphasis{\sphinxupquote{dict}}) \textendash{} 

\item {} 
\sphinxAtStartPar
\sphinxstyleliteralstrong{\sphinxupquote{layer}} (\sphinxstyleliteralemphasis{\sphinxupquote{leaflet\_layer\_object}}) \textendash{} 

\end{itemize}

\end{description}\end{quote}

\end{fulllineitems}


\sphinxstepscope


\subparagraph{Network Editor}
\label{\detokenize{docs_gui/js_api/network_editor/index:network-editor}}\label{\detokenize{docs_gui/js_api/network_editor/index::doc}}
\sphinxstepscope


\subparagraph{generate\_editable\_network.js}
\label{\detokenize{docs_gui/js_api/network_editor/generate_editable_network:generate-editable-network-js}}\label{\detokenize{docs_gui/js_api/network_editor/generate_editable_network::doc}}\index{GetPandapowerAndWriteGeoJSONNet() (built\sphinxhyphen{}in function)@\spxentry{GetPandapowerAndWriteGeoJSONNet()}\spxextra{built\sphinxhyphen{}in function}}

\begin{fulllineitems}
\phantomsection\label{\detokenize{docs_gui/js_api/network_editor/generate_editable_network:GetPandapowerAndWriteGeoJSONNet}}
\pysigstartsignatures
\pysiglinewithargsret{\sphinxbfcode{\sphinxupquote{\DUrole{n}{GetPandapowerAndWriteGeoJSONNet}}}}{}{}
\pysigstopsignatures
\sphinxAtStartPar
gets called when the window is first loaded, retrieves preprocessed Geojson data from the backend
and calls necessary functions to display network on the map and fill editor windows

\end{fulllineitems}

\index{changeTrafoBusNames() (built\sphinxhyphen{}in function)@\spxentry{changeTrafoBusNames()}\spxextra{built\sphinxhyphen{}in function}}

\begin{fulllineitems}
\phantomsection\label{\detokenize{docs_gui/js_api/network_editor/generate_editable_network:changeTrafoBusNames}}
\pysigstartsignatures
\pysiglinewithargsret{\sphinxbfcode{\sphinxupquote{\DUrole{n}{changeTrafoBusNames}}}}{\emph{\DUrole{n}{busList}}, \emph{\DUrole{n}{trafoList}}}{}
\pysigstopsignatures
\sphinxAtStartPar
The buses connected to the trafo need to have specific names to work with urbs later so we change them here
\begin{quote}\begin{description}
\sphinxlineitem{Arguments}\begin{itemize}
\item {} 
\sphinxAtStartPar
\sphinxstyleliteralstrong{\sphinxupquote{busList}} (\sphinxstyleliteralemphasis{\sphinxupquote{list}}) \textendash{} list of all buses

\item {} 
\sphinxAtStartPar
\sphinxstyleliteralstrong{\sphinxupquote{trafoList}} (\sphinxstyleliteralemphasis{\sphinxupquote{list}}) \textendash{} list of all trafos

\end{itemize}

\end{description}\end{quote}

\end{fulllineitems}

\index{extractStdTypes() (built\sphinxhyphen{}in function)@\spxentry{extractStdTypes()}\spxextra{built\sphinxhyphen{}in function}}

\begin{fulllineitems}
\phantomsection\label{\detokenize{docs_gui/js_api/network_editor/generate_editable_network:extractStdTypes}}
\pysigstartsignatures
\pysiglinewithargsret{\sphinxbfcode{\sphinxupquote{\DUrole{n}{extractStdTypes}}}}{\emph{\DUrole{n}{ppdata}}}{}
\pysigstopsignatures
\sphinxAtStartPar
small aggregate function for all std types
\begin{quote}\begin{description}
\sphinxlineitem{Arguments}\begin{itemize}
\item {} 
\sphinxAtStartPar
\sphinxstyleliteralstrong{\sphinxupquote{ppdata}} (\sphinxstyleliteralemphasis{\sphinxupquote{dict}}) \textendash{} dict retrieved from the backend

\end{itemize}

\end{description}\end{quote}

\end{fulllineitems}

\index{displayNetNew() (built\sphinxhyphen{}in function)@\spxentry{displayNetNew()}\spxextra{built\sphinxhyphen{}in function}}

\begin{fulllineitems}
\phantomsection\label{\detokenize{docs_gui/js_api/network_editor/generate_editable_network:displayNetNew}}
\pysigstartsignatures
\pysiglinewithargsret{\sphinxbfcode{\sphinxupquote{\DUrole{n}{displayNetNew}}}}{\emph{\DUrole{n}{ppdata}}}{}
\pysigstopsignatures
\sphinxAtStartPar
aggregate function calling the actual functions that place the feature geojsons on the leaflet map
\begin{quote}\begin{description}
\sphinxlineitem{Arguments}\begin{itemize}
\item {} 
\sphinxAtStartPar
\sphinxstyleliteralstrong{\sphinxupquote{ppdata}} (\sphinxstyleliteralemphasis{\sphinxupquote{dict}}) \textendash{} dict retrieved from the backend

\end{itemize}

\end{description}\end{quote}

\end{fulllineitems}

\index{addGeoJSONtoMap() (built\sphinxhyphen{}in function)@\spxentry{addGeoJSONtoMap()}\spxextra{built\sphinxhyphen{}in function}}

\begin{fulllineitems}
\phantomsection\label{\detokenize{docs_gui/js_api/network_editor/generate_editable_network:addGeoJSONtoMap}}
\pysigstartsignatures
\pysiglinewithargsret{\sphinxbfcode{\sphinxupquote{\DUrole{n}{addGeoJSONtoMap}}}}{\emph{\DUrole{n}{isLines}}, \emph{\DUrole{n}{input\_geoJSON}}, \emph{\DUrole{n}{featureName}}}{}
\pysigstopsignatures
\sphinxAtStartPar
function that adds a FeatureCollection to the leaflet map
we set styles and onclick functions here and save references to each added feature in the NetworkObject
lines (lines, trafos) and circlemarkers (buses, ext\_grids) need to be handled differently because lines do not have the pointToLayer function
\begin{quote}\begin{description}
\sphinxlineitem{Arguments}\begin{itemize}
\item {} 
\sphinxAtStartPar
\sphinxstyleliteralstrong{\sphinxupquote{isLines}} (\sphinxstyleliteralemphasis{\sphinxupquote{boolean}}) \textendash{} distinguishes if the geometry we want to place on the map is a point or a line

\item {} 
\sphinxAtStartPar
\sphinxstyleliteralstrong{\sphinxupquote{input\_geoJSON}} (\sphinxstyleliteralemphasis{\sphinxupquote{geoJSON\_FeatureCollection}}) \textendash{} the data structure containing all our grid information

\item {} 
\sphinxAtStartPar
\sphinxstyleliteralstrong{\sphinxupquote{featureName}} (\sphinxstyleliteralemphasis{\sphinxupquote{string}}) \textendash{} name of the feature type to be added to the map (e.g. line, bus etc)

\end{itemize}

\end{description}\end{quote}

\end{fulllineitems}


\sphinxstepscope


\subparagraph{display\_editable\_network\_features.js}
\label{\detokenize{docs_gui/js_api/network_editor/display_editable_network_features:display-editable-network-features-js}}\label{\detokenize{docs_gui/js_api/network_editor/display_editable_network_features::doc}}\index{fillStdTypeList() (built\sphinxhyphen{}in function)@\spxentry{fillStdTypeList()}\spxextra{built\sphinxhyphen{}in function}}

\begin{fulllineitems}
\phantomsection\label{\detokenize{docs_gui/js_api/network_editor/display_editable_network_features:fillStdTypeList}}
\pysigstartsignatures
\pysiglinewithargsret{\sphinxbfcode{\sphinxupquote{\DUrole{n}{fillStdTypeList}}}}{}{}
\pysigstopsignatures
\sphinxAtStartPar
function creates options for each std\_type select element in the GUI

\end{fulllineitems}

\index{fillStdTypeEditor() (built\sphinxhyphen{}in function)@\spxentry{fillStdTypeEditor()}\spxextra{built\sphinxhyphen{}in function}}

\begin{fulllineitems}
\phantomsection\label{\detokenize{docs_gui/js_api/network_editor/display_editable_network_features:fillStdTypeEditor}}
\pysigstartsignatures
\pysiglinewithargsret{\sphinxbfcode{\sphinxupquote{\DUrole{n}{fillStdTypeEditor}}}}{\emph{\DUrole{n}{sel}}, \emph{\DUrole{n}{listName}}}{}
\pysigstopsignatures
\sphinxAtStartPar
displays the right std\_types editor window and fills the input fields with the right values
\begin{quote}\begin{description}
\sphinxlineitem{Arguments}\begin{itemize}
\item {} 
\sphinxAtStartPar
\sphinxstyleliteralstrong{\sphinxupquote{sel}} (\sphinxstyleliteralemphasis{\sphinxupquote{HTML\_select\_element}}) \textendash{} HTML select element we want to fill

\item {} 
\sphinxAtStartPar
\sphinxstyleliteralstrong{\sphinxupquote{listName}} (\sphinxstyleliteralemphasis{\sphinxupquote{string}}) \textendash{} key of the std\_type group

\end{itemize}

\end{description}\end{quote}

\end{fulllineitems}

\index{populateLists() (built\sphinxhyphen{}in function)@\spxentry{populateLists()}\spxextra{built\sphinxhyphen{}in function}}

\begin{fulllineitems}
\phantomsection\label{\detokenize{docs_gui/js_api/network_editor/display_editable_network_features:populateLists}}
\pysigstartsignatures
\pysiglinewithargsret{\sphinxbfcode{\sphinxupquote{\DUrole{n}{populateLists}}}}{\emph{\DUrole{n}{listName}}}{}
\pysigstopsignatures
\sphinxAtStartPar
fills html element with options for a given list of network features at intial editable network generation
the index property of a feature and the option index do not have to match
\begin{quote}\begin{description}
\sphinxlineitem{Arguments}\begin{itemize}
\item {} 
\sphinxAtStartPar
\sphinxstyleliteralstrong{\sphinxupquote{listName}} (\sphinxstyleliteralemphasis{\sphinxupquote{string}}) \textendash{} key for accessing and setting html element ids

\end{itemize}

\end{description}\end{quote}

\end{fulllineitems}

\index{populateEditableNetworkEditor() (built\sphinxhyphen{}in function)@\spxentry{populateEditableNetworkEditor()}\spxextra{built\sphinxhyphen{}in function}}

\begin{fulllineitems}
\phantomsection\label{\detokenize{docs_gui/js_api/network_editor/display_editable_network_features:populateEditableNetworkEditor}}
\pysigstartsignatures
\pysiglinewithargsret{\sphinxbfcode{\sphinxupquote{\DUrole{n}{populateEditableNetworkEditor}}}}{\emph{\DUrole{n}{listName}}, \emph{\DUrole{n}{selectedProperties}}, \emph{\DUrole{n}{std\_typeList}}, \emph{\DUrole{n}{std\_type\_properties}}}{}
\pysigstopsignatures
\sphinxAtStartPar
The feature editor window template for all feature types gets filled at runtime
input fields and labels depend entirely on the properties defined in the displayNetwork function
\begin{quote}\begin{description}
\sphinxlineitem{Arguments}\begin{itemize}
\item {} 
\sphinxAtStartPar
\sphinxstyleliteralstrong{\sphinxupquote{listName}} (\sphinxstyleliteralemphasis{\sphinxupquote{string}}) \textendash{} key for accessing and setting html element ids

\item {} 
\sphinxAtStartPar
\sphinxstyleliteralstrong{\sphinxupquote{selectedProperties}} (\sphinxstyleliteralemphasis{\sphinxupquote{dict}}) \textendash{} dict containing all properties of the feature for which we create the editor window

\item {} 
\sphinxAtStartPar
\sphinxstyleliteralstrong{\sphinxupquote{std\_typeList}} (\sphinxstyleliteralemphasis{\sphinxupquote{dict}}) \textendash{} dict containing all std\_types, only needed for keys here

\item {} 
\sphinxAtStartPar
\sphinxstyleliteralstrong{\sphinxupquote{std\_type\_properties}} (\sphinxstyleliteralemphasis{\sphinxupquote{dict}}) \textendash{} dict containing all properties of the std\_type associated with the feature

\end{itemize}

\end{description}\end{quote}

\end{fulllineitems}

\index{populateEditableNetworkEditorSecondaryFeature() (built\sphinxhyphen{}in function)@\spxentry{populateEditableNetworkEditorSecondaryFeature()}\spxextra{built\sphinxhyphen{}in function}}

\begin{fulllineitems}
\phantomsection\label{\detokenize{docs_gui/js_api/network_editor/display_editable_network_features:populateEditableNetworkEditorSecondaryFeature}}
\pysigstartsignatures
\pysiglinewithargsret{\sphinxbfcode{\sphinxupquote{\DUrole{n}{populateEditableNetworkEditorSecondaryFeature}}}}{\emph{\DUrole{n}{primaryFeatureName}}, \emph{\DUrole{n}{secondaryFeatureName}}}{}
\pysigstopsignatures
\sphinxAtStartPar
creates html elements for the secondary feature editor and adds select options for the secondary features
\begin{quote}\begin{description}
\sphinxlineitem{Arguments}\begin{itemize}
\item {} 
\sphinxAtStartPar
\sphinxstyleliteralstrong{\sphinxupquote{primaryFeatureName}} (\sphinxstyleliteralemphasis{\sphinxupquote{string}}) \textendash{} name of the primary feature (e.g. bus)

\item {} 
\sphinxAtStartPar
\sphinxstyleliteralstrong{\sphinxupquote{secondaryFeatureName}} (\sphinxstyleliteralemphasis{\sphinxupquote{string}}) \textendash{} name of the secondary feature (e.g. load)

\end{itemize}

\end{description}\end{quote}

\end{fulllineitems}

\index{addSecondaryFeature() (built\sphinxhyphen{}in function)@\spxentry{addSecondaryFeature()}\spxextra{built\sphinxhyphen{}in function}}

\begin{fulllineitems}
\phantomsection\label{\detokenize{docs_gui/js_api/network_editor/display_editable_network_features:addSecondaryFeature}}
\pysigstartsignatures
\pysiglinewithargsret{\sphinxbfcode{\sphinxupquote{\DUrole{n}{addSecondaryFeature}}}}{\emph{\DUrole{n}{primaryFeatureName}}, \emph{\DUrole{n}{secondaryFeatureName}}}{}
\pysigstopsignatures
\sphinxAtStartPar
onclick function for the Add Secondary Feature buttons. Adds a new secondary feature to the GUI and the network object
\begin{quote}\begin{description}
\sphinxlineitem{Arguments}\begin{itemize}
\item {} 
\sphinxAtStartPar
\sphinxstyleliteralstrong{\sphinxupquote{primaryFeatureName}} (\sphinxstyleliteralemphasis{\sphinxupquote{string}}) \textendash{} 

\item {} 
\sphinxAtStartPar
\sphinxstyleliteralstrong{\sphinxupquote{secondaryFeatureName}} (\sphinxstyleliteralemphasis{\sphinxupquote{string}}) \textendash{} 

\end{itemize}

\end{description}\end{quote}

\end{fulllineitems}

\index{openEditableNetworkList() (built\sphinxhyphen{}in function)@\spxentry{openEditableNetworkList()}\spxextra{built\sphinxhyphen{}in function}}

\begin{fulllineitems}
\phantomsection\label{\detokenize{docs_gui/js_api/network_editor/display_editable_network_features:openEditableNetworkList}}
\pysigstartsignatures
\pysiglinewithargsret{\sphinxbfcode{\sphinxupquote{\DUrole{n}{openEditableNetworkList}}}}{\emph{\DUrole{n}{e}}, \emph{\DUrole{n}{listName}}}{}
\pysigstopsignatures
\sphinxAtStartPar
gets called when one of the tablink buttons in the GUI gets pressed and opens the relevant feature list, while hiding all other GUI elements
\begin{quote}\begin{description}
\sphinxlineitem{Arguments}\begin{itemize}
\item {} 
\sphinxAtStartPar
\sphinxstyleliteralstrong{\sphinxupquote{e}} (\sphinxstyleliteralemphasis{\sphinxupquote{event}}) \textendash{} onclick event which triggered the function call

\item {} 
\sphinxAtStartPar
\sphinxstyleliteralstrong{\sphinxupquote{listName}} (\sphinxstyleliteralemphasis{\sphinxupquote{string}}) \textendash{} name of the list to be opened

\end{itemize}

\end{description}\end{quote}

\end{fulllineitems}

\index{fillSelectedEditableNetworkFeatureEditor() (built\sphinxhyphen{}in function)@\spxentry{fillSelectedEditableNetworkFeatureEditor()}\spxextra{built\sphinxhyphen{}in function}}

\begin{fulllineitems}
\phantomsection\label{\detokenize{docs_gui/js_api/network_editor/display_editable_network_features:fillSelectedEditableNetworkFeatureEditor}}
\pysigstartsignatures
\pysiglinewithargsret{\sphinxbfcode{\sphinxupquote{\DUrole{n}{fillSelectedEditableNetworkFeatureEditor}}}}{\emph{\DUrole{n}{sel}}, \emph{\DUrole{n}{listName}}}{}
\pysigstopsignatures
\sphinxAtStartPar
writes values of the currently selected feature into the input fields of the editor window
\begin{quote}\begin{description}
\sphinxlineitem{Arguments}\begin{itemize}
\item {} 
\sphinxAtStartPar
\sphinxstyleliteralstrong{\sphinxupquote{sel}} (\sphinxstyleliteralemphasis{\sphinxupquote{HTML\_select\_element}}) \textendash{} the html select element whose onchange event triggered the function call

\item {} 
\sphinxAtStartPar
\sphinxstyleliteralstrong{\sphinxupquote{listName}} (\sphinxstyleliteralemphasis{\sphinxupquote{string}}) \textendash{} key for accessing data to be written into the editor windows

\end{itemize}

\end{description}\end{quote}

\end{fulllineitems}

\index{resetStyle() (built\sphinxhyphen{}in function)@\spxentry{resetStyle()}\spxextra{built\sphinxhyphen{}in function}}

\begin{fulllineitems}
\phantomsection\label{\detokenize{docs_gui/js_api/network_editor/display_editable_network_features:resetStyle}}
\pysigstartsignatures
\pysiglinewithargsret{\sphinxbfcode{\sphinxupquote{\DUrole{n}{resetStyle}}}}{\emph{\DUrole{n}{target}}, \emph{\DUrole{n}{feature}}}{}
\pysigstopsignatures
\sphinxAtStartPar
resets the styling of the previously selected feature and sets the new styling of the now selected feature
\begin{quote}\begin{description}
\sphinxlineitem{Arguments}\begin{itemize}
\item {} 
\sphinxAtStartPar
\sphinxstyleliteralstrong{\sphinxupquote{target}} (\sphinxstyleliteralemphasis{\sphinxupquote{event\_target\_object}}) \textendash{} leaflet map object whose style is to be changed

\item {} 
\sphinxAtStartPar
\sphinxstyleliteralstrong{\sphinxupquote{feature}} (\sphinxstyleliteralemphasis{\sphinxupquote{string}}) \textendash{} name of the target type (e.g. bus, line)

\end{itemize}

\end{description}\end{quote}

\end{fulllineitems}

\index{clickOnMarker() (built\sphinxhyphen{}in function)@\spxentry{clickOnMarker()}\spxextra{built\sphinxhyphen{}in function}}

\begin{fulllineitems}
\phantomsection\label{\detokenize{docs_gui/js_api/network_editor/display_editable_network_features:clickOnMarker}}
\pysigstartsignatures
\pysiglinewithargsret{\sphinxbfcode{\sphinxupquote{\DUrole{n}{clickOnMarker}}}}{\emph{\DUrole{n}{target}}, \emph{\DUrole{n}{feature}}, \emph{\DUrole{n}{drawModeOverride}}}{}
\pysigstopsignatures\begin{description}
\sphinxlineitem{When clicking on a map element or making a selection from a list,}
\sphinxAtStartPar
we highlight the relevant element, open the Editor window and fill its input fields with the relevant values

\end{description}
\begin{quote}\begin{description}
\sphinxlineitem{Arguments}\begin{itemize}
\item {} 
\sphinxAtStartPar
\sphinxstyleliteralstrong{\sphinxupquote{target}} (\sphinxstyleliteralemphasis{\sphinxupquote{event\_target\_object}}) \textendash{} leaflet map objects

\item {} 
\sphinxAtStartPar
\sphinxstyleliteralstrong{\sphinxupquote{feature}} (\sphinxstyleliteralemphasis{\sphinxupquote{string}}) \textendash{} name of the feature that was selected

\item {} 
\sphinxAtStartPar
\sphinxstyleliteralstrong{\sphinxupquote{drawModeOverride}} (\sphinxstyleliteralemphasis{\sphinxupquote{bool}}) \textendash{} if this is false, onclick events are not registered on map features while creating new features

\end{itemize}

\end{description}\end{quote}

\end{fulllineitems}

\index{openSecondaryEditor() (built\sphinxhyphen{}in function)@\spxentry{openSecondaryEditor()}\spxextra{built\sphinxhyphen{}in function}}

\begin{fulllineitems}
\phantomsection\label{\detokenize{docs_gui/js_api/network_editor/display_editable_network_features:openSecondaryEditor}}
\pysigstartsignatures
\pysiglinewithargsret{\sphinxbfcode{\sphinxupquote{\DUrole{n}{openSecondaryEditor}}}}{\emph{\DUrole{n}{sel}}, \emph{\DUrole{n}{secondaryFeatureName}}}{}
\pysigstopsignatures
\sphinxAtStartPar
onclick function for the Secondary Feature buttons in the editor window, opens the secondary feature editor window
\begin{quote}\begin{description}
\sphinxlineitem{Arguments}\begin{itemize}
\item {} 
\sphinxAtStartPar
\sphinxstyleliteralstrong{\sphinxupquote{sel}} (\sphinxstyleliteralemphasis{\sphinxupquote{HTML\_select\_element}}) \textendash{} the html select element whose onchange event triggered the function calls

\item {} 
\sphinxAtStartPar
\sphinxstyleliteralstrong{\sphinxupquote{secondaryFeatureName}} (\sphinxstyleliteralemphasis{\sphinxupquote{string}}) \textendash{} key for accessing he secondary feature data

\end{itemize}

\end{description}\end{quote}

\end{fulllineitems}

\index{writeBackEditedNetworkFeature() (built\sphinxhyphen{}in function)@\spxentry{writeBackEditedNetworkFeature()}\spxextra{built\sphinxhyphen{}in function}}

\begin{fulllineitems}
\phantomsection\label{\detokenize{docs_gui/js_api/network_editor/display_editable_network_features:writeBackEditedNetworkFeature}}
\pysigstartsignatures
\pysiglinewithargsret{\sphinxbfcode{\sphinxupquote{\DUrole{n}{writeBackEditedNetworkFeature}}}}{\emph{\DUrole{n}{target}}, \emph{\DUrole{n}{targetDiv}}}{}
\pysigstopsignatures
\sphinxAtStartPar
onchange function for editor view. If a field is changed, its new value is written back to the relevant object
\begin{quote}\begin{description}
\sphinxlineitem{Arguments}\begin{itemize}
\item {} 
\sphinxAtStartPar
\sphinxstyleliteralstrong{\sphinxupquote{target}} (\sphinxstyleliteralemphasis{\sphinxupquote{event\_target\_object}}) \textendash{} the editor element that was changed

\item {} 
\sphinxAtStartPar
\sphinxstyleliteralstrong{\sphinxupquote{targetDiv}} (\sphinxstyleliteralemphasis{\sphinxupquote{string}}) \textendash{} the div in which the changed editor element resides

\end{itemize}

\end{description}\end{quote}

\end{fulllineitems}

\index{updateStdTypeFeaturesInEditor() (built\sphinxhyphen{}in function)@\spxentry{updateStdTypeFeaturesInEditor()}\spxextra{built\sphinxhyphen{}in function}}

\begin{fulllineitems}
\phantomsection\label{\detokenize{docs_gui/js_api/network_editor/display_editable_network_features:updateStdTypeFeaturesInEditor}}
\pysigstartsignatures
\pysiglinewithargsret{\sphinxbfcode{\sphinxupquote{\DUrole{n}{updateStdTypeFeaturesInEditor}}}}{\emph{\DUrole{n}{id}}, \emph{\DUrole{n}{feature}}}{}
\pysigstopsignatures
\sphinxAtStartPar
if a new std\_type is selected in an editor window of another feature, this function updates all input fields associated with the std\_type
\begin{quote}\begin{description}
\sphinxlineitem{Arguments}\begin{itemize}
\item {} 
\sphinxAtStartPar
\sphinxstyleliteralstrong{\sphinxupquote{id}} (\sphinxstyleliteralemphasis{\sphinxupquote{string}}) \textendash{} name of the std\_type

\item {} 
\sphinxAtStartPar
\sphinxstyleliteralstrong{\sphinxupquote{feature}} (\sphinxstyleliteralemphasis{\sphinxupquote{string}}) \textendash{} name of the feature for which the std\_type was changed

\end{itemize}

\end{description}\end{quote}

\end{fulllineitems}

\index{createPopup() (built\sphinxhyphen{}in function)@\spxentry{createPopup()}\spxextra{built\sphinxhyphen{}in function}}

\begin{fulllineitems}
\phantomsection\label{\detokenize{docs_gui/js_api/network_editor/display_editable_network_features:createPopup}}
\pysigstartsignatures
\pysiglinewithargsret{\sphinxbfcode{\sphinxupquote{\DUrole{n}{createPopup}}}}{\emph{\DUrole{n}{feature}}, \emph{\DUrole{n}{layer}}}{}
\pysigstopsignatures
\sphinxAtStartPar
Purely for debug atm, we will want to keep feature information within the markers themselves
might be worth considering to display the editor window via the popup (visually too messy?)
\begin{quote}\begin{description}
\sphinxlineitem{Arguments}\begin{itemize}
\item {} 
\sphinxAtStartPar
\sphinxstyleliteralstrong{\sphinxupquote{feature}} (\sphinxstyleliteralemphasis{\sphinxupquote{dict}}) \textendash{} the feature for which we create a popup

\item {} 
\sphinxAtStartPar
\sphinxstyleliteralstrong{\sphinxupquote{layer}} (\sphinxstyleliteralemphasis{\sphinxupquote{leaflet\_layer\_object}}) \textendash{} the leaflet map layer we attach the popup to

\end{itemize}

\end{description}\end{quote}

\end{fulllineitems}


\sphinxstepscope


\subparagraph{add\_delete\_new\_network\_features.js}
\label{\detokenize{docs_gui/js_api/network_editor/add_delete_new_network_features:add-delete-new-network-features-js}}\label{\detokenize{docs_gui/js_api/network_editor/add_delete_new_network_features::doc}}\index{addFeature() (built\sphinxhyphen{}in function)@\spxentry{addFeature()}\spxextra{built\sphinxhyphen{}in function}}

\begin{fulllineitems}
\phantomsection\label{\detokenize{docs_gui/js_api/network_editor/add_delete_new_network_features:addFeature}}
\pysigstartsignatures
\pysiglinewithargsret{\sphinxbfcode{\sphinxupquote{\DUrole{n}{addFeature}}}}{\emph{\DUrole{n}{feature}}}{}
\pysigstopsignatures
\sphinxAtStartPar
onclick function for the add feature buttons in the GUI
switches leaflet map mode to draw and makes sure we place down the correct marker type
\begin{quote}\begin{description}
\sphinxlineitem{Arguments}\begin{itemize}
\item {} 
\sphinxAtStartPar
\sphinxstyleliteralstrong{\sphinxupquote{feature}} (\sphinxstyleliteralemphasis{\sphinxupquote{string}}) \textendash{} name of the network feature type we want to create

\end{itemize}

\end{description}\end{quote}

\end{fulllineitems}

\index{prepareFeatureDelete() (built\sphinxhyphen{}in function)@\spxentry{prepareFeatureDelete()}\spxextra{built\sphinxhyphen{}in function}}

\begin{fulllineitems}
\phantomsection\label{\detokenize{docs_gui/js_api/network_editor/add_delete_new_network_features:prepareFeatureDelete}}
\pysigstartsignatures
\pysiglinewithargsret{\sphinxbfcode{\sphinxupquote{\DUrole{n}{prepareFeatureDelete}}}}{\emph{\DUrole{n}{featureName}}, \emph{\DUrole{n}{featureLists}}}{}
\pysigstopsignatures
\sphinxAtStartPar
if you try to delete a bus, the function tries to find all connected features (lines, ext\_grids, trafos) and marks them as about to be deleted as well
\begin{quote}\begin{description}
\sphinxlineitem{Arguments}\begin{itemize}
\item {} 
\sphinxAtStartPar
\sphinxstyleliteralstrong{\sphinxupquote{featureName}} (\sphinxstyleliteralemphasis{\sphinxupquote{string}}) \textendash{} name of the feature type to delete

\item {} 
\sphinxAtStartPar
\sphinxstyleliteralstrong{\sphinxupquote{featureLists}} (\sphinxstyleliteralemphasis{\sphinxupquote{list}}) \textendash{} contains feature names of all features that may be connected to the deletable feature

\end{itemize}

\end{description}\end{quote}

\end{fulllineitems}

\index{deleteConnectedFeatures() (built\sphinxhyphen{}in function)@\spxentry{deleteConnectedFeatures()}\spxextra{built\sphinxhyphen{}in function}}

\begin{fulllineitems}
\phantomsection\label{\detokenize{docs_gui/js_api/network_editor/add_delete_new_network_features:deleteConnectedFeatures}}
\pysigstartsignatures
\pysiglinewithargsret{\sphinxbfcode{\sphinxupquote{\DUrole{n}{deleteConnectedFeatures}}}}{}{}
\pysigstopsignatures
\sphinxAtStartPar
removes all features attached to a deletable bus from the map, the network object and the feature list gui

\end{fulllineitems}

\index{deleteFeature() (built\sphinxhyphen{}in function)@\spxentry{deleteFeature()}\spxextra{built\sphinxhyphen{}in function}}

\begin{fulllineitems}
\phantomsection\label{\detokenize{docs_gui/js_api/network_editor/add_delete_new_network_features:deleteFeature}}
\pysigstartsignatures
\pysiglinewithargsret{\sphinxbfcode{\sphinxupquote{\DUrole{n}{deleteFeature}}}}{\emph{\DUrole{n}{featureName}}}{}
\pysigstopsignatures
\sphinxAtStartPar
delete function used for features that do not have other features attached to them aka lines, trafos, ext\_grids
\begin{quote}\begin{description}
\sphinxlineitem{Arguments}\begin{itemize}
\item {} 
\sphinxAtStartPar
\sphinxstyleliteralstrong{\sphinxupquote{featureName}} (\sphinxstyleliteralemphasis{\sphinxupquote{string}}) \textendash{} key for the type of the feature we want to delete

\end{itemize}

\end{description}\end{quote}

\end{fulllineitems}


\sphinxstepscope


\subparagraph{collect\_returnable\_network\_features.js}
\label{\detokenize{docs_gui/js_api/network_editor/collect_returnable_network_features:collect-returnable-network-features-js}}\label{\detokenize{docs_gui/js_api/network_editor/collect_returnable_network_features::doc}}\index{returnEditedNet() (built\sphinxhyphen{}in function)@\spxentry{returnEditedNet()}\spxextra{built\sphinxhyphen{}in function}}

\begin{fulllineitems}
\phantomsection\label{\detokenize{docs_gui/js_api/network_editor/collect_returnable_network_features:returnEditedNet}}
\pysigstartsignatures
\pysiglinewithargsret{\sphinxbfcode{\sphinxupquote{\DUrole{n}{returnEditedNet}}}}{}{}
\pysigstopsignatures
\sphinxAtStartPar
executes network feature extraction function and returns network data to the backend

\end{fulllineitems}

\index{extractNetworFeatures() (built\sphinxhyphen{}in function)@\spxentry{extractNetworFeatures()}\spxextra{built\sphinxhyphen{}in function}}

\begin{fulllineitems}
\phantomsection\label{\detokenize{docs_gui/js_api/network_editor/collect_returnable_network_features:extractNetworFeatures}}
\pysigstartsignatures
\pysiglinewithargsret{\sphinxbfcode{\sphinxupquote{\DUrole{n}{extractNetworFeatures}}}}{}{}
\pysigstopsignatures
\sphinxAtStartPar
goes through every list in the network object and extracts features and their properties from the saved map objects
\begin{quote}\begin{description}
\sphinxlineitem{Returns}
\sphinxAtStartPar
dict containing all data of the network to be processed in the backend

\end{description}\end{quote}

\end{fulllineitems}


\sphinxstepscope


\subparagraph{Urbs Editor}
\label{\detokenize{docs_gui/js_api/urbs_editor/index:urbs-editor}}\label{\detokenize{docs_gui/js_api/urbs_editor/index::doc}}
\sphinxstepscope


\subparagraph{buildings\_editor.js}
\label{\detokenize{docs_gui/js_api/urbs_editor/buildings_editor:buildings-editor-js}}\label{\detokenize{docs_gui/js_api/urbs_editor/buildings_editor::doc}}\index{prepareBuildingsObject() (built\sphinxhyphen{}in function)@\spxentry{prepareBuildingsObject()}\spxextra{built\sphinxhyphen{}in function}}

\begin{fulllineitems}
\phantomsection\label{\detokenize{docs_gui/js_api/urbs_editor/buildings_editor:prepareBuildingsObject}}
\pysigstartsignatures
\pysiglinewithargsret{\sphinxbfcode{\sphinxupquote{\DUrole{n}{prepareBuildingsObject}}}}{\emph{\DUrole{n}{UrbsPropertiesJSON}}}{}
\pysigstopsignatures
\sphinxAtStartPar
called during setup of the urbs editor window
Fills the buildingsPropertiesList with dicts containing rudimentary information about each building
Properties of a building are determined by the UrbsPropertiesJSON and set to null initially, while names and coordinates are extracted from the corresponding buses
\begin{quote}\begin{description}
\sphinxlineitem{Arguments}\begin{itemize}
\item {} 
\sphinxAtStartPar
\sphinxstyleliteralstrong{\sphinxupquote{UrbsPropertiesJSON}} (\sphinxstyleliteralemphasis{\sphinxupquote{dict}}) \textendash{} contains information about all inputs for all urbs setup features

\end{itemize}

\end{description}\end{quote}

\end{fulllineitems}

\index{fillSelectedFeatureBuildingEditor() (built\sphinxhyphen{}in function)@\spxentry{fillSelectedFeatureBuildingEditor()}\spxextra{built\sphinxhyphen{}in function}}

\begin{fulllineitems}
\phantomsection\label{\detokenize{docs_gui/js_api/urbs_editor/buildings_editor:fillSelectedFeatureBuildingEditor}}
\pysigstartsignatures
\pysiglinewithargsret{\sphinxbfcode{\sphinxupquote{\DUrole{n}{fillSelectedFeatureBuildingEditor}}}}{\emph{\DUrole{n}{target}}}{}
\pysigstopsignatures
\sphinxAtStartPar
onclick method for bus circlemarkers on the map and onchange method of the buildings list element
resets the style of the previously selected bus and fills all editor input fields with the correct values
\begin{quote}\begin{description}
\sphinxlineitem{Arguments}\begin{itemize}
\item {} 
\sphinxAtStartPar
\sphinxstyleliteralstrong{\sphinxupquote{target}} (\sphinxstyleliteralemphasis{\sphinxupquote{event\_target\_object}}) \textendash{} the map object that has been selected

\end{itemize}

\end{description}\end{quote}

\end{fulllineitems}

\index{writeBackEditedBuildingFeatures() (built\sphinxhyphen{}in function)@\spxentry{writeBackEditedBuildingFeatures()}\spxextra{built\sphinxhyphen{}in function}}

\begin{fulllineitems}
\phantomsection\label{\detokenize{docs_gui/js_api/urbs_editor/buildings_editor:writeBackEditedBuildingFeatures}}
\pysigstartsignatures
\pysiglinewithargsret{\sphinxbfcode{\sphinxupquote{\DUrole{n}{writeBackEditedBuildingFeatures}}}}{\emph{\DUrole{n}{target}}}{}
\pysigstopsignatures
\sphinxAtStartPar
onchange function for all building editor input fields
writes changed value back to the relevant entry in the BuildingsObject
\begin{quote}\begin{description}
\sphinxlineitem{Arguments}\begin{itemize}
\item {} 
\sphinxAtStartPar
\sphinxstyleliteralstrong{\sphinxupquote{target}} (\sphinxstyleliteralemphasis{\sphinxupquote{event\_target\_object}}) \textendash{} the html editor element that has been changed

\end{itemize}

\end{description}\end{quote}

\end{fulllineitems}


\sphinxstepscope


\subparagraph{commodity\_editor.js}
\label{\detokenize{docs_gui/js_api/urbs_editor/commodity_editor:commodity-editor-js}}\label{\detokenize{docs_gui/js_api/urbs_editor/commodity_editor::doc}}\index{fetchCommodityProfiles() (built\sphinxhyphen{}in function)@\spxentry{fetchCommodityProfiles()}\spxextra{built\sphinxhyphen{}in function}}

\begin{fulllineitems}
\phantomsection\label{\detokenize{docs_gui/js_api/urbs_editor/commodity_editor:fetchCommodityProfiles}}
\pysigstartsignatures
\pysiglinewithargsret{\sphinxbfcode{\sphinxupquote{\DUrole{n}{fetchCommodityProfiles}}}}{}{}
\pysigstopsignatures
\sphinxAtStartPar
retrieves commodity feature templates from the backend and generates a dict for each commodity, holding default values for all inputs

\end{fulllineitems}

\index{openNewCommodityForm() (built\sphinxhyphen{}in function)@\spxentry{openNewCommodityForm()}\spxextra{built\sphinxhyphen{}in function}}

\begin{fulllineitems}
\phantomsection\label{\detokenize{docs_gui/js_api/urbs_editor/commodity_editor:openNewCommodityForm}}
\pysigstartsignatures
\pysiglinewithargsret{\sphinxbfcode{\sphinxupquote{\DUrole{n}{openNewCommodityForm}}}}{}{}
\pysigstopsignatures
\sphinxAtStartPar
opens the GUI form for creating a new commodity

\end{fulllineitems}

\index{closeNewCommodityForm() (built\sphinxhyphen{}in function)@\spxentry{closeNewCommodityForm()}\spxextra{built\sphinxhyphen{}in function}}

\begin{fulllineitems}
\phantomsection\label{\detokenize{docs_gui/js_api/urbs_editor/commodity_editor:closeNewCommodityForm}}
\pysigstartsignatures
\pysiglinewithargsret{\sphinxbfcode{\sphinxupquote{\DUrole{n}{closeNewCommodityForm}}}}{}{}
\pysigstopsignatures
\sphinxAtStartPar
closes the GUI form for creating a new commodity and resets its input values

\end{fulllineitems}

\index{commodityFormCheckValidInput() (built\sphinxhyphen{}in function)@\spxentry{commodityFormCheckValidInput()}\spxextra{built\sphinxhyphen{}in function}}

\begin{fulllineitems}
\phantomsection\label{\detokenize{docs_gui/js_api/urbs_editor/commodity_editor:commodityFormCheckValidInput}}
\pysigstartsignatures
\pysiglinewithargsret{\sphinxbfcode{\sphinxupquote{\DUrole{n}{commodityFormCheckValidInput}}}}{\emph{\DUrole{n}{comm\_name}}}{}
\pysigstopsignatures
\sphinxAtStartPar
makes sure we cannot create a commodity with now name and
enables or disables the confirm button in the new commodity form
\begin{quote}\begin{description}
\sphinxlineitem{Arguments}\begin{itemize}
\item {} 
\sphinxAtStartPar
\sphinxstyleliteralstrong{\sphinxupquote{comm\_name}} (\sphinxstyleliteralemphasis{\sphinxupquote{string}}) \textendash{} text contained in the input field for a new commodity name

\end{itemize}

\end{description}\end{quote}

\end{fulllineitems}

\index{addCommToProcessCreationFormList() (built\sphinxhyphen{}in function)@\spxentry{addCommToProcessCreationFormList()}\spxextra{built\sphinxhyphen{}in function}}

\begin{fulllineitems}
\phantomsection\label{\detokenize{docs_gui/js_api/urbs_editor/commodity_editor:addCommToProcessCreationFormList}}
\pysigstartsignatures
\pysiglinewithargsret{\sphinxbfcode{\sphinxupquote{\DUrole{n}{addCommToProcessCreationFormList}}}}{\emph{\DUrole{n}{name}}}{}
\pysigstopsignatures
\sphinxAtStartPar
Once a new commodity has been created, this function adds it to the GUI forms for creating new commodities and adding commodities to processes
\begin{quote}\begin{description}
\sphinxlineitem{Arguments}\begin{itemize}
\item {} 
\sphinxAtStartPar
\sphinxstyleliteralstrong{\sphinxupquote{name}} (\sphinxstyleliteralemphasis{\sphinxupquote{string}}) \textendash{} of the new commodity

\end{itemize}

\end{description}\end{quote}

\end{fulllineitems}

\index{addCommToStorageComList() (built\sphinxhyphen{}in function)@\spxentry{addCommToStorageComList()}\spxextra{built\sphinxhyphen{}in function}}

\begin{fulllineitems}
\phantomsection\label{\detokenize{docs_gui/js_api/urbs_editor/commodity_editor:addCommToStorageComList}}
\pysigstartsignatures
\pysiglinewithargsret{\sphinxbfcode{\sphinxupquote{\DUrole{n}{addCommToStorageComList}}}}{\emph{\DUrole{n}{name}}}{}
\pysigstopsignatures
\sphinxAtStartPar
Once a new commodity has been created, this function adds it to the GUI form of the storage editor
\begin{quote}\begin{description}
\sphinxlineitem{Arguments}\begin{itemize}
\item {} 
\sphinxAtStartPar
\sphinxstyleliteralstrong{\sphinxupquote{name}} (\sphinxstyleliteralemphasis{\sphinxupquote{string}}) \textendash{} the name of the commodity

\end{itemize}

\end{description}\end{quote}

\end{fulllineitems}

\index{createNewCommodity() (built\sphinxhyphen{}in function)@\spxentry{createNewCommodity()}\spxextra{built\sphinxhyphen{}in function}}

\begin{fulllineitems}
\phantomsection\label{\detokenize{docs_gui/js_api/urbs_editor/commodity_editor:createNewCommodity}}
\pysigstartsignatures
\pysiglinewithargsret{\sphinxbfcode{\sphinxupquote{\DUrole{n}{createNewCommodity}}}}{\emph{\DUrole{n}{com\_name}}}{}
\pysigstopsignatures
\sphinxAtStartPar
adds the newly created commodity to the commodity editor list, the pro\_conf table and the CommodityObject
\begin{quote}\begin{description}
\sphinxlineitem{Arguments}\begin{itemize}
\item {} 
\sphinxAtStartPar
\sphinxstyleliteralstrong{\sphinxupquote{com\_name}} (\sphinxstyleliteralemphasis{\sphinxupquote{string}}) \textendash{} the name of the commodity

\end{itemize}

\end{description}\end{quote}

\end{fulllineitems}

\index{writeBackCommodityFeatures() (built\sphinxhyphen{}in function)@\spxentry{writeBackCommodityFeatures()}\spxextra{built\sphinxhyphen{}in function}}

\begin{fulllineitems}
\phantomsection\label{\detokenize{docs_gui/js_api/urbs_editor/commodity_editor:writeBackCommodityFeatures}}
\pysigstartsignatures
\pysiglinewithargsret{\sphinxbfcode{\sphinxupquote{\DUrole{n}{writeBackCommodityFeatures}}}}{\emph{\DUrole{n}{target}}}{}
\pysigstopsignatures
\sphinxAtStartPar
onchange function for all commodity editor input fields
writes changed value back to the relevant entry in the CommodityObject
\begin{quote}\begin{description}
\sphinxlineitem{Arguments}\begin{itemize}
\item {} 
\sphinxAtStartPar
\sphinxstyleliteralstrong{\sphinxupquote{target}} (\sphinxstyleliteralemphasis{\sphinxupquote{event\_target\_object}}) \textendash{} the html element whose onchange event called the function

\end{itemize}

\end{description}\end{quote}

\end{fulllineitems}


\sphinxstepscope


\subparagraph{demand\_editor.js}
\label{\detokenize{docs_gui/js_api/urbs_editor/demand_editor:demand-editor-js}}\label{\detokenize{docs_gui/js_api/urbs_editor/demand_editor::doc}}\index{fetchDemandProfiles() (built\sphinxhyphen{}in function)@\spxentry{fetchDemandProfiles()}\spxextra{built\sphinxhyphen{}in function}}

\begin{fulllineitems}
\phantomsection\label{\detokenize{docs_gui/js_api/urbs_editor/demand_editor:fetchDemandProfiles}}
\pysigstartsignatures
\pysiglinewithargsret{\sphinxbfcode{\sphinxupquote{\DUrole{n}{fetchDemandProfiles}}}}{}{}
\pysigstopsignatures
\sphinxAtStartPar
retrieves all demand profiles from the backend
\begin{quote}\begin{description}
\sphinxlineitem{Returns}
\sphinxAtStartPar
Promise to make sure functions during setup are called in order

\end{description}\end{quote}

\end{fulllineitems}

\index{fillSelectedFeatureDemandEditor() (built\sphinxhyphen{}in function)@\spxentry{fillSelectedFeatureDemandEditor()}\spxextra{built\sphinxhyphen{}in function}}

\begin{fulllineitems}
\phantomsection\label{\detokenize{docs_gui/js_api/urbs_editor/demand_editor:fillSelectedFeatureDemandEditor}}
\pysigstartsignatures
\pysiglinewithargsret{\sphinxbfcode{\sphinxupquote{\DUrole{n}{fillSelectedFeatureDemandEditor}}}}{\emph{\DUrole{n}{target}}}{}
\pysigstopsignatures
\sphinxAtStartPar
onclick function for map features, fills editor with the demand info of the clicked feature
\begin{quote}\begin{description}
\sphinxlineitem{Arguments}\begin{itemize}
\item {} 
\sphinxAtStartPar
\sphinxstyleliteralstrong{\sphinxupquote{target}} (\sphinxstyleliteralemphasis{\sphinxupquote{event\_target\_object}}) \textendash{} map feature that has been selected

\end{itemize}

\end{description}\end{quote}

\end{fulllineitems}

\index{populateDemandEditor() (built\sphinxhyphen{}in function)@\spxentry{populateDemandEditor()}\spxextra{built\sphinxhyphen{}in function}}

\begin{fulllineitems}
\phantomsection\label{\detokenize{docs_gui/js_api/urbs_editor/demand_editor:populateDemandEditor}}
\pysigstartsignatures
\pysiglinewithargsret{\sphinxbfcode{\sphinxupquote{\DUrole{n}{populateDemandEditor}}}}{\emph{\DUrole{n}{demand\_data}}, \emph{\DUrole{n}{demandName}}, \emph{\DUrole{n}{demandIndex}}}{}
\pysigstopsignatures
\sphinxAtStartPar
TODO: switch demand data with Object.keys(demand\_data).length \sphinxhyphen{} 1 as parameter, no need to pass entire object
creates checkboxes for every profile of a demand type and attaches them to the correct panel in the demand editor div
\begin{quote}\begin{description}
\sphinxlineitem{Arguments}\begin{itemize}
\item {} 
\sphinxAtStartPar
\sphinxstyleliteralstrong{\sphinxupquote{demand\_data}} (\sphinxstyleliteralemphasis{\sphinxupquote{dict}}) \textendash{} 

\item {} 
\sphinxAtStartPar
\sphinxstyleliteralstrong{\sphinxupquote{demandName}} (\sphinxstyleliteralemphasis{\sphinxupquote{string}}) \textendash{} key for getting html div container and setting checkbox onclick functions

\item {} 
\sphinxAtStartPar
\sphinxstyleliteralstrong{\sphinxupquote{demandIndex}} (\sphinxstyleliteralemphasis{\sphinxupquote{int}}) \textendash{} needed for setup of the onclick function

\end{itemize}

\end{description}\end{quote}

\end{fulllineitems}

\index{check\_uncheck\_demand() (built\sphinxhyphen{}in function)@\spxentry{check\_uncheck\_demand()}\spxextra{built\sphinxhyphen{}in function}}

\begin{fulllineitems}
\phantomsection\label{\detokenize{docs_gui/js_api/urbs_editor/demand_editor:check_uncheck_demand}}
\pysigstartsignatures
\pysiglinewithargsret{\sphinxbfcode{\sphinxupquote{\DUrole{n}{check\_uncheck\_demand}}}}{\emph{\DUrole{n}{checkbox}}, \emph{\DUrole{n}{demand\_type}}, \emph{\DUrole{n}{key}}, \emph{\DUrole{n}{demandIndex}}}{}
\pysigstopsignatures
\sphinxAtStartPar
removes or adds a graph to the demand chart and marks whether the checkbox is set in the DemandObject
\begin{quote}\begin{description}
\sphinxlineitem{Arguments}\begin{itemize}
\item {} 
\sphinxAtStartPar
\sphinxstyleliteralstrong{\sphinxupquote{checkbox}} (\sphinxstyleliteralemphasis{\sphinxupquote{HTML\_element}}) \textendash{} checkbox html element whose onchange event triggered the function calls

\item {} 
\sphinxAtStartPar
\sphinxstyleliteralstrong{\sphinxupquote{demand\_type}} (\sphinxstyleliteralemphasis{\sphinxupquote{string}}) \textendash{} which demand category the checkbox belongs to

\item {} 
\sphinxAtStartPar
\sphinxstyleliteralstrong{\sphinxupquote{key}} (\sphinxstyleliteralemphasis{\sphinxupquote{int}}) \textendash{} the index of the checkbox in the list

\item {} 
\sphinxAtStartPar
\sphinxstyleliteralstrong{\sphinxupquote{demandIndex}} (\sphinxstyleliteralemphasis{\sphinxupquote{int}}) \textendash{} the index of the demand category

\end{itemize}

\end{description}\end{quote}

\end{fulllineitems}


\sphinxstepscope


\subparagraph{generate\_urbs\_setup\_editor.js}
\label{\detokenize{docs_gui/js_api/urbs_editor/generate_urbs_setup_editor:generate-urbs-setup-editor-js}}\label{\detokenize{docs_gui/js_api/urbs_editor/generate_urbs_setup_editor::doc}}\index{GetUrbsSetupProperties() (built\sphinxhyphen{}in function)@\spxentry{GetUrbsSetupProperties()}\spxextra{built\sphinxhyphen{}in function}}

\begin{fulllineitems}
\phantomsection\label{\detokenize{docs_gui/js_api/urbs_editor/generate_urbs_setup_editor:GetUrbsSetupProperties}}
\pysigstartsignatures
\pysiglinewithargsret{\sphinxbfcode{\sphinxupquote{\DUrole{n}{GetUrbsSetupProperties}}}}{}{}
\pysigstopsignatures
\sphinxAtStartPar
function that fetches the urbsPropertyJSON file stored in the backend. The file defines all inputs for all features as well as tooltips, types and default values
\begin{quote}\begin{description}
\sphinxlineitem{Returns}
\sphinxAtStartPar
Promise to make sure functions during setup are called in order

\end{description}\end{quote}

\end{fulllineitems}

\index{SetupUrbsEditor() (built\sphinxhyphen{}in function)@\spxentry{SetupUrbsEditor()}\spxextra{built\sphinxhyphen{}in function}}

\begin{fulllineitems}
\phantomsection\label{\detokenize{docs_gui/js_api/urbs_editor/generate_urbs_setup_editor:SetupUrbsEditor}}
\pysigstartsignatures
\pysiglinewithargsret{\sphinxbfcode{\sphinxupquote{\DUrole{n}{SetupUrbsEditor}}}}{}{}
\pysigstopsignatures
\sphinxAtStartPar
main aggregate function for editor generation: All the different preparatory functions for each editor component are called here
after the network data is retrieved from Flask

\end{fulllineitems}

\index{displayUrbsEditorNet() (built\sphinxhyphen{}in function)@\spxentry{displayUrbsEditorNet()}\spxextra{built\sphinxhyphen{}in function}}

\begin{fulllineitems}
\phantomsection\label{\detokenize{docs_gui/js_api/urbs_editor/generate_urbs_setup_editor:displayUrbsEditorNet}}
\pysigstartsignatures
\pysiglinewithargsret{\sphinxbfcode{\sphinxupquote{\DUrole{n}{displayUrbsEditorNet}}}}{\emph{\DUrole{n}{ppdata}}}{}
\pysigstopsignatures
\sphinxAtStartPar
aggregate function for displaying the full network
\begin{quote}\begin{description}
\sphinxlineitem{Arguments}\begin{itemize}
\item {} 
\sphinxAtStartPar
\sphinxstyleliteralstrong{\sphinxupquote{ppdata}} (\sphinxstyleliteralemphasis{\sphinxupquote{dict}}) \textendash{} the grid data

\end{itemize}

\end{description}\end{quote}

\end{fulllineitems}

\index{addGeoJSONtoUrbsEditorMap() (built\sphinxhyphen{}in function)@\spxentry{addGeoJSONtoUrbsEditorMap()}\spxextra{built\sphinxhyphen{}in function}}

\begin{fulllineitems}
\phantomsection\label{\detokenize{docs_gui/js_api/urbs_editor/generate_urbs_setup_editor:addGeoJSONtoUrbsEditorMap}}
\pysigstartsignatures
\pysiglinewithargsret{\sphinxbfcode{\sphinxupquote{\DUrole{n}{addGeoJSONtoUrbsEditorMap}}}}{\emph{\DUrole{n}{isLines}}, \emph{\DUrole{n}{input\_geoJSON}}, \emph{\DUrole{n}{featureName}}}{}
\pysigstopsignatures
\sphinxAtStartPar
creates a new geojson layer for leaflet map
distinguishes between line (for lines and trafos) and circle marker (for buses and ext\_grids) formats
because line geojsons do not have the pointToLayer option
\begin{quote}\begin{description}
\sphinxlineitem{Arguments}\begin{itemize}
\item {} 
\sphinxAtStartPar
\sphinxstyleliteralstrong{\sphinxupquote{isLines}} (\sphinxstyleliteralemphasis{\sphinxupquote{boolean}}) \textendash{} distinguishes if the geometry we want to place on the map is a point or a line

\item {} 
\sphinxAtStartPar
\sphinxstyleliteralstrong{\sphinxupquote{input\_geoJSON}} (\sphinxstyleliteralemphasis{\sphinxupquote{geoJSON\_FeatureCollection}}) \textendash{} the data structure containing all our grid information

\item {} 
\sphinxAtStartPar
\sphinxstyleliteralstrong{\sphinxupquote{featureName}} (\sphinxstyleliteralemphasis{\sphinxupquote{string}}) \textendash{} name of the feature type to be added to the map (e.g. line, bus etc)

\end{itemize}

\end{description}\end{quote}

\end{fulllineitems}

\index{populateUrbsEditorLoadBusLists() (built\sphinxhyphen{}in function)@\spxentry{populateUrbsEditorLoadBusLists()}\spxextra{built\sphinxhyphen{}in function}}

\begin{fulllineitems}
\phantomsection\label{\detokenize{docs_gui/js_api/urbs_editor/generate_urbs_setup_editor:populateUrbsEditorLoadBusLists}}
\pysigstartsignatures
\pysiglinewithargsret{\sphinxbfcode{\sphinxupquote{\DUrole{n}{populateUrbsEditorLoadBusLists}}}}{\emph{\DUrole{n}{htmlListName}}}{}
\pysigstopsignatures\begin{quote}\begin{description}
\sphinxlineitem{Arguments}\begin{itemize}
\item {} 
\sphinxAtStartPar
\sphinxstyleliteralstrong{\sphinxupquote{htmlListName}} (\sphinxstyleliteralemphasis{\sphinxupquote{string}}) \textendash{} name of the html select element that is supposed to be filled with the buses with attached loads

\end{itemize}

\end{description}\end{quote}

\end{fulllineitems}

\index{populateUrbsEditor() (built\sphinxhyphen{}in function)@\spxentry{populateUrbsEditor()}\spxextra{built\sphinxhyphen{}in function}}

\begin{fulllineitems}
\phantomsection\label{\detokenize{docs_gui/js_api/urbs_editor/generate_urbs_setup_editor:populateUrbsEditor}}
\pysigstartsignatures
\pysiglinewithargsret{\sphinxbfcode{\sphinxupquote{\DUrole{n}{populateUrbsEditor}}}}{\emph{\DUrole{n}{feature}}, \emph{\DUrole{n}{propertiesToAdd}}, \emph{\DUrole{n}{writebackFunction}}}{}
\pysigstopsignatures
\sphinxAtStartPar
We create the form and div elements holding input fields and associated labels from scratch right after page load and attach them to the corresponding static
editor div
\begin{quote}\begin{description}
\sphinxlineitem{Arguments}\begin{itemize}
\item {} 
\sphinxAtStartPar
\sphinxstyleliteralstrong{\sphinxupquote{feature}} (\sphinxstyleliteralemphasis{\sphinxupquote{string}}) \textendash{} name of the feature (i.e demand, process etc) for which input fields are to be generated

\item {} 
\sphinxAtStartPar
\sphinxstyleliteralstrong{\sphinxupquote{propertiesToAdd}} (\sphinxstyleliteralemphasis{\sphinxupquote{dict}}) \textendash{} properties for which inputs are created, as well as tooltips, types and default values loaded from the UrbsPropertiesJSON file fetched from the backend

\item {} 
\sphinxAtStartPar
\sphinxstyleliteralstrong{\sphinxupquote{writebackFunction}} (\sphinxstyleliteralemphasis{\sphinxupquote{function}}) \textendash{} since the storage objects for each feature may not be the same, we can pass a custom function for each feature that

\end{itemize}

\end{description}\end{quote}

\end{fulllineitems}

\index{populateUrbsNetworkEditorSecondaryFeature() (built\sphinxhyphen{}in function)@\spxentry{populateUrbsNetworkEditorSecondaryFeature()}\spxextra{built\sphinxhyphen{}in function}}

\begin{fulllineitems}
\phantomsection\label{\detokenize{docs_gui/js_api/urbs_editor/generate_urbs_setup_editor:populateUrbsNetworkEditorSecondaryFeature}}
\pysigstartsignatures
\pysiglinewithargsret{\sphinxbfcode{\sphinxupquote{\DUrole{n}{populateUrbsNetworkEditorSecondaryFeature}}}}{\emph{\DUrole{n}{primaryFeatureName}}, \emph{\DUrole{n}{secondaryFeatureName}}}{}
\pysigstopsignatures
\sphinxAtStartPar
creates html elements for the secondary feature editor and adds select options for the secondary features
\begin{quote}\begin{description}
\sphinxlineitem{Arguments}\begin{itemize}
\item {} 
\sphinxAtStartPar
\sphinxstyleliteralstrong{\sphinxupquote{primaryFeatureName}} (\sphinxstyleliteralemphasis{\sphinxupquote{string}}) \textendash{} Key for the primary feature editor to which the secondary feature list is getting attached to

\item {} 
\sphinxAtStartPar
\sphinxstyleliteralstrong{\sphinxupquote{secondaryFeatureName}} (\sphinxstyleliteralemphasis{\sphinxupquote{string}}) \textendash{} Key for the secondary feature

\end{itemize}

\end{description}\end{quote}

\end{fulllineitems}

\index{openUrbsEditorList() (built\sphinxhyphen{}in function)@\spxentry{openUrbsEditorList()}\spxextra{built\sphinxhyphen{}in function}}

\begin{fulllineitems}
\phantomsection\label{\detokenize{docs_gui/js_api/urbs_editor/generate_urbs_setup_editor:openUrbsEditorList}}
\pysigstartsignatures
\pysiglinewithargsret{\sphinxbfcode{\sphinxupquote{\DUrole{n}{openUrbsEditorList}}}}{\emph{\DUrole{n}{e}}, \emph{\DUrole{n}{listName}}, \emph{\DUrole{n}{hasEditor}}}{}
\pysigstopsignatures
\sphinxAtStartPar
Function gets called when one of the tablink buttons in the GUI gets pressed and opens the relevant feature list, while hiding all other GUI elements
\begin{quote}\begin{description}
\sphinxlineitem{Arguments}\begin{itemize}
\item {} 
\sphinxAtStartPar
\sphinxstyleliteralstrong{\sphinxupquote{e}} (\sphinxstyleliteralemphasis{\sphinxupquote{event}}) \textendash{} object for the onclick event of the clicked tablink button, necassary to change the button to active

\item {} 
\sphinxAtStartPar
\sphinxstyleliteralstrong{\sphinxupquote{listName}} (\sphinxstyleliteralemphasis{\sphinxupquote{string}}) \textendash{} key to access the relevant list tab html element by id

\item {} 
\sphinxAtStartPar
\sphinxstyleliteralstrong{\sphinxupquote{hasEditor}} (\sphinxstyleliteralemphasis{\sphinxupquote{boolean}}) \textendash{} some features (like global) do not have a separate editor which means that the section of the code that opens previously open editors doesn’t apply to them. This boolean acts as a flag to ensure that portion is not called

\end{itemize}

\end{description}\end{quote}

\end{fulllineitems}

\index{resetLoadBusStyle() (built\sphinxhyphen{}in function)@\spxentry{resetLoadBusStyle()}\spxextra{built\sphinxhyphen{}in function}}

\begin{fulllineitems}
\phantomsection\label{\detokenize{docs_gui/js_api/urbs_editor/generate_urbs_setup_editor:resetLoadBusStyle}}
\pysigstartsignatures
\pysiglinewithargsret{\sphinxbfcode{\sphinxupquote{\DUrole{n}{resetLoadBusStyle}}}}{\emph{\DUrole{n}{target}}}{}
\pysigstopsignatures
\sphinxAtStartPar
function that switches bus styles back to default once they are deselected when the user clicks on another node on the map or another element in the list
\begin{quote}\begin{description}
\sphinxlineitem{Arguments}\begin{itemize}
\item {} 
\sphinxAtStartPar
\sphinxstyleliteralstrong{\sphinxupquote{target}} (\sphinxstyleliteralemphasis{\sphinxupquote{event\_target\_object}}) \textendash{} the leaflet object whose onclick method has been triggered via click on the map or selection via the list

\end{itemize}

\end{description}\end{quote}

\end{fulllineitems}

\index{highlightSelectedElementInList() (built\sphinxhyphen{}in function)@\spxentry{highlightSelectedElementInList()}\spxextra{built\sphinxhyphen{}in function}}

\begin{fulllineitems}
\phantomsection\label{\detokenize{docs_gui/js_api/urbs_editor/generate_urbs_setup_editor:highlightSelectedElementInList}}
\pysigstartsignatures
\pysiglinewithargsret{\sphinxbfcode{\sphinxupquote{\DUrole{n}{highlightSelectedElementInList}}}}{\emph{\DUrole{n}{target}}, \emph{\DUrole{n}{selectId}}}{}
\pysigstopsignatures
\sphinxAtStartPar
Method that changes the selectedIndex of a given select element to the option corresponding to the onclick event target
\begin{quote}\begin{description}
\sphinxlineitem{Arguments}\begin{itemize}
\item {} 
\sphinxAtStartPar
\sphinxstyleliteralstrong{\sphinxupquote{target}} (\sphinxstyleliteralemphasis{\sphinxupquote{event\_target\_object}}) \textendash{} the leaflet object whose onclick method has been triggered via click on the map or selection via the list

\item {} 
\sphinxAtStartPar
\sphinxstyleliteralstrong{\sphinxupquote{selectId}} (\sphinxstyleliteralemphasis{\sphinxupquote{string}}) \textendash{} id of the relevant html element

\end{itemize}

\end{description}\end{quote}

\end{fulllineitems}

\index{fillSelectedEditor() (built\sphinxhyphen{}in function)@\spxentry{fillSelectedEditor()}\spxextra{built\sphinxhyphen{}in function}}

\begin{fulllineitems}
\phantomsection\label{\detokenize{docs_gui/js_api/urbs_editor/generate_urbs_setup_editor:fillSelectedEditor}}
\pysigstartsignatures
\pysiglinewithargsret{\sphinxbfcode{\sphinxupquote{\DUrole{n}{fillSelectedEditor}}}}{\emph{\DUrole{n}{sel}}, \emph{\DUrole{n}{featureName}}}{}
\pysigstopsignatures
\sphinxAtStartPar
An aggregate function that calls the relevant editor fill method for each feature and makes sure other editor windows are closeds
\begin{quote}\begin{description}
\sphinxlineitem{Arguments}\begin{itemize}
\item {} 
\sphinxAtStartPar
\sphinxstyleliteralstrong{\sphinxupquote{sel}} (\sphinxstyleliteralemphasis{\sphinxupquote{HTML\_select\_element}}) \textendash{} gets passed as “this” reference when the onchange method for the select element is called, needed to retrieve the currently selected secondary feature

\item {} 
\sphinxAtStartPar
\sphinxstyleliteralstrong{\sphinxupquote{featureName}} (\sphinxstyleliteralemphasis{\sphinxupquote{string}}) \textendash{} key for corresponding html element and call of correct editor fill method

\end{itemize}

\end{description}\end{quote}

\end{fulllineitems}

\index{fillSelectedFeatureEditorFields() (built\sphinxhyphen{}in function)@\spxentry{fillSelectedFeatureEditorFields()}\spxextra{built\sphinxhyphen{}in function}}

\begin{fulllineitems}
\phantomsection\label{\detokenize{docs_gui/js_api/urbs_editor/generate_urbs_setup_editor:fillSelectedFeatureEditorFields}}
\pysigstartsignatures
\pysiglinewithargsret{\sphinxbfcode{\sphinxupquote{\DUrole{n}{fillSelectedFeatureEditorFields}}}}{\emph{\DUrole{n}{target}}, \emph{\DUrole{n}{featureName}}}{}
\pysigstopsignatures
\sphinxAtStartPar
Function that fills each input field of a single feature once the editor window is opened
\begin{quote}\begin{description}
\sphinxlineitem{Arguments}\begin{itemize}
\item {} 
\sphinxAtStartPar
\sphinxstyleliteralstrong{\sphinxupquote{target}} (\sphinxstyleliteralemphasis{\sphinxupquote{dict}}) \textendash{} Object that contains all current input values for a single feature

\item {} 
\sphinxAtStartPar
\sphinxstyleliteralstrong{\sphinxupquote{featureName}} (\sphinxstyleliteralemphasis{\sphinxupquote{string}}) \textendash{} key used to access the html elements

\end{itemize}

\end{description}\end{quote}

\end{fulllineitems}

\index{getUrbsPropertiesJSON() (built\sphinxhyphen{}in function)@\spxentry{getUrbsPropertiesJSON()}\spxextra{built\sphinxhyphen{}in function}}

\begin{fulllineitems}
\phantomsection\label{\detokenize{docs_gui/js_api/urbs_editor/generate_urbs_setup_editor:getUrbsPropertiesJSON}}
\pysigstartsignatures
\pysiglinewithargsret{\sphinxbfcode{\sphinxupquote{\DUrole{n}{getUrbsPropertiesJSON}}}}{}{}
\pysigstopsignatures
\sphinxAtStartPar
getter function for the UrbsPropertiesJSON
\begin{quote}\begin{description}
\sphinxlineitem{Returns}
\sphinxAtStartPar
urbs properties dict

\end{description}\end{quote}

\end{fulllineitems}


\sphinxstepscope


\subparagraph{process\_editor.js}
\label{\detokenize{docs_gui/js_api/urbs_editor/process_editor:process-editor-js}}\label{\detokenize{docs_gui/js_api/urbs_editor/process_editor::doc}}\index{fetchProcessProfiles() (built\sphinxhyphen{}in function)@\spxentry{fetchProcessProfiles()}\spxextra{built\sphinxhyphen{}in function}}

\begin{fulllineitems}
\phantomsection\label{\detokenize{docs_gui/js_api/urbs_editor/process_editor:fetchProcessProfiles}}
\pysigstartsignatures
\pysiglinewithargsret{\sphinxbfcode{\sphinxupquote{\DUrole{n}{fetchProcessProfiles}}}}{}{}
\pysigstopsignatures
\sphinxAtStartPar
retrieves pre\sphinxhyphen{}existing process property data from the pandapower2urbs templates in the backend

\end{fulllineitems}

\index{createProcessJSONTemplates() (built\sphinxhyphen{}in function)@\spxentry{createProcessJSONTemplates()}\spxextra{built\sphinxhyphen{}in function}}

\begin{fulllineitems}
\phantomsection\label{\detokenize{docs_gui/js_api/urbs_editor/process_editor:createProcessJSONTemplates}}
\pysigstartsignatures
\pysiglinewithargsret{\sphinxbfcode{\sphinxupquote{\DUrole{n}{createProcessJSONTemplates}}}}{\emph{\DUrole{n}{processes}}, \emph{\DUrole{n}{process\_commodities}}}{}
\pysigstopsignatures
\sphinxAtStartPar
creates dict entry templates to use if we want to add new processes or new process commodities
\begin{quote}\begin{description}
\sphinxlineitem{Arguments}\begin{itemize}
\item {} 
\sphinxAtStartPar
\sphinxstyleliteralstrong{\sphinxupquote{processes}} (\sphinxstyleliteralemphasis{\sphinxupquote{dict}}) \textendash{} dict with process feature:value key:value pairs

\item {} 
\sphinxAtStartPar
\sphinxstyleliteralstrong{\sphinxupquote{process\_commodities}} (\sphinxstyleliteralemphasis{\sphinxupquote{dict}}) \textendash{} dict with pro\_com feature:value key:value pairs

\end{itemize}

\end{description}\end{quote}

\end{fulllineitems}

\index{populateProcessEditorList() (built\sphinxhyphen{}in function)@\spxentry{populateProcessEditorList()}\spxextra{built\sphinxhyphen{}in function}}

\begin{fulllineitems}
\phantomsection\label{\detokenize{docs_gui/js_api/urbs_editor/process_editor:populateProcessEditorList}}
\pysigstartsignatures
\pysiglinewithargsret{\sphinxbfcode{\sphinxupquote{\DUrole{n}{populateProcessEditorList}}}}{\emph{\DUrole{n}{htmlListName}}, \emph{\DUrole{n}{listEntries}}}{}
\pysigstopsignatures
\sphinxAtStartPar
we create options with text based on the keys of a dict and attach them to a html select
\begin{quote}\begin{description}
\sphinxlineitem{Arguments}\begin{itemize}
\item {} 
\sphinxAtStartPar
\sphinxstyleliteralstrong{\sphinxupquote{htmlListName}} (\sphinxstyleliteralemphasis{\sphinxupquote{string}}) \textendash{} id of the html select element we want to add options to

\item {} 
\sphinxAtStartPar
\sphinxstyleliteralstrong{\sphinxupquote{listEntries}} (\sphinxstyleliteralemphasis{\sphinxupquote{dict}}) \textendash{} dict containing data for all options we want to add

\end{itemize}

\end{description}\end{quote}

\end{fulllineitems}

\index{fillSecondaryEditorList() (built\sphinxhyphen{}in function)@\spxentry{fillSecondaryEditorList()}\spxextra{built\sphinxhyphen{}in function}}

\begin{fulllineitems}
\phantomsection\label{\detokenize{docs_gui/js_api/urbs_editor/process_editor:fillSecondaryEditorList}}
\pysigstartsignatures
\pysiglinewithargsret{\sphinxbfcode{\sphinxupquote{\DUrole{n}{fillSecondaryEditorList}}}}{\emph{\DUrole{n}{target\_properties}}}{}
\pysigstopsignatures
\sphinxAtStartPar
called at runtime to make sure only the correct elements are displayed in the pro\_com\_prop list. All options whose names are keys in target\_properties
are made visible, all others are hidden
The pro\_com\_propSelect element technically contains options for all commodities added to all processes, but we hide all options not added to the
currently selected process
\begin{quote}\begin{description}
\sphinxlineitem{Arguments}\begin{itemize}
\item {} 
\sphinxAtStartPar
\sphinxstyleliteralstrong{\sphinxupquote{target\_properties}} (\sphinxstyleliteralemphasis{\sphinxupquote{dict}}) \textendash{} 

\end{itemize}

\end{description}\end{quote}

\end{fulllineitems}

\index{openNewProcessForm() (built\sphinxhyphen{}in function)@\spxentry{openNewProcessForm()}\spxextra{built\sphinxhyphen{}in function}}

\begin{fulllineitems}
\phantomsection\label{\detokenize{docs_gui/js_api/urbs_editor/process_editor:openNewProcessForm}}
\pysigstartsignatures
\pysiglinewithargsret{\sphinxbfcode{\sphinxupquote{\DUrole{n}{openNewProcessForm}}}}{\emph{\DUrole{n}{isCommodity}}}{}
\pysigstopsignatures\begin{quote}\begin{description}
\sphinxlineitem{Arguments}\begin{itemize}
\item {} 
\sphinxAtStartPar
\sphinxstyleliteralstrong{\sphinxupquote{isCommodity}} (\sphinxstyleliteralemphasis{\sphinxupquote{bool}}) \textendash{} determines whether the process creation dialogue or the pro\_com\_prop creation dialogue is opened

\end{itemize}

\end{description}\end{quote}

\end{fulllineitems}

\index{closeNewProcessForm() (built\sphinxhyphen{}in function)@\spxentry{closeNewProcessForm()}\spxextra{built\sphinxhyphen{}in function}}

\begin{fulllineitems}
\phantomsection\label{\detokenize{docs_gui/js_api/urbs_editor/process_editor:closeNewProcessForm}}
\pysigstartsignatures
\pysiglinewithargsret{\sphinxbfcode{\sphinxupquote{\DUrole{n}{closeNewProcessForm}}}}{\emph{\DUrole{n}{isCommodity}}}{}
\pysigstopsignatures
\sphinxAtStartPar
onclick button for the forms’ cancel button, closes dialogue window and resets all input fields
\begin{quote}\begin{description}
\sphinxlineitem{Arguments}\begin{itemize}
\item {} 
\sphinxAtStartPar
\sphinxstyleliteralstrong{\sphinxupquote{isCommodity}} (\sphinxstyleliteralemphasis{\sphinxupquote{bool}}) \textendash{} determines which form needs to be closed

\end{itemize}

\end{description}\end{quote}

\end{fulllineitems}

\index{newProcessFormVerifyInputs() (built\sphinxhyphen{}in function)@\spxentry{newProcessFormVerifyInputs()}\spxextra{built\sphinxhyphen{}in function}}

\begin{fulllineitems}
\phantomsection\label{\detokenize{docs_gui/js_api/urbs_editor/process_editor:newProcessFormVerifyInputs}}
\pysigstartsignatures
\pysiglinewithargsret{\sphinxbfcode{\sphinxupquote{\DUrole{n}{newProcessFormVerifyInputs}}}}{\emph{\DUrole{n}{sel}}}{}
\pysigstopsignatures
\sphinxAtStartPar
onchange function for the input fields of the process creation popup form
makes sure “Create Process” button is disabled until all inputs are set correctly
\begin{quote}\begin{description}
\sphinxlineitem{Arguments}\begin{itemize}
\item {} 
\sphinxAtStartPar
\sphinxstyleliteralstrong{\sphinxupquote{sel}} (\sphinxstyleliteralemphasis{\sphinxupquote{HTML\_select\_element}}) \textendash{} the commodity select element

\end{itemize}

\end{description}\end{quote}

\end{fulllineitems}

\index{processAddCommoditySelection() (built\sphinxhyphen{}in function)@\spxentry{processAddCommoditySelection()}\spxextra{built\sphinxhyphen{}in function}}

\begin{fulllineitems}
\phantomsection\label{\detokenize{docs_gui/js_api/urbs_editor/process_editor:processAddCommoditySelection}}
\pysigstartsignatures
\pysiglinewithargsret{\sphinxbfcode{\sphinxupquote{\DUrole{n}{processAddCommoditySelection}}}}{\emph{\DUrole{n}{sel}}}{}
\pysigstopsignatures
\sphinxAtStartPar
onchange function for the input fields of the popup form for adding a commodity to a process
makes sure “Create Process” button is disabled until all inputs are set correctly
\begin{quote}\begin{description}
\sphinxlineitem{Arguments}\begin{itemize}
\item {} 
\sphinxAtStartPar
\sphinxstyleliteralstrong{\sphinxupquote{sel}} (\sphinxstyleliteralemphasis{\sphinxupquote{HTML\_select\_element}}) \textendash{} the commodity select element

\end{itemize}

\end{description}\end{quote}

\end{fulllineitems}

\index{openSecondaryProcessEditor() (built\sphinxhyphen{}in function)@\spxentry{openSecondaryProcessEditor()}\spxextra{built\sphinxhyphen{}in function}}

\begin{fulllineitems}
\phantomsection\label{\detokenize{docs_gui/js_api/urbs_editor/process_editor:openSecondaryProcessEditor}}
\pysigstartsignatures
\pysiglinewithargsret{\sphinxbfcode{\sphinxupquote{\DUrole{n}{openSecondaryProcessEditor}}}}{\emph{\DUrole{n}{sel}}, \emph{\DUrole{n}{secondaryFeatureName}}}{}
\pysigstopsignatures
\sphinxAtStartPar
Function makes secondary feature window visible and fills all input fields with the saved values, if any exist.
At the moment the process editor is the only one with a secondary editor, namely the pro\_com\_prop editor
\begin{quote}\begin{description}
\sphinxlineitem{Arguments}\begin{itemize}
\item {} 
\sphinxAtStartPar
\sphinxstyleliteralstrong{\sphinxupquote{sel}} (\sphinxstyleliteralemphasis{\sphinxupquote{HTML\_select\_element}}) \textendash{} gets passed as “this” reference when the onchange method for the select element is called, needed to retrieve the currently selected secondary feature

\item {} 
\sphinxAtStartPar
\sphinxstyleliteralstrong{\sphinxupquote{secondaryFeatureName}} (\sphinxstyleliteralemphasis{\sphinxupquote{string}}) \textendash{} key for relevant html elements

\end{itemize}

\end{description}\end{quote}

\end{fulllineitems}

\index{createNewProcess() (built\sphinxhyphen{}in function)@\spxentry{createNewProcess()}\spxextra{built\sphinxhyphen{}in function}}

\begin{fulllineitems}
\phantomsection\label{\detokenize{docs_gui/js_api/urbs_editor/process_editor:createNewProcess}}
\pysigstartsignatures
\pysiglinewithargsret{\sphinxbfcode{\sphinxupquote{\DUrole{n}{createNewProcess}}}}{}{}
\pysigstopsignatures
\sphinxAtStartPar
onclick function of the “Add Process” button. Calls helper function that adds process to process list and pro\_conf table
as well as attachment of process commodity

\end{fulllineitems}

\index{addCommodityToProcess() (built\sphinxhyphen{}in function)@\spxentry{addCommodityToProcess()}\spxextra{built\sphinxhyphen{}in function}}

\begin{fulllineitems}
\phantomsection\label{\detokenize{docs_gui/js_api/urbs_editor/process_editor:addCommodityToProcess}}
\pysigstartsignatures
\pysiglinewithargsret{\sphinxbfcode{\sphinxupquote{\DUrole{n}{addCommodityToProcess}}}}{}{}
\pysigstopsignatures
\sphinxAtStartPar
adds process commodity to the pro\_com secondary feature editor and calls function creating a new commodity, if the user chooses to create an entirely
new commodity to attach to the process

\end{fulllineitems}

\index{createNewProcessProperty() (built\sphinxhyphen{}in function)@\spxentry{createNewProcessProperty()}\spxextra{built\sphinxhyphen{}in function}}

\begin{fulllineitems}
\phantomsection\label{\detokenize{docs_gui/js_api/urbs_editor/process_editor:createNewProcessProperty}}
\pysigstartsignatures
\pysiglinewithargsret{\sphinxbfcode{\sphinxupquote{\DUrole{n}{createNewProcessProperty}}}}{\emph{\DUrole{n}{name}}}{}
\pysigstopsignatures
\sphinxAtStartPar
creates a new process and adds it to the ProcessObject \& html pro\_prop select
\begin{quote}\begin{description}
\sphinxlineitem{Arguments}\begin{itemize}
\item {} 
\sphinxAtStartPar
\sphinxstyleliteralstrong{\sphinxupquote{name}} (\sphinxstyleliteralemphasis{\sphinxupquote{string}}) \textendash{} name of the new process

\end{itemize}

\end{description}\end{quote}

\end{fulllineitems}

\index{createNewProcessCommodity() (built\sphinxhyphen{}in function)@\spxentry{createNewProcessCommodity()}\spxextra{built\sphinxhyphen{}in function}}

\begin{fulllineitems}
\phantomsection\label{\detokenize{docs_gui/js_api/urbs_editor/process_editor:createNewProcessCommodity}}
\pysigstartsignatures
\pysiglinewithargsret{\sphinxbfcode{\sphinxupquote{\DUrole{n}{createNewProcessCommodity}}}}{\emph{\DUrole{n}{pro\_name}}, \emph{\DUrole{n}{com\_name}}, \emph{\DUrole{n}{inOrOut}}}{}
\pysigstopsignatures
\sphinxAtStartPar
we must add a newly defined process commodity to the list of commodities, the process\_config table and the editor window of the process the commodity is associated with
\begin{quote}\begin{description}
\sphinxlineitem{Arguments}\begin{itemize}
\item {} 
\sphinxAtStartPar
\sphinxstyleliteralstrong{\sphinxupquote{pro\_name}} (\sphinxstyleliteralemphasis{\sphinxupquote{string}}) \textendash{} name of the process the commodity is attached tos

\item {} 
\sphinxAtStartPar
\sphinxstyleliteralstrong{\sphinxupquote{com\_name}} (\sphinxstyleliteralemphasis{\sphinxupquote{string}}) \textendash{} name of the new process commodity

\item {} 
\sphinxAtStartPar
\sphinxstyleliteralstrong{\sphinxupquote{inOrOut}} (\sphinxstyleliteralemphasis{\sphinxupquote{string}}) \textendash{} type of the new process commodity

\end{itemize}

\end{description}\end{quote}

\end{fulllineitems}

\index{writeBackProcessFeatures() (built\sphinxhyphen{}in function)@\spxentry{writeBackProcessFeatures()}\spxextra{built\sphinxhyphen{}in function}}

\begin{fulllineitems}
\phantomsection\label{\detokenize{docs_gui/js_api/urbs_editor/process_editor:writeBackProcessFeatures}}
\pysigstartsignatures
\pysiglinewithargsret{\sphinxbfcode{\sphinxupquote{\DUrole{n}{writeBackProcessFeatures}}}}{\emph{\DUrole{n}{target}}, \emph{\DUrole{n}{isPro\_com\_prop}}}{}
\pysigstopsignatures
\sphinxAtStartPar
saves edited feature in the ProcessObject
\begin{quote}\begin{description}
\sphinxlineitem{Arguments}\begin{itemize}
\item {} 
\sphinxAtStartPar
\sphinxstyleliteralstrong{\sphinxupquote{target}} (\sphinxstyleliteralemphasis{\sphinxupquote{event\_target\_object}}) \textendash{} 

\item {} 
\sphinxAtStartPar
\sphinxstyleliteralstrong{\sphinxupquote{isPro\_com\_prop}} (\sphinxstyleliteralemphasis{\sphinxupquote{bool}}) \textendash{} 

\end{itemize}

\end{description}\end{quote}

\end{fulllineitems}

\index{createPro\_ConfEditor() (built\sphinxhyphen{}in function)@\spxentry{createPro\_ConfEditor()}\spxextra{built\sphinxhyphen{}in function}}

\begin{fulllineitems}
\phantomsection\label{\detokenize{docs_gui/js_api/urbs_editor/process_editor:createPro_ConfEditor}}
\pysigstartsignatures
\pysiglinewithargsret{\sphinxbfcode{\sphinxupquote{\DUrole{n}{createPro\_ConfEditor}}}}{}{}
\pysigstopsignatures
\sphinxAtStartPar
sets up pro\_conf handsontable table in the GUI

\end{fulllineitems}


\sphinxstepscope


\subparagraph{return\_urbs\_setup.js}
\label{\detokenize{docs_gui/js_api/urbs_editor/return_urbs_setup:return-urbs-setup-js}}\label{\detokenize{docs_gui/js_api/urbs_editor/return_urbs_setup::doc}}\index{returnUrbsSetup() (built\sphinxhyphen{}in function)@\spxentry{returnUrbsSetup()}\spxextra{built\sphinxhyphen{}in function}}

\begin{fulllineitems}
\phantomsection\label{\detokenize{docs_gui/js_api/urbs_editor/return_urbs_setup:returnUrbsSetup}}
\pysigstartsignatures
\pysiglinewithargsret{\sphinxbfcode{\sphinxupquote{\DUrole{n}{returnUrbsSetup}}}}{}{}
\pysigstopsignatures
\sphinxAtStartPar
returns all urbs input values step by step, waits until all have been returned and signals to the backend to execute pdp2urbs

\end{fulllineitems}

\index{postData() (built\sphinxhyphen{}in function)@\spxentry{postData()}\spxextra{built\sphinxhyphen{}in function}}

\begin{fulllineitems}
\phantomsection\label{\detokenize{docs_gui/js_api/urbs_editor/return_urbs_setup:postData}}
\pysigstartsignatures
\pysiglinewithargsret{\sphinxbfcode{\sphinxupquote{\DUrole{n}{postData}}}}{\emph{\DUrole{n}{url}}, \emph{\DUrole{n}{jsonData}}}{}
\pysigstopsignatures
\sphinxAtStartPar
returns data of one setup category to the backend
\begin{quote}\begin{description}
\sphinxlineitem{Arguments}\begin{itemize}
\item {} 
\sphinxAtStartPar
\sphinxstyleliteralstrong{\sphinxupquote{url}} (\sphinxstyleliteralemphasis{\sphinxupquote{string}}) \textendash{} address we send data to on the backend side

\item {} 
\sphinxAtStartPar
\sphinxstyleliteralstrong{\sphinxupquote{jsonData}} (\sphinxstyleliteralemphasis{\sphinxupquote{json}}) \textendash{} data we want to return to the backend

\end{itemize}

\sphinxlineitem{Returns}
\sphinxAtStartPar
Promise signifying that the data was correctly received and processed in the backend

\end{description}\end{quote}

\end{fulllineitems}


\sphinxstepscope


\subparagraph{storage\_editor.js}
\label{\detokenize{docs_gui/js_api/urbs_editor/storage_editor:storage-editor-js}}\label{\detokenize{docs_gui/js_api/urbs_editor/storage_editor::doc}}\index{fetchStorageProfiles() (built\sphinxhyphen{}in function)@\spxentry{fetchStorageProfiles()}\spxextra{built\sphinxhyphen{}in function}}

\begin{fulllineitems}
\phantomsection\label{\detokenize{docs_gui/js_api/urbs_editor/storage_editor:fetchStorageProfiles}}
\pysigstartsignatures
\pysiglinewithargsret{\sphinxbfcode{\sphinxupquote{\DUrole{n}{fetchStorageProfiles}}}}{}{}
\pysigstopsignatures
\sphinxAtStartPar
called from generate\_urbs\_setup\_editor.js during setup of the urbs setup editor window
\begin{quote}\begin{description}
\sphinxlineitem{Returns}
\sphinxAtStartPar
Promise signalling that the fetch operation concluded

\end{description}\end{quote}

\end{fulllineitems}

\index{fillStorageEditorCommodityList() (built\sphinxhyphen{}in function)@\spxentry{fillStorageEditorCommodityList()}\spxextra{built\sphinxhyphen{}in function}}

\begin{fulllineitems}
\phantomsection\label{\detokenize{docs_gui/js_api/urbs_editor/storage_editor:fillStorageEditorCommodityList}}
\pysigstartsignatures
\pysiglinewithargsret{\sphinxbfcode{\sphinxupquote{\DUrole{n}{fillStorageEditorCommodityList}}}}{\emph{\DUrole{n}{commodities}}}{}
\pysigstopsignatures
\sphinxAtStartPar
creates html select options for all commodites in the storage editor window
\begin{quote}\begin{description}
\sphinxlineitem{Arguments}\begin{itemize}
\item {} 
\sphinxAtStartPar
\sphinxstyleliteralstrong{\sphinxupquote{commodities}} (\sphinxstyleliteralemphasis{\sphinxupquote{list}}) \textendash{} contains names of all commodities

\end{itemize}

\end{description}\end{quote}

\end{fulllineitems}

\index{createSto\_ConfEditor() (built\sphinxhyphen{}in function)@\spxentry{createSto\_ConfEditor()}\spxextra{built\sphinxhyphen{}in function}}

\begin{fulllineitems}
\phantomsection\label{\detokenize{docs_gui/js_api/urbs_editor/storage_editor:createSto_ConfEditor}}
\pysigstartsignatures
\pysiglinewithargsret{\sphinxbfcode{\sphinxupquote{\DUrole{n}{createSto\_ConfEditor}}}}{}{}
\pysigstopsignatures
\sphinxAtStartPar
sets up handsontable html element for sto\_conf

\end{fulllineitems}


\sphinxstepscope


\subparagraph{supim\_editor.js}
\label{\detokenize{docs_gui/js_api/urbs_editor/supim_editor:supim-editor-js}}\label{\detokenize{docs_gui/js_api/urbs_editor/supim_editor::doc}}\index{fetchSupimProfiles() (built\sphinxhyphen{}in function)@\spxentry{fetchSupimProfiles()}\spxextra{built\sphinxhyphen{}in function}}

\begin{fulllineitems}
\phantomsection\label{\detokenize{docs_gui/js_api/urbs_editor/supim_editor:fetchSupimProfiles}}
\pysigstartsignatures
\pysiglinewithargsret{\sphinxbfcode{\sphinxupquote{\DUrole{n}{fetchSupimProfiles}}}}{}{}
\pysigstopsignatures
\sphinxAtStartPar
retrieves supim profiles
\begin{quote}\begin{description}
\sphinxlineitem{Returns}
\sphinxAtStartPar
Promise to make sure functions during setup are called in order

\end{description}\end{quote}

\end{fulllineitems}

\index{populateSupimEditor() (built\sphinxhyphen{}in function)@\spxentry{populateSupimEditor()}\spxextra{built\sphinxhyphen{}in function}}

\begin{fulllineitems}
\phantomsection\label{\detokenize{docs_gui/js_api/urbs_editor/supim_editor:populateSupimEditor}}
\pysigstartsignatures
\pysiglinewithargsret{\sphinxbfcode{\sphinxupquote{\DUrole{n}{populateSupimEditor}}}}{\emph{\DUrole{n}{data}}, \emph{\DUrole{n}{name}}, \emph{\DUrole{n}{index}}}{}
\pysigstopsignatures
\sphinxAtStartPar
TODO: Generalize this function
generates html elements for the checkboxes
\begin{quote}\begin{description}
\sphinxlineitem{Arguments}\begin{itemize}
\item {} 
\sphinxAtStartPar
\sphinxstyleliteralstrong{\sphinxupquote{data}} (\sphinxstyleliteralemphasis{\sphinxupquote{list}}) \textendash{} list containing all possible time series of a profile, could be swapped out with just length value tbh

\item {} 
\sphinxAtStartPar
\sphinxstyleliteralstrong{\sphinxupquote{name}} (\sphinxstyleliteralemphasis{\sphinxupquote{string}}) \textendash{} key for getting the correct html div element and setting the checkbox onclick function

\item {} 
\sphinxAtStartPar
\sphinxstyleliteralstrong{\sphinxupquote{index}} (\sphinxstyleliteralemphasis{\sphinxupquote{int}}) \textendash{} secondary key for setting the checkbox onclick function

\end{itemize}

\end{description}\end{quote}

\end{fulllineitems}

\index{fillSelectedFeatureSupimEditor() (built\sphinxhyphen{}in function)@\spxentry{fillSelectedFeatureSupimEditor()}\spxextra{built\sphinxhyphen{}in function}}

\begin{fulllineitems}
\phantomsection\label{\detokenize{docs_gui/js_api/urbs_editor/supim_editor:fillSelectedFeatureSupimEditor}}
\pysigstartsignatures
\pysiglinewithargsret{\sphinxbfcode{\sphinxupquote{\DUrole{n}{fillSelectedFeatureSupimEditor}}}}{\emph{\DUrole{n}{target}}}{}
\pysigstopsignatures
\sphinxAtStartPar
when the user selects an element in the ui in the supim tab, this function sets all checkboxes for that element to the right value
\begin{quote}\begin{description}
\sphinxlineitem{Arguments}\begin{itemize}
\item {} 
\sphinxAtStartPar
\sphinxstyleliteralstrong{\sphinxupquote{target}} (\sphinxstyleliteralemphasis{\sphinxupquote{event\_target\_object}}) \textendash{} html object whose onchange event triggered the function call

\end{itemize}

\end{description}\end{quote}

\end{fulllineitems}


\sphinxstepscope


\subparagraph{timevareff\_editor.js}
\label{\detokenize{docs_gui/js_api/urbs_editor/timevareff_editor:timevareff-editor-js}}\label{\detokenize{docs_gui/js_api/urbs_editor/timevareff_editor::doc}}\index{fetchFeatureProfiles() (built\sphinxhyphen{}in function)@\spxentry{fetchFeatureProfiles()}\spxextra{built\sphinxhyphen{}in function}}

\begin{fulllineitems}
\phantomsection\label{\detokenize{docs_gui/js_api/urbs_editor/timevareff_editor:fetchFeatureProfiles}}
\pysigstartsignatures
\pysiglinewithargsret{\sphinxbfcode{\sphinxupquote{\DUrole{n}{fetchFeatureProfiles}}}}{}{}
\pysigstopsignatures
\sphinxAtStartPar
retrieves timevareff profiles
\begin{quote}\begin{description}
\sphinxlineitem{Returns}
\sphinxAtStartPar
Promise to make sure functions during setup are called in order

\end{description}\end{quote}

\end{fulllineitems}

\index{fillSelectedFeatureTimevareffEditor() (built\sphinxhyphen{}in function)@\spxentry{fillSelectedFeatureTimevareffEditor()}\spxextra{built\sphinxhyphen{}in function}}

\begin{fulllineitems}
\phantomsection\label{\detokenize{docs_gui/js_api/urbs_editor/timevareff_editor:fillSelectedFeatureTimevareffEditor}}
\pysigstartsignatures
\pysiglinewithargsret{\sphinxbfcode{\sphinxupquote{\DUrole{n}{fillSelectedFeatureTimevareffEditor}}}}{\emph{\DUrole{n}{target}}}{}
\pysigstopsignatures
\sphinxAtStartPar
TODO: generalize this function probably
when the user selects an element in the ui in the timevareff tab, this function sets all checkboxes for that element to the right value
\begin{quote}\begin{description}
\sphinxlineitem{Arguments}\begin{itemize}
\item {} 
\sphinxAtStartPar
\sphinxstyleliteralstrong{\sphinxupquote{target}} (\sphinxstyleliteralemphasis{\sphinxupquote{event\_target\_object}}) \textendash{} html object whose onchange event triggered the function call

\end{itemize}

\end{description}\end{quote}

\end{fulllineitems}

\index{check\_uncheck\_timevareff() (built\sphinxhyphen{}in function)@\spxentry{check\_uncheck\_timevareff()}\spxextra{built\sphinxhyphen{}in function}}

\begin{fulllineitems}
\phantomsection\label{\detokenize{docs_gui/js_api/urbs_editor/timevareff_editor:check_uncheck_timevareff}}
\pysigstartsignatures
\pysiglinewithargsret{\sphinxbfcode{\sphinxupquote{\DUrole{n}{check\_uncheck\_timevareff}}}}{\emph{\DUrole{n}{checkbox}}, \emph{\DUrole{n}{demand\_type}}, \emph{\DUrole{n}{key}}, \emph{\DUrole{n}{demandIndex}}}{}
\pysigstopsignatures
\sphinxAtStartPar
TODO: generalize function or at least rename it (needs to generalize select Id and FeatureObject)
removes or adds a graph to the demand chart and marks whether the checkbox is set in the DemandObject
\begin{quote}\begin{description}
\sphinxlineitem{Arguments}\begin{itemize}
\item {} 
\sphinxAtStartPar
\sphinxstyleliteralstrong{\sphinxupquote{checkbox}} (\sphinxstyleliteralemphasis{\sphinxupquote{HTML\_element}}) \textendash{} checkbox html element whose onchange event triggered the function calls

\item {} 
\sphinxAtStartPar
\sphinxstyleliteralstrong{\sphinxupquote{demand\_type}} (\sphinxstyleliteralemphasis{\sphinxupquote{string}}) \textendash{} which demand category the checkbox belongs to

\item {} 
\sphinxAtStartPar
\sphinxstyleliteralstrong{\sphinxupquote{key}} (\sphinxstyleliteralemphasis{\sphinxupquote{int}}) \textendash{} the index of the checkbox in the list

\item {} 
\sphinxAtStartPar
\sphinxstyleliteralstrong{\sphinxupquote{demandIndex}} (\sphinxstyleliteralemphasis{\sphinxupquote{int}}) \textendash{} the index of the demand category

\end{itemize}

\end{description}\end{quote}

\end{fulllineitems}

\index{populatTimevareffEditor() (built\sphinxhyphen{}in function)@\spxentry{populatTimevareffEditor()}\spxextra{built\sphinxhyphen{}in function}}

\begin{fulllineitems}
\phantomsection\label{\detokenize{docs_gui/js_api/urbs_editor/timevareff_editor:populatTimevareffEditor}}
\pysigstartsignatures
\pysiglinewithargsret{\sphinxbfcode{\sphinxupquote{\DUrole{n}{populatTimevareffEditor}}}}{\emph{\DUrole{n}{data}}, \emph{\DUrole{n}{name}}, \emph{\DUrole{n}{index}}}{}
\pysigstopsignatures
\sphinxAtStartPar
TODO: Generalize this function
generates html elements for the checkboxes
\begin{quote}\begin{description}
\sphinxlineitem{Arguments}\begin{itemize}
\item {} 
\sphinxAtStartPar
\sphinxstyleliteralstrong{\sphinxupquote{data}} (\sphinxstyleliteralemphasis{\sphinxupquote{list}}) \textendash{} list containing all possible time series of a profile, could be swapped out with just length value tbh

\item {} 
\sphinxAtStartPar
\sphinxstyleliteralstrong{\sphinxupquote{name}} (\sphinxstyleliteralemphasis{\sphinxupquote{string}}) \textendash{} key for getting the correct html div element and setting the checkbox onclick function

\item {} 
\sphinxAtStartPar
\sphinxstyleliteralstrong{\sphinxupquote{index}} (\sphinxstyleliteralemphasis{\sphinxupquote{int}}) \textendash{} secondary key for setting the checkbox onclick function

\end{itemize}

\end{description}\end{quote}

\end{fulllineitems}


\sphinxstepscope


\subparagraph{transmission\_editor.js}
\label{\detokenize{docs_gui/js_api/urbs_editor/transmission_editor:transmission-editor-js}}\label{\detokenize{docs_gui/js_api/urbs_editor/transmission_editor::doc}}\index{fetchTransmissionProfiles() (built\sphinxhyphen{}in function)@\spxentry{fetchTransmissionProfiles()}\spxextra{built\sphinxhyphen{}in function}}

\begin{fulllineitems}
\phantomsection\label{\detokenize{docs_gui/js_api/urbs_editor/transmission_editor:fetchTransmissionProfiles}}
\pysigstartsignatures
\pysiglinewithargsret{\sphinxbfcode{\sphinxupquote{\DUrole{n}{fetchTransmissionProfiles}}}}{}{}
\pysigstopsignatures
\sphinxAtStartPar
Called from generate\_urbs\_setup\_editor.js during setup of the urbs setup editor window.
We fetch default values from the backend.
the fetched data contains default values for trafo\_data as well as
the sn\_mva value of the network trafo to generate the kont option for trafo data
\begin{quote}\begin{description}
\sphinxlineitem{Returns}
\sphinxAtStartPar
Promise signalling that the fetch operation concluded

\end{description}\end{quote}

\end{fulllineitems}

\index{prepareCableDataList() (built\sphinxhyphen{}in function)@\spxentry{prepareCableDataList()}\spxextra{built\sphinxhyphen{}in function}}

\begin{fulllineitems}
\phantomsection\label{\detokenize{docs_gui/js_api/urbs_editor/transmission_editor:prepareCableDataList}}
\pysigstartsignatures
\pysiglinewithargsret{\sphinxbfcode{\sphinxupquote{\DUrole{n}{prepareCableDataList}}}}{\emph{\DUrole{n}{TransmissionPropertiesJSON}}, \emph{\DUrole{n}{listName}}}{}
\pysigstopsignatures
\sphinxAtStartPar
prefills the TransmissionObject
\begin{quote}\begin{description}
\sphinxlineitem{Arguments}\begin{itemize}
\item {} 
\sphinxAtStartPar
\sphinxstyleliteralstrong{\sphinxupquote{TransmissionPropertiesJSON}} (\sphinxstyleliteralemphasis{\sphinxupquote{dict}}) \textendash{} dict containing all data for the transmission object

\item {} 
\sphinxAtStartPar
\sphinxstyleliteralstrong{\sphinxupquote{listName}} (\sphinxstyleliteralemphasis{\sphinxupquote{string}}) \textendash{} name of the list in the JSOn whose features we want to extract

\end{itemize}

\end{description}\end{quote}

\end{fulllineitems}

\index{populateTransmissionEditorList() (built\sphinxhyphen{}in function)@\spxentry{populateTransmissionEditorList()}\spxextra{built\sphinxhyphen{}in function}}

\begin{fulllineitems}
\phantomsection\label{\detokenize{docs_gui/js_api/urbs_editor/transmission_editor:populateTransmissionEditorList}}
\pysigstartsignatures
\pysiglinewithargsret{\sphinxbfcode{\sphinxupquote{\DUrole{n}{populateTransmissionEditorList}}}}{\emph{\DUrole{n}{UrbsPropertiesJSON}}}{}
\pysigstopsignatures
\sphinxAtStartPar
called from generate\_urbs\_setup\_editor.js during setup of the urbs setup editor window
Creates select options for cable\_data, trafo\_data and voltage\_limits
\begin{quote}\begin{description}
\sphinxlineitem{Arguments}\begin{itemize}
\item {} 
\sphinxAtStartPar
\sphinxstyleliteralstrong{\sphinxupquote{UrbsPropertiesJSON}} (\sphinxstyleliteralemphasis{\sphinxupquote{dict}}) \textendash{} json containing all features and their properties for all input categories of the urbs setup

\end{itemize}

\end{description}\end{quote}

\end{fulllineitems}

\index{fillTrafoDataEditorIdSelect() (built\sphinxhyphen{}in function)@\spxentry{fillTrafoDataEditorIdSelect()}\spxextra{built\sphinxhyphen{}in function}}

\begin{fulllineitems}
\phantomsection\label{\detokenize{docs_gui/js_api/urbs_editor/transmission_editor:fillTrafoDataEditorIdSelect}}
\pysigstartsignatures
\pysiglinewithargsret{\sphinxbfcode{\sphinxupquote{\DUrole{n}{fillTrafoDataEditorIdSelect}}}}{}{}
\pysigstopsignatures
\sphinxAtStartPar
after the editor form for trafo\_data has been selected, we add the kont and ronts as options to the id selection

\end{fulllineitems}

\index{writeBackTransmissionFeatures() (built\sphinxhyphen{}in function)@\spxentry{writeBackTransmissionFeatures()}\spxextra{built\sphinxhyphen{}in function}}

\begin{fulllineitems}
\phantomsection\label{\detokenize{docs_gui/js_api/urbs_editor/transmission_editor:writeBackTransmissionFeatures}}
\pysigstartsignatures
\pysiglinewithargsret{\sphinxbfcode{\sphinxupquote{\DUrole{n}{writeBackTransmissionFeatures}}}}{\emph{\DUrole{n}{target}}}{}
\pysigstopsignatures
\sphinxAtStartPar
Saves changed inputs in the corresponding feature object within the TransmissionObject and, if the changed input was
the id select, calls a function that adjusts all other fields to match the id
\begin{quote}\begin{description}
\sphinxlineitem{Arguments}\begin{itemize}
\item {} 
\sphinxAtStartPar
\sphinxstyleliteralstrong{\sphinxupquote{target}} (\sphinxstyleliteralemphasis{\sphinxupquote{HTML\_element}}) \textendash{} reference to the changed html input element

\end{itemize}

\end{description}\end{quote}

\end{fulllineitems}

\index{fillInputFieldsOfSelectedID() (built\sphinxhyphen{}in function)@\spxentry{fillInputFieldsOfSelectedID()}\spxextra{built\sphinxhyphen{}in function}}

\begin{fulllineitems}
\phantomsection\label{\detokenize{docs_gui/js_api/urbs_editor/transmission_editor:fillInputFieldsOfSelectedID}}
\pysigstartsignatures
\pysiglinewithargsret{\sphinxbfcode{\sphinxupquote{\DUrole{n}{fillInputFieldsOfSelectedID}}}}{\emph{\DUrole{n}{id}}}{}
\pysigstopsignatures
\sphinxAtStartPar
If the Select element in the transmission editor changes, all other fields are updated with the values corresponding to the newly selected element
\begin{quote}\begin{description}
\sphinxlineitem{Arguments}\begin{itemize}
\item {} 
\sphinxAtStartPar
\sphinxstyleliteralstrong{\sphinxupquote{id}} (\sphinxstyleliteralemphasis{\sphinxupquote{string}}) \textendash{} key for TransmissionObject list element that contains values for a given id

\end{itemize}

\end{description}\end{quote}

\end{fulllineitems}

\index{openNewTrafoDataForm() (built\sphinxhyphen{}in function)@\spxentry{openNewTrafoDataForm()}\spxextra{built\sphinxhyphen{}in function}}

\begin{fulllineitems}
\phantomsection\label{\detokenize{docs_gui/js_api/urbs_editor/transmission_editor:openNewTrafoDataForm}}
\pysigstartsignatures
\pysiglinewithargsret{\sphinxbfcode{\sphinxupquote{\DUrole{n}{openNewTrafoDataForm}}}}{}{}
\pysigstopsignatures
\sphinxAtStartPar
onclick function for the \#newTrafo\_dataButton
opens the form for creating a new ront

\end{fulllineitems}

\index{closeNewTrafoDataForm() (built\sphinxhyphen{}in function)@\spxentry{closeNewTrafoDataForm()}\spxextra{built\sphinxhyphen{}in function}}

\begin{fulllineitems}
\phantomsection\label{\detokenize{docs_gui/js_api/urbs_editor/transmission_editor:closeNewTrafoDataForm}}
\pysigstartsignatures
\pysiglinewithargsret{\sphinxbfcode{\sphinxupquote{\DUrole{n}{closeNewTrafoDataForm}}}}{}{}
\pysigstopsignatures
\sphinxAtStartPar
onclick function for the cancel and accept button of the \#urbsNewTrafoDataPopupForm
closes the form for creating a new ront and resets its input field

\end{fulllineitems}

\index{trafoDataFormCheckValidInput() (built\sphinxhyphen{}in function)@\spxentry{trafoDataFormCheckValidInput()}\spxextra{built\sphinxhyphen{}in function}}

\begin{fulllineitems}
\phantomsection\label{\detokenize{docs_gui/js_api/urbs_editor/transmission_editor:trafoDataFormCheckValidInput}}
\pysigstartsignatures
\pysiglinewithargsret{\sphinxbfcode{\sphinxupquote{\DUrole{n}{trafoDataFormCheckValidInput}}}}{\emph{\DUrole{n}{text}}}{}
\pysigstopsignatures
\sphinxAtStartPar
onchange method for the \#newTrafoDataTextInput
checks that the sn\_mva value the user puts in is valid and that a value has been given at all before
enabling the accept button of the creation form
\begin{quote}\begin{description}
\sphinxlineitem{Arguments}\begin{itemize}
\item {} 
\sphinxAtStartPar
\sphinxstyleliteralstrong{\sphinxupquote{text}} (\sphinxstyleliteralemphasis{\sphinxupquote{string}}) \textendash{} value in the input field

\end{itemize}

\end{description}\end{quote}

\end{fulllineitems}

\index{createNewTrafoData() (built\sphinxhyphen{}in function)@\spxentry{createNewTrafoData()}\spxextra{built\sphinxhyphen{}in function}}

\begin{fulllineitems}
\phantomsection\label{\detokenize{docs_gui/js_api/urbs_editor/transmission_editor:createNewTrafoData}}
\pysigstartsignatures
\pysiglinewithargsret{\sphinxbfcode{\sphinxupquote{\DUrole{n}{createNewTrafoData}}}}{\emph{\DUrole{n}{sn\_mva}}}{}
\pysigstopsignatures
\sphinxAtStartPar
onclick function for the \#newTrafoDataCreateButton of the new trafo data form
creates a new ront and adds it to the TransmissionObject, trafo\_data editor id select
\begin{quote}\begin{description}
\sphinxlineitem{Arguments}\begin{itemize}
\item {} 
\sphinxAtStartPar
\sphinxstyleliteralstrong{\sphinxupquote{sn\_mva}} (\sphinxstyleliteralemphasis{\sphinxupquote{string}}) \textendash{} 

\end{itemize}

\end{description}\end{quote}

\end{fulllineitems}


\sphinxstepscope


\subparagraph{Urbs Results}
\label{\detokenize{docs_gui/js_api/urbs_results/index:urbs-results}}\label{\detokenize{docs_gui/js_api/urbs_results/index::doc}}
\sphinxstepscope


\subparagraph{setup\_urbs\_results.js}
\label{\detokenize{docs_gui/js_api/urbs_results/setup_urbs_results:setup-urbs-results-js}}\label{\detokenize{docs_gui/js_api/urbs_results/setup_urbs_results::doc}}\index{SetupUrbsResultEditor() (built\sphinxhyphen{}in function)@\spxentry{SetupUrbsResultEditor()}\spxextra{built\sphinxhyphen{}in function}}

\begin{fulllineitems}
\phantomsection\label{\detokenize{docs_gui/js_api/urbs_results/setup_urbs_results:SetupUrbsResultEditor}}
\pysigstartsignatures
\pysiglinewithargsret{\sphinxbfcode{\sphinxupquote{\DUrole{n}{SetupUrbsResultEditor}}}}{}{}
\pysigstopsignatures
\sphinxAtStartPar
retrieve network data and setup editor

\end{fulllineitems}

\index{displayUrbsNet() (built\sphinxhyphen{}in function)@\spxentry{displayUrbsNet()}\spxextra{built\sphinxhyphen{}in function}}

\begin{fulllineitems}
\phantomsection\label{\detokenize{docs_gui/js_api/urbs_results/setup_urbs_results:displayUrbsNet}}
\pysigstartsignatures
\pysiglinewithargsret{\sphinxbfcode{\sphinxupquote{\DUrole{n}{displayUrbsNet}}}}{\emph{\DUrole{n}{ppdata}}}{}
\pysigstopsignatures
\sphinxAtStartPar
aggregate function calling the actual functions that place the feature sgeojsons on the leaflet map
\begin{quote}\begin{description}
\sphinxlineitem{Arguments}\begin{itemize}
\item {} 
\sphinxAtStartPar
\sphinxstyleliteralstrong{\sphinxupquote{ppdata}} (\sphinxstyleliteralemphasis{\sphinxupquote{dict}}) \textendash{} grid data

\end{itemize}

\end{description}\end{quote}

\end{fulllineitems}

\index{addUrbsGeoJSONtoMap() (built\sphinxhyphen{}in function)@\spxentry{addUrbsGeoJSONtoMap()}\spxextra{built\sphinxhyphen{}in function}}

\begin{fulllineitems}
\phantomsection\label{\detokenize{docs_gui/js_api/urbs_results/setup_urbs_results:addUrbsGeoJSONtoMap}}
\pysigstartsignatures
\pysiglinewithargsret{\sphinxbfcode{\sphinxupquote{\DUrole{n}{addUrbsGeoJSONtoMap}}}}{\emph{\DUrole{n}{isLines}}, \emph{\DUrole{n}{input\_geoJSON}}, \emph{\DUrole{n}{featureName}}}{}
\pysigstopsignatures
\sphinxAtStartPar
function that adds a FeatureCollection to the leaflet map
we set styles and onclick functions here and save references to each added feature in the NetworkObject
lines (lines, trafos) and circlemarkers (buses, ext\_grids) need to be handled differently because lines do not have the pointToLayer function
\begin{quote}\begin{description}
\sphinxlineitem{Arguments}\begin{itemize}
\item {} 
\sphinxAtStartPar
\sphinxstyleliteralstrong{\sphinxupquote{isLines}} (\sphinxstyleliteralemphasis{\sphinxupquote{boolean}}) \textendash{} 

\item {} 
\sphinxAtStartPar
\sphinxstyleliteralstrong{\sphinxupquote{input\_geoJSON}} (\sphinxstyleliteralemphasis{\sphinxupquote{geoJSON\_FeatureCollection}}) \textendash{} 

\item {} 
\sphinxAtStartPar
\sphinxstyleliteralstrong{\sphinxupquote{featureName}} (\sphinxstyleliteralemphasis{\sphinxupquote{string}}) \textendash{} 

\end{itemize}

\end{description}\end{quote}

\end{fulllineitems}

\index{populateUrbsResultsLoadBusLists() (built\sphinxhyphen{}in function)@\spxentry{populateUrbsResultsLoadBusLists()}\spxextra{built\sphinxhyphen{}in function}}

\begin{fulllineitems}
\phantomsection\label{\detokenize{docs_gui/js_api/urbs_results/setup_urbs_results:populateUrbsResultsLoadBusLists}}
\pysigstartsignatures
\pysiglinewithargsret{\sphinxbfcode{\sphinxupquote{\DUrole{n}{populateUrbsResultsLoadBusLists}}}}{\emph{\DUrole{n}{htmlListName}}}{}
\pysigstopsignatures\begin{quote}\begin{description}
\sphinxlineitem{Arguments}\begin{itemize}
\item {} 
\sphinxAtStartPar
\sphinxstyleliteralstrong{\sphinxupquote{htmlListName}} (\sphinxstyleliteralemphasis{\sphinxupquote{string}}) \textendash{} name of the html select element that is supposed to be filled with the buses with attached loads

\end{itemize}

\end{description}\end{quote}

\end{fulllineitems}

\index{openUrbsNetworkList() (built\sphinxhyphen{}in function)@\spxentry{openUrbsNetworkList()}\spxextra{built\sphinxhyphen{}in function}}

\begin{fulllineitems}
\phantomsection\label{\detokenize{docs_gui/js_api/urbs_results/setup_urbs_results:openUrbsNetworkList}}
\pysigstartsignatures
\pysiglinewithargsret{\sphinxbfcode{\sphinxupquote{\DUrole{n}{openUrbsNetworkList}}}}{\emph{\DUrole{n}{e}}, \emph{\DUrole{n}{listName}}}{}
\pysigstopsignatures
\sphinxAtStartPar
gets called when one of the tablink buttons in the GUI gets pressed and opens the relevant feature list, while hiding all other GUI elements
editors are closed, lists are hidden, buttons are set to inactive
\begin{quote}\begin{description}
\sphinxlineitem{Arguments}\begin{itemize}
\item {} 
\sphinxAtStartPar
\sphinxstyleliteralstrong{\sphinxupquote{e}} (\sphinxstyleliteralemphasis{\sphinxupquote{event}}) \textendash{} event that triggered the function call

\item {} 
\sphinxAtStartPar
\sphinxstyleliteralstrong{\sphinxupquote{listName}} (\sphinxstyleliteralemphasis{\sphinxupquote{string}}) \textendash{} name of the element to be opened

\end{itemize}

\end{description}\end{quote}

\end{fulllineitems}

\index{fillSelectedUrbsNetworkFeatureEditor() (built\sphinxhyphen{}in function)@\spxentry{fillSelectedUrbsNetworkFeatureEditor()}\spxextra{built\sphinxhyphen{}in function}}

\begin{fulllineitems}
\phantomsection\label{\detokenize{docs_gui/js_api/urbs_results/setup_urbs_results:fillSelectedUrbsNetworkFeatureEditor}}
\pysigstartsignatures
\pysiglinewithargsret{\sphinxbfcode{\sphinxupquote{\DUrole{n}{fillSelectedUrbsNetworkFeatureEditor}}}}{\emph{\DUrole{n}{sel}}, \emph{\DUrole{n}{listName}}}{}
\pysigstopsignatures
\sphinxAtStartPar
onchange method for the feature lists in the GUI window
picks the corresponding map object for the selected list element to execute the clickonmarker function on
\begin{quote}\begin{description}
\sphinxlineitem{Arguments}\begin{itemize}
\item {} 
\sphinxAtStartPar
\sphinxstyleliteralstrong{\sphinxupquote{sel}} (\sphinxstyleliteralemphasis{\sphinxupquote{HTML\_select\_object}}) \textendash{} reference to the select element that just changed

\item {} 
\sphinxAtStartPar
\sphinxstyleliteralstrong{\sphinxupquote{listName}} (\sphinxstyleliteralemphasis{\sphinxupquote{string}}) \textendash{} key for NetworkObject list

\end{itemize}

\end{description}\end{quote}

\end{fulllineitems}

\index{resetUrbsStyle() (built\sphinxhyphen{}in function)@\spxentry{resetUrbsStyle()}\spxextra{built\sphinxhyphen{}in function}}

\begin{fulllineitems}
\phantomsection\label{\detokenize{docs_gui/js_api/urbs_results/setup_urbs_results:resetUrbsStyle}}
\pysigstartsignatures
\pysiglinewithargsret{\sphinxbfcode{\sphinxupquote{\DUrole{n}{resetUrbsStyle}}}}{\emph{\DUrole{n}{target}}, \emph{\DUrole{n}{feature}}}{}
\pysigstopsignatures
\sphinxAtStartPar
resets the style of the previously selected feature
\begin{quote}\begin{description}
\sphinxlineitem{Arguments}\begin{itemize}
\item {} 
\sphinxAtStartPar
\sphinxstyleliteralstrong{\sphinxupquote{target}} (\sphinxstyleliteralemphasis{\sphinxupquote{HTML\_element}}) \textendash{} the map object whose style we want to change

\item {} 
\sphinxAtStartPar
\sphinxstyleliteralstrong{\sphinxupquote{feature}} (\sphinxstyleliteralemphasis{\sphinxupquote{string}}) \textendash{} the name of the map object type

\end{itemize}

\end{description}\end{quote}

\end{fulllineitems}

\index{clickOnUrbsMarker() (built\sphinxhyphen{}in function)@\spxentry{clickOnUrbsMarker()}\spxextra{built\sphinxhyphen{}in function}}

\begin{fulllineitems}
\phantomsection\label{\detokenize{docs_gui/js_api/urbs_results/setup_urbs_results:clickOnUrbsMarker}}
\pysigstartsignatures
\pysiglinewithargsret{\sphinxbfcode{\sphinxupquote{\DUrole{n}{clickOnUrbsMarker}}}}{\emph{\DUrole{n}{target}}, \emph{\DUrole{n}{feature}}}{}
\pysigstopsignatures
\sphinxAtStartPar
When clicking on a map element or making a selection from a list,
we highlight the relevant element, open the Editor window and fill its input fields with the relevant values
\begin{quote}\begin{description}
\sphinxlineitem{Arguments}\begin{itemize}
\item {} 
\sphinxAtStartPar
\sphinxstyleliteralstrong{\sphinxupquote{target}} (\sphinxstyleliteralemphasis{\sphinxupquote{HTML\_element}}) \textendash{} the object that has been interacted with on the map. Null in case we have clicked on a list element

\item {} 
\sphinxAtStartPar
\sphinxstyleliteralstrong{\sphinxupquote{feature}} (\sphinxstyleliteralemphasis{\sphinxupquote{string}}) \textendash{} key for accessing f.e. featurelist in the NetworkObject

\end{itemize}

\end{description}\end{quote}

\end{fulllineitems}

\index{getPlotOfFeature() (built\sphinxhyphen{}in function)@\spxentry{getPlotOfFeature()}\spxextra{built\sphinxhyphen{}in function}}

\begin{fulllineitems}
\phantomsection\label{\detokenize{docs_gui/js_api/urbs_results/setup_urbs_results:getPlotOfFeature}}
\pysigstartsignatures
\pysiglinewithargsret{\sphinxbfcode{\sphinxupquote{\DUrole{n}{getPlotOfFeature}}}}{\emph{\DUrole{n}{feature}}, \emph{\DUrole{n}{targetName}}}{}
\pysigstopsignatures
\sphinxAtStartPar
sends the feature type and the name of the feature we want to generate a plot for to the backend
\begin{quote}\begin{description}
\sphinxlineitem{Arguments}\begin{itemize}
\item {} 
\sphinxAtStartPar
\sphinxstyleliteralstrong{\sphinxupquote{feature}} (\sphinxstyleliteralemphasis{\sphinxupquote{string}}) \textendash{} to distinguish what type of plot we want to generate in the backend

\item {} 
\sphinxAtStartPar
\sphinxstyleliteralstrong{\sphinxupquote{targetName}} (\sphinxstyleliteralemphasis{\sphinxupquote{string}}) \textendash{} key needed to access data for a specific site in the hdf5 file

\end{itemize}

\end{description}\end{quote}

\end{fulllineitems}

\index{setInnerHTML() (built\sphinxhyphen{}in function)@\spxentry{setInnerHTML()}\spxextra{built\sphinxhyphen{}in function}}

\begin{fulllineitems}
\phantomsection\label{\detokenize{docs_gui/js_api/urbs_results/setup_urbs_results:setInnerHTML}}
\pysigstartsignatures
\pysiglinewithargsret{\sphinxbfcode{\sphinxupquote{\DUrole{n}{setInnerHTML}}}}{\emph{\DUrole{n}{elm}}, \emph{\DUrole{n}{html}}}{}
\pysigstopsignatures
\sphinxAtStartPar
makes sure inline script is executed once we add a plot to the GUI, which makes the plot visible and interactible
from  \sphinxurl{https://stackoverflow.com/questions/2592092/executing-script-elements-inserted-with-innerhtml}
\begin{quote}\begin{description}
\sphinxlineitem{Arguments}\begin{itemize}
\item {} 
\sphinxAtStartPar
\sphinxstyleliteralstrong{\sphinxupquote{elm}} (\sphinxstyleliteralemphasis{\sphinxupquote{HTML\_element}}) \textendash{} html div element we want to add the plot to

\item {} 
\sphinxAtStartPar
\sphinxstyleliteralstrong{\sphinxupquote{html}} (\sphinxstyleliteralemphasis{\sphinxupquote{*}}) \textendash{} html code we want to add to the div

\end{itemize}

\end{description}\end{quote}

\end{fulllineitems}



\subsubsection{Indices and tables}
\label{\detokenize{docs_gui/index:indices-and-tables}}\begin{itemize}
\item {} 
\sphinxAtStartPar
\DUrole{xref,std,std-ref}{genindex}

\item {} 
\sphinxAtStartPar
\DUrole{xref,std,std-ref}{modindex}

\item {} 
\sphinxAtStartPar
\DUrole{xref,std,std-ref}{search}

\end{itemize}



\renewcommand{\indexname}{Index}
\printindex
\end{document}